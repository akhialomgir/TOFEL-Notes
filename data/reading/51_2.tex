\section{TPO 51 Passage 02}

\subsection{Words}

\begin{tabular}{lll}
    conform & v. & 顺从;遵从;随大流,顺应习俗    \\
    endow   & v. & 向(院校、医院等)捐款,捐赠,资助 \\
    vent    & v. & 发泄,表达(负面的情绪)      \\
\end{tabular}

\subsection{Analyse}

\begin{blk}
    \begin{qst}
        Q3. Paragraph 1 supports all of the following statements about \textbf{fluids} EXCEPT:
    \end{qst}

    \begin{chc}
        A.They can chemically react with particles on a planet’s surface.

        B.Most of their mass does not flow but remains in place.

        C.Their movement can reshape the surface of certain kinds of planets.

        D. Their movement is driven by the Sun and by gravity.
    \end{chc}

    \begin{psgq}
        A fluid is a substance, such as a liquid or gas, in which the component particles (usually molecules) can move past one another. Fluids flow easily and conform to the shape of their containers. \textbf{The geologic processes related to the movement of fluids on a planet’s surface can completely resurface a planet many times.}\textsubscript{C} \textbf{These processes derive their energy from the Sun and the gravitational forces of the planet itself.}\textsubscript{D} \textbf{As these fluids interact with surface materials, they move particles about or react chemically with them to modify or produce materials.}\textsubscript{A} On a solid planet with a hydrosphere the combined mass of water on, under, or above a planet’s surface and an atmosphere, \textbf{only a tiny fraction of the planetary mass flows as surface fluids.} Yet the movements of these fluids can drastically alter a planet. Consider Venus and Earth, both terrestrial planets with atmospheres.
    \end{psgq}

    \begin{nlz}
        本题为否定事实信息题,要选出与事实信息不符的选项。下面我们来看选项:

        A选项:流体可以与星球表面发生化学反应。根据选项中的关键词“chemically”,定位到第一段倒数第4句“As these fluids interact with surface materials, they move particles about or react chemically with them to modify or produce materials.”这句话提到了流体可以与别的物质发生化学反应,故A选项符合原文,排除。

        B选项:流体的大部分物质是不会移动的,会留在原处。根据关键词“mass”定位到第一段倒数第3句“On a solid planet with a hydrosphere and an atmosphere, only a tiny fraction of the planetary mass flows as surface fluids.”这句话是说行星的物质中,只有一小部分物质是流体。而B选项说的是,在流体中,大部分的物质不会移动。所以2者所针对的对象不一样。其次,即使只看选项我们也能判断其不符合常识,因为第一段第1句就说流体的组成粒子会相互移动。因此,流体的所有组成部分都是会流动的。故B选项不符合原文,为正确答案。

        C选项:流体的移动会重塑行星地表。对应第一段第3句“The geologic processes related to the movement of fluids on a planet's surface can completely resurface a planet many times.”选项中“reshape”一词正好是原文中“resurface”的同义替换。故C选项符合原文,排除。

        D选项:流体的运动是由太阳和引力作用驱动的。对应第4句“These processes derive their energy from the Sun and the gravitational forces of the planet itself.”故D选项完全符合原文,排除。所以本题选B。
    \end{nlz}
\end{blk}

\begin{blk}
    \begin{qst}
        Q13. Insert
    \end{qst}

    \begin{chc}
        Venus may not have always been \textbf{this way}.
    \end{chc}

    \begin{psgq}
        Venus and Earth are commonly regarded as twin planets but not identical twins. They are about the same size, are composed of roughly the same mix of materials, and may have been comparably endowed at their beginning with carbon dioxide and water. However, the twins evolved differently, largely because of differences in their distance from the Sun. With a significant amount of internal heat, Venus may continue to be geologically active with volcanoes, rifting, and folding. However, it lacks any sign of a hydrologic system (water circulation and distribution): there are no streams, lakes, oceans, or glaciers. Space probes suggest that Venus may have started with as much water as Earth, but it was unable to keep its water in liquid form. Because Venus receives more heat from the Sun, water released from the interior evaporated and rose to the upper atmosphere where the Sun’s ultraviolet rays broke the molecules apart. Much of the freed hydrogen escaped into space, and Venus lost its water. Without water, Venus became less and less like Earth and kept an atmosphere filled with carbon dioxide. The carbon dioxide acts as a blanket, creating an intense greenhouse effect and driving surface temperatures high enough to melt lead and to prohibit the formation of carbonate minerals. Volcanoes continually vented more carbon dioxide into the atmosphere. On Earth, liquid water removes carbon dioxide from the atmosphere and combines it with calcium, from rock weathering, to form carbonate sedimentary rocks. Without liquid water to remove carbon from the atmosphere, the level of carbon dioxide in the atmosphere of Venus remains high.
    \end{psgq}

    \begin{nlz}
        本题为句子插入题。待插入句的意思是“金星可能也并不总是这样。”然后我们回到原文来看。 A方框前面一句话是在说金星的地质活动活跃,A方框后一句话转折说金星上没有水文系统。从句意方面看,句子插在A处显然不合适。 B方框前一句话说金星上没有水文系统。而B方框后一句话说航天探测器显示,金星和地球的含水量在开始的时候可能是一样的。这两句话存在转折关系。故句子插在B处最合适。——“金星可能并不一直是这样(没有水文系统)的,一开始金星和地球一样也含有水。” 而C方框和D方框前后的内容都是在讨论为什么金星上没有水。逻辑非常完整,不需要再插入句子了,所以C、D排除。所以本题选B。
    \end{nlz}
\end{blk}