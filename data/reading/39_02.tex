\section{TPO 39 Passage 02}

\subsection{Words}

\begin{tabular}{lll}
    abrupt            & adj. & 突然的                   \\
    striking          & adj. & 惊人的                   \\
    landmass          & n.   & 地块                    \\
    seal              & n.   & 海豹                    \\
    gigantic          & adj. & 巨大的,庞大的               \\
    remain\textbf{s}  & n.   & 剩余物;残留物;遗迹,遗址         \\
    niche             & n.   & 适合的工作(或职位);称心的工作      \\
    herbivore         & n.   & 食草动物,草食动物             \\
    shrubland         & n.   & 灌丛草地                  \\
    intensive         & adj. & mijide                \\
    deplete           & v.   & 消耗                    \\
    excavate          & v.   & 发掘;挖出                 \\
    artifact/artefact & n.   & (有史学价值的)人工制品,制造物,手工艺品 \\
    settlement        & n.   & (结束争端的)协议             \\
    eradicate         & v.   & 根除;消灭;杜绝              \\
    speculative       & adj. & 猜测的;推测的,推断的           \\
    vulnerability     & n.   & 脆弱性;脆弱之物              \\
    flora             & n.   & (某一地点或时期的)植物群         \\
    fauna             & n.   & (某一地区的)动物群            \\
\end{tabular}

\subsection{Collocation}

\begin{tabular}{ll}
    in concert & 共同,一起;一致 \\
\end{tabular}

\newpage

\subsection{Analyse}

\begin{blk}
    \begin{qst}
        Q8. Why does the author say that “At one excavated Maori site, moa remains filled six railway cars.”?
    \end{qst}

    \begin{chc}
        A. To indicate how large the moa population was before it was hunted

        C. To illustrate the intensity with which the Maori hunted moa
    \end{chc}

    \begin{psgq}
        For South Island, human predation appears to have been a significant factor in the depletion of the population of moa. \textbf{At one excavated Maori site, moa remains filled six railway cars.} The density of Maori settlements and artifacts increased substantially at the time of the most intensive moa hunting (900 to 600 years ago). This period was followed by a time of decline in the Maori population and a societal transition to smaller, less numerous settlements. The apparent decline fits the pattern expected as a consequence of the Maori’s overexploitation of moa.
    \end{psgq}

    \begin{nlz}
        选项A说为了证实恐鸟的数量有多大,太片面。。选项C体现毛利人对于恐鸟捕杀的力度。
    \end{nlz}
\end{blk}

\newpage