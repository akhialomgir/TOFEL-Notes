\section{TPO 4 Passage 1}

\subsection{Words}

\begin{tabular}{lll}
    inhibit    & v.   & 阻碍      \\
    succulent  & adj. & 多汁的;诱人的 \\
    starvation & n.   & 饥荒      \\
    rebound    & n.   & 反弹      \\
\end{tabular}

\subsection{Analyse}

\begin{blk}
    \begin{qst}
        Q5.The author tells the story of the explorers Lewis and Clark in paragraph 3 in order to illustrate which of the following points?
    \end{qst}

    \begin{chc}
        A. The number of deer within the Puget Sound region has varied over time.

        B. Most of the explorers who came to the Puget Sound area were primarily interested in hunting game.

        C. There was more game for hunting in the East of the United States than in the West.

        D. Individual explorers were not as successful at locating games as were the trading companies.
    \end{chc}

    \begin{psgq}
        The numbers of deer have fluctuated markedly since the entry of Europeans into Puget Sound country. The early explorers and settlers told of abundant deer in the early 1800s and yet almost in the same breath bemoaned the lack of this succulent game animal. Famous explorers of the north American frontier, Lewis and Clark arrived at the mouth of the Columbia River on November 14, 1805, in nearly starved circumstances. They had experienced great difficulty finding game west of the Rockies and not until the second of December did they kill their first elk. To keep 40 people alive that winter, they consumed approximately 150 elk and 20 deer. And when game moved out of the lowlands in early spring, the expedition decided to return east rather than face possible starvation. Later on in the early years of the nineteenth century, when Fort Vancouver became the headquarters of the Hudson's Bay Company, deer populations continued to fluctuate. David Douglas,Scottish botanical explorer of the 1830s, found a disturbing change in the animal life around the fort during the period between his first visit in 1825 and his final contact with the fort in 1832. A recent Douglas biographer states:" The deer which once picturesquely dotted the meadows around the fort were gone [in 1832], hunted to extermination in order to protect the crops.
    \end{psgq}

    \begin{nlz}
        功能目的题,往前看,这两个人明显是早期探险家的一个例子,读前句说他们知道原本有很多鹿的但又没找到,很显然这句话不足以作为一个观点,往前看本段中心句:
        The numbers of deer have fluctuated markedly since the entry of Europeans into Puget Sound country.
        本段中心句说鹿的数量变化很大,对应 A 选项的 varied,所以 A 正确。BCD 都没有提及。
    \end{nlz}
\end{blk}

\begin{blk}
    \begin{qst}
        Q14
    \end{qst}

    \begin{chc}
        A.The balance of deer species in the Puget Sound region has changed over time, with the Columbian white-tailed deer now outnumbering other types of deer.

        B.Deer populations naturally fluctuate, but early settlers in the Puget Sound environment caused an overall decline in the deer populations of the areas at that time.
    \end{chc}

    \begin{nlz}
        The balance 选项前半句是对的,但后半句与第一段的最后两句说反,应该是黑多,错

        Deer populations 选项对应原文第三段首句和第四段第二三句,正确。in the long term 选项对应原文第五段首句,第二句和第四句,正确。because 选项太细节,不选。although 选项对应原文第四段首句和第五句,正确。wildlife 选项原文没有提及,错。
    \end{nlz}
\end{blk}
