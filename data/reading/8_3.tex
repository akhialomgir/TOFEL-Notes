\section{TPO 8 Passage 3}

\subsection{Words}

\begin{tabular}{lll}
    recede & v. & 后退 \\
\end{tabular}

\subsection{Analyse}

\begin{blk}
    \begin{qst}
        Q6.According to paragraph 2, all of the following are true of the outflow channels on Mars EXCEPT:
    \end{qst}

    \begin{chc}
        A. They formed at around the same time that volcanic activity was occurring on the northern plains.

        B. They are found only on certain parts of the Martian surface.

        C. They sometimes empty onto what appear to have once been the wet sands of tidal beaches.

        D. They are thought to have carried water northward from the equatorial regions.
    \end{chc}

    \begin{psgq}
        Outflow channels are probably relics of catastrophic flooding on Mars long ago. They appear only in equatorial regions and generally do not form extensive interconnected networks. Instead, they are probably the paths taken by huge volumes of water draining from the southern highlands into the northern plains. The onrushing water arising from these flash floods likely also formed the odd teardrop-shaped “islands” (resembling the miniature versions seen in the wet sand of our beaches at low tide) that have been found on the plains close to the ends of the outflow channels. Judging from the width and depth of the channels, the flow rates must have been truly enormous—perhaps as much as a hundred times greater than the 105 tons per second carried by the great Amazon river. Flooding shaped the outflow channels approximately 3 billion years ago, about the same times as the northern volcanic plains formed.
    \end{psgq}

    \begin{nlz}
        正确,不选;B 的 certain parts 与原文第二句的 equatorial regions 同义重合,正确,不选;C 的 beaches 做关键词定位至倒数第三句,但原文说洪水形成的小岛形状像海滩上的沙子,跟 C 说的不同,所以 C 错,选;D 的 northward 做关键词定位至第三句,再结合第二句,说明 D 正确,不选。
    \end{nlz}
\end{blk}

\begin{blk}
    \begin{qst}
        Q13. These landscape features differ from runoff channels in a number of ways..
    \end{qst}

    \begin{psgq}
        Outflow channels are probably relics of catastrophic flooding on Mars long ago. $\blacksquare$They appear only in equatorial regions and generally do not form extensive interconnected networks. $\blacksquare$Instead, they are probably the paths taken by huge volumes of water draining from the southern highlands into the northern plains. $\blacksquare$The onrushing water arising from these flash floods likely also formed the odd teardrop-shaped “islands” (resembling the miniature versions seen in the wet sand of our beaches at low tide) that have been found on the plains close to the ends of the outflow channels. [These landscape features differ from runoff channels in a number of ways.]Judging from the width and depth of the channels, the flow rates must have been truly enormous—perhaps as much as a hundred times greater than the 105 tons per second carried by the great Amazon river. Flooding shaped the outflow channels approximately 3 billion years ago, about the same times as the northern volcanic plains formed.
    \end{psgq}

    \begin{nlz}
        A,两个过渡点,these landscape features 和 a number of ways,特别注意 a number of ways 应该是个提纲性的句子,所以应该尽量往前插,后面的若干句话都在叙述 outflow 与 runoff channel 的不同,所以 A 正确。these landscape features 指代前面的 relics of catastrophic flooding。

        提纲而非总结
    \end{nlz}
\end{blk}


\begin{blk}
    \begin{qst}
        Q14
    \end{qst}

    \begin{nlz}
        arious 选项 MS 对应第一段第一句,但原文没说相片有 various type,不选。
    \end{nlz}
\end{blk}
