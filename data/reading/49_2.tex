\section{TPO 49 Passage 02}

\subsection{Words}

\begin{tabular}{lll}
    medieval    & adj. & 中世纪(约公元600年至1500年间)的,中古时代的 \\
    culmination & n.   & 顶点                         \\
    antiquity   & n.   & (尤指六世纪以前的)古代               \\
    vellum      & n.   & 羊皮纸                        \\
    merit       & n.   & 优点;价值;功绩                   \\
    monk        & n.   & 修道士                        \\
    rivalry     & n.   & 相互较劲                       \\
\end{tabular}

\subsection{Analyse}

\begin{blk}
    \begin{qst}
        Q4. According to paragraph 2, all of the following were \textbf{advantages of movable metal type} EXCEPT:
    \end{qst}

    \begin{chc}
        A. It could be reused.

        B. It made letters of standardized size.

        C. It did not require specialized skill to use.

        D. It could easily be restored from a mold.
    \end{chc}

    \begin{psgq}
        In the years 1446-1448, two German goldsmiths, Johannes Gutenberg and Johann Fust, made use of cheap paper to introduce a critical improvement in the way written pages were reproduced. Printing from wooden blocks was the old method; what the Germans did was to invent movable type for the letterpress. \textbf{It had three merits: it could be used repeatedly until worn out; it was cast in metal from a mold and so could be renewed without difficulty; and it made lettering uniform.} In 1450, Gutenberg began work on his Bible, the first printed book, known as the Gutenberg. It was completed in 1455 and is a marvel. As Gutenberg, apart from getting the key idea, had to solve a lot of practical problems, including imposing paper and ink into the process, and the actual printing itself, for which he adapted the screw press used by winemakers, it is amazing that his first product does not look at all rudimentary. Those who handle it are struck by its clarity and quality.
    \end{psgq}

    \begin{nlz}
        根据题干中的advantages of movable metal type 定位到 It had three merits... 这句话,A  It could be reused  对应 it could be used repeatedly,排除。B It made letters of standardized size 对应it made lettering uniform(一样的; 规格一致的),排除。 D It could easily be restored from a mold 对应 cast in metal from a mold and so could be renewed,排除。
    \end{nlz}
\end{blk}

\begin{blk}
    \begin{qst}
        Q7. According to the passage, the role of the monastic scriptoria was to
    \end{qst}

    \begin{chc}
        A. translate old religious texts into modern languages

        B. turn books printed at the new presses into luxury items

        C. produce reference works solely for religious use

        D. create books of a quality that was beyond the means of the printing industry
    \end{chc}

    \begin{psgq}
        The old \textbf{monastic scriptoria}—monastery workshops where monks copied texts by hand—worked closely alongside the new presses, continuing to \textbf{produce the luxury goods that movable-type printing could not yet supply}. Printing aimed at a cheap mass sale.
    \end{psgq}

    \begin{nlz}
        D选项中的a quality that was beyond the means of the printing industry 对应produce the luxury goods that movable-type printing could not yet supply,所以选D。
    \end{nlz}
\end{blk}

\begin{blk}
    \begin{qst}
        Q9. Why does the author mention \textbf{24 presses} in the discussion?
    \end{qst}

    \begin{chc}
        A. To indicate the extent to which the printing industry had grown in Germany

        B. To emphasize that printing presses far outnumbered monastic scriptoria

        C. To indicate the importance of trade fairs as a way of promoting printing presses

        D. To argue that the centers of printing had begun to shift from Germany to other parts of Europe
    \end{chc}

    \begin{psgq}
        Presses sprang up in several German cities, and by 1470, Nuremberg, \textbf{Germany, had established itself as the center of the international publishing trade}, printing books from 24 presses and distributing them at trade fairs all over western and central Europe. The old monastic scriptoria—monastery workshops where monks copied texts by hand—worked closely alongside the new presses, continuing to produce the luxury goods that movable-type printing could not yet supply. Printing aimed at a cheap mass sale.
    \end{psgq}

    \begin{nlz}
        根据题干中的24 presses定位到第3段倒数第3句前半部分,意思是印刷机在德国几个城市兴起,到 1470年,德国纽伦堡市已经成为国际出版贸易的中心。 综合句意选A
    \end{nlz}
\end{blk}

\begin{blk}
    \begin{qst}
        Q10. Which of the following can be inferred from paragraph 4 about the “\textbf{Gothic}” typeface used in Germany?
    \end{qst}

    \begin{chc}
        A. It was adopted by the Italians when they hired two leading German printers.

        B. It was more difficult to read than roman typeface.

        C. It was easier to print than other styles of typeface.

        D. It was widely popular with international readers.
    \end{chc}

    \begin{psgq}
        German printers had the disadvantage of working with the complex typeface that the Italians sneeringly referred to as “\textbf{Gothic}” and that later became known as black letter. Outside Germany, readers found this typeface disagreeable. The Italians, \textbf{on the other hand}, had a \textbf{clear typeface} known as roman that became the type of the future.
    \end{psgq}

    \begin{nlz}
        根据题干中的 Gothic 定位到第4段倒数第3句,意思是德国印刷工人用complex typeface 复杂的印刷字体工作有劣势,再结合下一句意思是在德国之外,读者们发现这种字体不合意,综合意思,B最符合。
    \end{nlz}
\end{blk}

\begin{blk}
    \begin{qst}
        Q12. According to paragraphs 4 and 5, how did German and Italian contributions to the printing industry differ?
    \end{qst}

    \begin{chc}
        A. German printers originated and applied a technique that Italian printers adapted more artistically.

        B. German printers mass-produced books, while Italian printers produced fewer books of better quality.

        C. German printers used only black typeface, while Italian printers used a variety of typeface colors and styles.

        D. German printers had greater technological skill, while Italian printers were more commercially successful.
    \end{chc}

    \begin{psgq}
        Although there was no competition between the technologies, there was rivalry between nations. The Italians made energetic and successful efforts to catch up with Germany. Their most successful scriptorium quickly imported two leading German printers to set up presses in their book-producing shop. German printers had the disadvantage of working with the complex typeface that the Italians sneeringly referred to as “Gothic” and that later became known as black letter. Outside Germany, readers found this typeface disagreeable. The Italians, on the other hand, had a clear typeface known as roman that became the type of the future.

        \textbf{Hence, although the Germans made use of the paper revolution to introduce movable type, the Italians went far to regain the initiative by their artistry.} By 1500 there were printing firms in 60 German cities, but there were 150 presses in Venice alone. However, since many nations and governments wanted their own presses, the trade quickly became international. The cumulative impact of this industrial spread was spectacular. Before printing, only the very largest libraries, of which there were a dozen in Europe, had as many as 600 books. The total number of books on the entire Continent was well under 100,000. But by 1500, after only 45 years of the printed book, there were 9 million in circulation.
    \end{psgq}

    \begin{nlz}
        主要根据第5段第1句意思,尽管德国人利用纸革命引入活字印刷,但意大利人根据他们的艺术技艺获得了主动权而扬名。综合意思,A最符合。
    \end{nlz}
\end{blk}

\begin{blk}
    \begin{qst}
        Q14.Insert

        The typeface that followed was italic, with a slanted appearance in the style of handwriting and a name that was recognizably Italian.
    \end{qst}

    \begin{psgq}
        Although there was no competition between the technologies, there was rivalry between nations. The Italians made energetic and successful efforts to catch up with Germany. Their most successful scriptorium quickly imported two leading German printers to set up presses in their book-producing shop. $\square$ German printers had the disadvantage of working with the complex typeface that the Italians sneeringly referred to as “Gothic” and that later became known as black letter. $\square$ Outside Germany, readers found this typeface disagreeable. $\square$ The Italians, on the other hand, had a clear typeface known as roman that became the type of the future. $\square$
    \end{psgq}

    \begin{nlz}
        根据插入句里的关键点The typeface that followed 和 recognizably Italian 分别对应第4个黑框前的句子的  a clear typeface 和the type of the future ,The Italians ,所以放在第4个黑框。
    \end{nlz}
\end{blk}

\begin{blk}
    \begin{qst}
        Q14. Summary
    \end{qst}

    \begin{chc}
        A. The industrial process for mass paper production was first introduced in the early fifteenth century.

        B. The mechanized production of books in the fifteenth century is the first instance of a modern industry in Europe.

        C. The Gutenberg Bible was the result of combined technologies in the mass production of paper and the newly invented manufacture of movable type.

        D. Hand-copied texts continued to be in as great demand as printed books in fifteenth-century Germany and Italy.

        E. The increased need for classical texts and reference books along with the existence of an established workshop system stimulated rapid growth in the printing trade.

        F. Printed works were located primarily in libraries at the end of the fifteenth century because they were still too expensive for mass sale.
    \end{chc}

    \begin{nlz}
        A 选项根据第1段中的这三句话 In the early Middle Ages, Europe imported an industrial process from China.... s the growing availability of cheap paper,表达错误,所以排除。

        B选项对应第4段内容,正确。

        C 选项对应第2段内容,正确。

        D 选项 根据第3段最后两句话意思,表达错误,所以排除。

        E选项对应第3段内容,正确。

        F选项 根据第5段最后三句话意思,表达错误,所以排除。
    \end{nlz}
\end{blk}
