\section{TPO 46 Passage 02}

\subsection{Words}

\begin{tabular}{lll}
    radical  & adj. & 激进的;过激的;极端的         \\
    league   & n.   & (体育项目)联合会,联赛,联盟     \\
    alliance & n.   & 结盟国家(或团体),同盟国家(或团体) \\
\end{tabular}

\subsection{Analyse}

\begin{blk}
    \begin{qst}
        Q8. Why does the author provide the information in paragraph 4 that \textbf{the commercial classes never exceeded 10 percent of the population}?
    \end{qst}

    \begin{chc}
        A. To argue that the wealth created by the commercial revolution benefited only a small number of people

        B. To challenge the view that the commercial classes made up a majority of the population of Europe

        C. To suggest a reason that the commercial revolution ended around A.D. 1300

        D. To emphasize the point that the commercial revolution was brought about by a small part of the population
    \end{chc}

    \begin{psgq}
        These developments added up to what one modern scholar has called “a commercial revolution.” In the long run, the commercial revolution of the High Middle Ages (A.D. 1000–1300) brought about radical change in European society. One remarkable aspect of this change was that the commercial classes constituted a small part of the total population—\textbf{never more than 10 percent}. They exercised an influence far in excess of their numbers. The commercial revolution created a great deal of new wealth, which meant a higher standard of living. The existence of wealth did not escape the attention of kings and other rulers. Wealth could be taxed, and through taxation, kings could create strong and centralized states. In the years to come, alliances with the middle classes were to enable kings to weaken aristocratic interests and build the states that came to be called modern.
    \end{psgq}

    \begin{nlz}
        根据原文One remarkable aspect of this change was that the commercial classes constituted a small part of the total population—never more than 10 percent.可知, never more than 10 percent是对前一句话的解释。故D选项正确。
    \end{nlz}
\end{blk}

\begin{blk}
    \begin{qst}
        Q9. According to paragraph 4, which of the following was associated with \textbf{the rise of modern states}?
    \end{qst}

    \begin{chc}
        A. Increased wealth for the ruling classes

        B. The weakening of the aristocracy

        C. The decline of the middle class

        D. A reduction in taxes
    \end{chc}

    \begin{psgq}
        These developments added up to what one modern scholar has called “a commercial revolution.” In the long run, the commercial revolution of the High Middle Ages (A.D. 1000–1300) brought about radical change in European society. One remarkable aspect of this change was that the commercial classes constituted a small part of the total population—never more than 10 percent. They exercised an influence far in excess of their numbers. The commercial revolution created a great deal of new wealth, which meant a higher standard of living. The existence of wealth did not escape the attention of kings and other rulers. Wealth could be taxed, and through taxation, kings could create strong and centralized states. In the years to come, alliances with the middle classes were to enable kings to \textbf{weaken aristocratic interests} \textbf{and} build the states that came to be called \textbf{modern}.
    \end{psgq}

    \begin{nlz}
        从文中最后一句可知
    \end{nlz}
\end{blk}

\begin{blk}
    \begin{qst}
        Q13. insert

        While \textbf{it} originated in the German city of Lübeck, it began to expand in \textbf{1241} when Lübeck entered into a mutual protection treaty with the city of Hamburg.
    \end{qst}

    \begin{psgq}
        The ventures of the German Hanseatic League illustrate these advancements. The Hanseatic League was a mercantile association of European towns dating from 1159.$\blacksquare$ The league grew by the end of the fourteenth century to include about 200 cities from Holland to Poland. $\blacksquare$ Across regular, well-defined trade routes along the Baltic and North seas, the ships of league cities carried furs, wax, copper, fish, grain, timber, and wine. $\blacksquare$ These goods were exchanged for finished products, mainly cloth and salt, from western cities. $\blacksquare$ At cities such as Bruges and London, Hanseatic merchants secured special trading concessions, exempting them from all tolls and allowing them to trade at local fairs. Hanseatic merchants established foreign trading centers, the most famous of which was the London Steelyard, a walled community with warehouses, offices, a church, and residential quarters for company representatives. By the late thirteenth century, Hanseatic merchants had developed an important business technique, the business register. Merchants publicly recorded their debts and contracts and received a league guarantee for them. This device proved a decisive factor in the later development of credit and commerce in northern Europe.
    \end{psgq}

    \begin{nlz}
        首先,待插入的句子中有近指代词it,说明该句子的上一句讲了一个单数名词概念,所以C、D被排除。其次,句子中提到了“it'在1241年开始扩张,是一个时间概念,和该段第二句说The Hanseatic League可追溯到1159年相呼应,可知选A。
    \end{nlz}
\end{blk}
