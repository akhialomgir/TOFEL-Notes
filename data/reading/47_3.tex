\section{TPO 47 Passage 03}

\subsection{Words}

\begin{tabular}{lll}
    pertinent    & adj. & 有关的,直接相关的       \\
    render       & v.   & 使成为;使变得;使处于某种状态 \\
    prerequisite & n.   & 先决条件,前提,必备条件    \\
    carnivorous  & adj. & 食肉的             \\
    zooplankton  & n.   & 浮游动物            \\
    atoll        & n.   & 环状珊瑚岛,环礁        \\
    lagoon       & n.   & 泻湖,濒海湖,环礁湖      \\
    fringe       & n.   & 边缘,外围;(活动的)次要部分 \\
    succession   & n.   & 一系列;接连;继任,接任;继承 \\
    substratum   & n.   & 底土层;底土,心土       \\
    complex      & n.   & 复合体             \\
    swamp        & n.   & 沼泽地;湿地          \\
\end{tabular}

\subsection{Collocation}

\begin{tabular}{ll}
    live off sb/sth & 依靠…生活 \\
\end{tabular}

\subsection{Analyse}

\begin{blk}
    \begin{qst}
        Q1. According to paragraph 1, all of the following are needed for the growth of coral reefs EXCEPT
    \end{qst}

    \begin{chc}
        A. a solid base to grow on

        B. exposure to light

        C. the presence of river-borne sediment

        D. ocean temperatures of 21°C or higher
    \end{chc}

    \begin{psgq}
        Coral reefs are found where the ocean water \textbf{temperature is not less than 21°C}\textsubscript{D}, where there is a \textbf{firm substratum}\textsubscript{A}, and where the seawater \textbf{is not rendered too dark}\textsubscript{B} by excessive amounts of river-borne sediment.
    \end{psgq}

    \begin{nlz}
        原文Coral reefs are found where the ocean water temperature is not less than 21 °C,说明温度不能低于21°C,对应选项D中:21 °C or higher,原文where there is a firm substratum,说明基础一定要坚实,对应选项A中:a solid base,原文the seawater is not rendered too dark,说明不能太暗,对应选项B中:light。过多的沉淀物会遮住光不利于珊瑚礁生长,选项C错在sediment。
    \end{nlz}
\end{blk}

\begin{blk}
    \begin{qst}
        Q7. According to paragraph 2, which of the following is NOT characteristic of a \textbf{barrier reef}?
    \end{qst}

    \begin{chc}
        A. It is located away from the shore of the neighboring land.

        B. It is separated from neighboring land by a wide channel.

        C. It is located in deep ocean water.

        D. It surrounds a small, central lagoon.
    \end{chc}

    \begin{psgq}
        Coral reefs have fascinated scientists for almost 200 years, and some of the most pertinent observations of them were made in the 1830s by Charles Darwin on the voyage of the Beagle. He recognized that there were three major kinds: fringing reefs, barrier reefs, and atolls; and he saw that they were related to each other in a logical and gradational sequence. A fringing reef is one that lies close to the shore of some continent or island. Its surface forms an uneven and rather rough platform around the coast, about the level of low water, and its outer edge slopes downwards into the sea. Between the fringing reef and the land there is sometimes a small channel or lagoon. When the lagoon is wide and deep and the reef lies \textbf{at some distance from the shore}\textsubscript{A B} and \textbf{rises from deep water}\textsubscript{C} it is called a barrier reef. An atoll is a reef in the form of a ring or horseshoe \textbf{with a lagoon in the center}\textsubscript{DX}.
    \end{psgq}

    \begin{nlz}
        原文An atoll is a reef … with a lagoon in the center,说明中间有泻湖的是环礁,所以选项D不是堡礁的特点。
    \end{nlz}
\end{blk}

\begin{blk}
    \begin{qst}
        Q8. Which of the sentences below best expresses the essential information in the highlighted sentence in the passage? Incorrect choices change the meaning in important ways or leave out essential information.
    \end{qst}

    \begin{chc}
        A. Darwin claimed that, of the three types of coral reefs, only an atoll would be able to survive on a sinking platform.

        B. Darwin recognized that coral reefs achieved success by growing upward from a sinking land platform and becoming an atoll.

        C. Darwin argued that as a coral reef grew up from a sinking island, it would become a fringing reef, then a barrier reef, and finally, with the disappearance of the island, an atoll.

        D. Darwin’s theory helped explain the disappearance of a number of islands by showing how coral reef growth caused them to sink below the ocean surface.
    \end{chc}

    \begin{psgq}
        Darwin’s theory was that the succession from one coral reef type to another could be achieved by the upward growth of coral from a sinking platform, \textbf{and that} there would be a progression from a fringing reef, through the barrier reef stage until, with the disappearance through subsidence (sinking) of the central island, only a reef-enclosed lagoon or atoll would survive.
    \end{psgq}

    \begin{nlz}
        划线句子意为珊瑚礁从下沉的岛屿上长出来,然后变成裙礁,再变成堡礁,最后岛屿消失变成环礁,对应选项C中:Darwin … as a coral reef … a sinking island, … fringing reef, … barrier reef, and … disappearance of the island, an atoll.
    \end{nlz}
\end{blk}

\begin{blk}
    \begin{qst}
        Q10. Why does the passage provide the information that the drill holes in the Pacific atolls \textbf{passed through more than a thousand meters of coral before reaching the rock substratum of the ocean floor}?
    \end{qst}

    \begin{chc}
        A. To emphasize that according to Darwin’s view coral can grow at great depths

        B. To indicate how scientists knew the rate at which Earth’s crust had subsided

        C. To support the claim that coral reefs take millions of years to form

        D. To present the evidence that confirmed Darwin’s account of coral reef evolution
    \end{chc}

    \begin{psgq}
        Darwin’s theory was that the succession from one coral reef type to another could be achieved by the upward growth of coral from a sinking platform, and that there would be a progression from a fringing reef, through the barrier reef stage until, with the disappearance through subsidence (sinking) of the central island, only a reef-enclosed lagoon or atoll would survive. A long time after Darwin put forward this theory, some deep boreholes were drilled in the Pacific atolls in the 1950s. \textbf{The drill holes passed through more than a thousand meters of coral before reaching the rock substratum of the ocean floor}, \textbf{and} indicated that the coral had been growing upward for tens of millions of years as Earth’s crust subsided at a rate of between 15 and 51 meters per million years. Darwin’s theory was \textbf{therefore} proved basically correct. There are some submarine islands called guyots and seamounts, in which subsidence associated with sea-floor spreading has been too speedy for coral growth to keep up.
    \end{psgq}

    \begin{nlz}
        原文Darwin s theory was therefore proved basically correct,说明举例太平洋环礁就是为了证明达尔文理论正确,对应选项D中:To present the evidence that confirmed Darwin’s account of coral reef evolution。
    \end{nlz}
\end{blk}
