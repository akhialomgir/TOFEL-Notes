\section{TPO 48 Passage 03}

\subsection{Words}

\begin{tabular}{lll}
    infrared      & adj.   & 红外线的                  \\
    retard        & v.     & 阻碍;减缓;智力迟钝者;弱智;笨蛋     \\
    exert         & v.     & 运用;行使(权威、权力等);施加(影响等) \\
    nucleus       & n.     & 原子核;细胞核               \\
    precipitation & n.     & (尤指雨或雪的)降落;降水         \\
    thermal       & adj.n. & 热的,热量的;上升热气流          \\
    outskirt      & n.     & 远离城市中心的地区,市郊,郊区       \\
\end{tabular}

\subsection{Analyse}

\begin{blk}
    \begin{qst}
        Q14. Summary
    \end{qst}

    \begin{chc}
        A. In the countryside, much solar energy is used in evaporation, but in the city this energy builds up as heat.

        B. The urban heat island is strongest in the summer, when the days are long and the sunlight is intense.

        C. Increased industrial and urban development has also increased average levels of humidity over the last century.

        D. Heat and air are trapped in the irregular spaces between buildings, which creates the atmospheric conditions that result in storms and winds.

        E. Pollution from cars and factories helps increase the amounts of fog and precipitation that occur in cities.

        F. Country breezes blow pollutants out \textbf{from the cities into the surrounding} countryside.
    \end{chc}

    \begin{nlz}
        文章首先介绍太阳能在农村都用于蒸发水分,而在城市都储存为热,对应选项A中:In the countryside, much solar energy is used in evaporation, but in the city this energy builds up as heat,
        文章然后介绍污染物会让城市有更多烟雾和降雨,对应许选项E中:Pollution from cars and factories helps increase the amounts of fog and precipitation that occur in cities,
        文章最后介绍压强差导致风暴,对应选项D中:Heat and air are trapped in the irregular spaces between buildings, which creates the atmospheric conditions that result in storms and winds。
        文章没有提到湿度的升降,选项C错在humidity。
        文章说冬天城市热岛效应更为明显,选项B错在summer。
        文章说污染物从郊区走向城市,选项F错在from cities into … countryside。
    \end{nlz}
\end{blk}
