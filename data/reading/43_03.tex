\section{TPO 43 Passage 03}

\subsection{Words}

\begin{tabular}{lll}
    citizenry          & n.   & (全体)市民(或公民)           \\
    assembly           & n.   & 聚集在一起的人,(所有)与会者;集合,聚集 \\
    jack-of-all-trades & n.   & 多面手,万事通               \\
    consequent         & adj. & 作为结果的;随之而来的;由此引起的     \\
    professionalism    & n.   & 专业水准;专业精神             \\
    ritual             & n.   & 例行公事,老规矩;(尤指)仪式       \\
    morality           & n.   & 道德体系,道德观              \\
    conscience         & n.   & 良心;良知                 \\
    corrupt            & adj. & 贪赃舞弊的;以权谋私的;腐败的;堕落的   \\
    receptive          & adj. & (对于新思想和建议)乐于接受的,从善如流的 \\
\end{tabular}

\subsection{Collocation}

\newpage

\subsection{Analyse}

\begin{blk}
    \begin{qst}
        Q8. According to paragraph 4, Alexander’s empire was characterized by all of the following EXCEPT
    \end{qst}

    \begin{chc}
        A. decreased need for military control

        B. growing professionalism

        C. growth of cities

        D. specialization in trades
    \end{chc}

    \begin{psgq}
        This implied that the city-state was based on the idea that citizens were not specialists but had multiple interests and talents—each a so-called jack-of-all-trades who could engage in many areas of life and politics. It implied a respect for the wholeness of life and a consequent dislike of specialization. It implied economic and military self-sufficiency. \textbf{But} with the development of trade and commerce in Alexander’s empire came \textbf{the growth of cities}\textsuperscript{C}; it was \textbf{no longer possible to be a jack-of-all-trades}\textsuperscript{D}. \textbf{One now had to specialize, and with specialization came professionalism.}\textsuperscript{B} There were getting to be too many persons to know; an easily observable community of interests was being replaced by a multiplicity of interests. The city-state was simply too “small-time.”
    \end{psgq}
\end{blk}

\newpage