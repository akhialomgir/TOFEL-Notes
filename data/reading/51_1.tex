\section{TPO 51 Passage 01}

\subsection{Words}

\begin{tabular}{lll}
    appreciate  & v. & 欣赏;赏识;重视;理解,领会;意识到 \\
    vicinity    & n. & 周围地区;邻近地区;附近       \\
    apex        & n. & 顶点;最高点             \\
    opportune   & n. & (时间)合适的,恰当的,适宜的    \\
    expanse     & n. & (陆地、水面或天空的)广阔区域    \\
    incorporate & v. & 包含;将…包括在内          \\
    terrain     & n. & 地形,地势;地带           \\
    conduit     & n. & (水、电线等的)管道,导管      \\
    artery      & n. & 动脉;要道;干道;干线        \\
    topography  & n. & 地形;地貌;地势           \\
    millennium  & n. & 一千年,千周年;千周年纪念日     \\
    wadi        & n. & (只在雨季有水的)干谷,干河床    \\
    monopolize  & v. & 垄断;包办;实行…的专卖       \\
    pasture     & n. & 牧场                 \\
\end{tabular}

\subsection{Analyse}

\begin{blk}
    \begin{qst}
        Q3. It can be inferred from paragraph 1 that one \textbf{consequence of the unification} of Egypt was
    \end{qst}

    \begin{chc}
        A. the reduction of the strategic importance of older centers of power

        B. the opportunity for the recently united Egypt to become economically self-sufficient

        C. the increase in political tensions between the rulers of Upper and Lower Egypt

        D. the reduction of Egypt’s dependence upon the Nile for trade and communications
    \end{chc}

    \begin{psgq}
        The city of Memphis, located on the Nile near the modern city of Cairo, was founded around 3100 B.C. as the first capital of a recently united Egypt. The choice of Memphis by Egypt’s first kings reflects \textbf{the site’s strategic importance}. First, and most obvious, the apex of the Nile River delta was a politically opportune location for the state’s administrative center, standing between the united lands of Upper and Lower Egypt and offering ready access to both parts of the country. The older predynastic (pre-3100 B.C.) centers of power, This and Hierakonpolis, were too remote from the vast expanse of the delta, which had been incorporated into the unified state. Only a city within easy reach of both the Nile valley to the south and the more spread out, difficult terrain to the north could provide the necessary political control that the rulers of early dynastic Egypt (roughly 3000–2600 B.C.) required.
    \end{psgq}

    \begin{nlz}
        本题为推理题。题目问埃及统一的一个结果是什么。根据第一段最后2句话,我们可以得知,“This”和“Hierakonpolis”是埃及的旧都,但是因为距离尼罗河三角洲太遥远了,所以无法为早期埃及王朝的统治者们提供政治统治的必要条件,因而地理位置更优越的孟斐斯所取代。所以A选项:旧都战略重要性的降低,是正确答案。 B选项:近代统一后的埃及有了实现经济自给自足的机会。错误,因为第一段根本没有提及与经济有关的任何内容,故B选项直接排除。 C选项:上埃及和下埃及统治者之间的关系变得紧张。错误,第一段中只有这句话提到了上埃及和下埃及“First, and most obvious, the apex of the Nile River delta ……standing between the united lands of Upper and Lower Egypt and offering ready access to both parts of the country.”但是这句话是在描述孟斐斯优越的地理位置,与政治关系紧张没有任何联系。所以C选项文中未提及,排除。 D选项:埃及通过尼罗河进行贸易和通讯的依赖性降低。错误,因为第一段中还没有提到埃及依靠尼罗河进行贸易和通讯,而依赖性降低更是无从谈起。故D选项在第一段未提及,排除。所以本题选A。
    \end{nlz}
\end{blk}

\begin{blk}
    \begin{qst}
        Q8. According to paragraph 3, recent research into the \textbf{topography of the Memphis} region in ancient times suggests which of the following?
    \end{qst}

    \begin{chc}
        A. The level of the Nile floodplains was much higher in predynastic and dynastic times than in later times.

        B. The sediment deposits of wadis were not as noticeable in predynastic and dynastic times than in later times.

        C. The Nile valley at the point of Memphis was narrower in predynastic and dynastic times than it was in later times.

        D. Frequent rainy periods may have caused a significant reduction of trade traffic during the predynastic and dynastic times.
    \end{chc}

    \begin{psgq}
        Equally important for the national administration was the ability to control communications within Egypt. The Nile provided the easiest and quickest artery of communication, and the national capital was, again, ideally located in this respect. Recent geological surveys of the Memphis region have revealed much about its topography in ancient times. It appears that the location of Memphis may have been even more advantageous for controlling trade, transport, and communications than was previously appreciated. Surveys and drill cores have shown that the level of the Nile floodplain has steadily risen over the last five millenniums. \textbf{When the floodplain was much lower, as it would have been in predynastic and early dynastic times, the outwash fans (fan-shaped deposits of sediments) of various wadis (stream-beds or channels that carry water only during rainy periods) would have been much more prominent features on the east bank.} The fan associated with the Wadi Hof extended a significant way into the Nile floodplain, forming a constriction in the vicinity of Memphis. \textbf{The valley may have narrowed at this point to a mere three kilometers, making it the ideal place for controlling river traffic.}
    \end{psgq}

    \begin{nlz}
        本题为事实信息题。通读第三段的内容,我们逐一来看选项:

        A选项:尼罗河泛滥平原的高度在“predynastic and dynastic times”的时候要更高。定位到原文这句话“When the floodplain was much lower, as it would have been in predynastic and early dynastic times……”因此,我们可以知道在埃及前王朝时期和埃及王朝早期,泛滥平原的高度要低得多,故A选项与原文矛盾,排除。

        B选项:相比后来,河流沉积物在“predynastic and dynastic times”这一时期更多,但文中并未提到这两个时间段河流沉积物的对比。文中只提到“much more prominent features on the east bank.”东岸的沉积物更多。故B选项文中未提及,排除。

        C选项:尼罗河谷在“predynastic and dynastic times”这一时期更窄。对应文中这句话“The valley may have narrowed at this point to a mere three kilometers, making it the ideal place for controlling river traffic.”意思是这个山谷在当时变得很窄,宽度仅为3公里,这也使它成为了控制河流交通的理想地点。故C选项符合原文,正确。

        D选项:频繁的降雨使“predynastic and dynastic times”的贸易交通减少。错误,因为文中并未提及。

        所以本题选C。
    \end{nlz}
\end{blk}

\begin{blk}
    \begin{qst}
        Q10. According to paragraph 4, which of the following is 、\textbf{NOT a reason Memphis was chosen as the capital of a united Egypt}?
    \end{qst}

    \begin{chc}
        A. It was at the junction of a major trade route with the Nile valley.

        B. It was near land that could be used for animal grazing and for growing crops.

        C. The nearby outwash fans led into wadis that could be used as desert trade routes.

        D. Since foreign traders had \textbf{settled in nearby Maadi}, trade between the two cities could be established.
    \end{chc}

    \begin{psgq}
        Furthermore, the Memphis region seems to have been favorably located for the control not only of river-based trade but also of desert trade routes. \textbf{The two outwash fans in the area gave access to the extensive wadi systems of the eastern desert.}\textsubscript{C} In predynastic times, the Wadi Digla may have served as a trade route between the Memphis region and the Near East, to judge from the unusual concentration of foreign artifacts found in the predynastic settlement of Maadi. Access to, and control of, trade routes between Egypt and the Near East seems to have been a preoccupation of Egypt’s rulers during the period of state formation. The desire to monopolize \textbf{foreign}\textsubscript{DX} trade may have been one of the primary factors behind the political unification of Egypt. The foundation of the national capital at the \textbf{junction}\textsubscript{A} of an important trade route with the Nile valley is not likely to have been accidental. Moreover, the Wadis Hof and Digla provided the Memphis region with accessible desert pasturage. As was the case with the cities of Hierakonpolis and Elkab, the combination within the same area of both desert pasturage and \textbf{alluvial arable land (land suitable for growing crops) was a particularly attractive one for early settlement}\textsubscript{B}; this combination no doubt contributed to the prosperity of the Memphis region from early predynastic times.
    \end{psgq}

    \begin{nlz}
        本题为否定事实信息题。我们从选项中寻找关键词,定位原文,逐一排除。

        A选项中,根据关键词“junction”定位到这一句“The foundation of the national capital at the junction of an important trade route with the Nile valley is not likely to have been accidental.” 综合第四段来看,控制埃及和近东地区之间的贸易路线,是埃及统治者们在国家刚建成时的当务之急,故A选项符合原文,排除。

        B选项对应第四段倒数1、2句话,这两句话说沙漠牧场和冲积耕地在该地区结合,所以B选项符合原文,排除。

        C选项对应第四段第2句“The two outwash fans in the area gave access to the extensive wadi systems of the eastern desert.” 故C选项符合原文,排除。

        D选项说外国商贩在马迪城附近定居,促进了两地的贸易。文中并未提及该信息,故D选项为正确答案。所以本题选D。
    \end{nlz}
\end{blk}

\begin{blk}
    \begin{qst}
        Q12. In paragraph 4, why does the author mention the cities of Hierakonpolis and Elkab?
    \end{qst}

    \begin{chc}
        A. To give an indication of the level of prosperity that Memphis is thought to have enjoyed from its earliest days

        B. To compare the Memphis region to them in terms of their similar combinations of characteristics providing advantages for early settlement

        C. To identify the models that the founders of Memphis followed in laying out the national capital

        D. To suggest that the combination of desert pasturage and alluvial arable land in the same area was very common
    \end{chc}

    \begin{psgq}
        Furthermore, the Memphis region seems to have been favorably located for the control not only of river-based trade but also of desert trade routes. The two outwash fans in the area gave access to the extensive wadi systems of the eastern desert. In predynastic times, the Wadi Digla may have served as a trade route between the Memphis region and the Near East, to judge from the unusual concentration of foreign artifacts found in the predynastic settlement of Maadi. Access to, and control of, trade routes between Egypt and the Near East seems to have been a preoccupation of Egypt’s rulers during the period of state formation. The desire to monopolize foreign trade may have been one of the primary factors behind the political unification of Egypt. The foundation of the national capital at the junction of an important trade route with the Nile valley is not likely to have been accidental. \textbf{Moreover, the Wadis Hof and Digla provided the Memphis region with accessible desert pasturage.} As was the case with the cities of \textbf{Hierakonpolis and Elkab}, the combination within the same area of both desert pasturage and alluvial arable land (land suitable for growing crops) was a particularly attractive one for early settlement; this combination no doubt contributed to the prosperity of the Memphis region from early predynastic times.
    \end{psgq}

    \begin{nlz}
        本题为功能目的题。首先我们通过关键词“Hierakonpolis and Elkab”定位到它在文中的位置,即本段最后一句,意思是“沙漠牧场和冲积耕地(适合耕种农作物的土地)在同一片区域的结合,使这片地区成为一个特别有吸引力的早期定居点;这样的组合无疑促进了前王朝时代早期孟斐斯地区的繁荣。”再来看前一句话“Moreover, the Wadis Hof and Digla provided the Memphis region with accessible desert pasturage.”通过结合这两句话,我们可知孟斐斯和Hierakonpolis and Elkab一样,都是沙漠牧场和冲积耕地结合的地区,因此作者举例是为了说明沙漠牧场和冲积耕地,同样能够促进孟斐斯的繁荣,为定居提供有利条件,这也是孟斐斯选址所考虑的一个重要原因。所以B选项正确。

        A选项:举例说明孟斐斯早期的繁荣水平。错误,因为作者举这两个城市的例子,并不是为了说明早期的孟斐斯有多么的繁荣,而是为了强调它们拥有相似的地理特征,这一地理特征对城市繁荣非常有利。故A选项排除。

        C选项:建造者是参照这两座城市来建造孟斐斯的。错误,因为文章没有提及该信息。故C排除。

        D选项:说明在该地沙漠牧场和冲积耕地的结合非常普遍。错误,文章没有提及该信息,而且这不是作者举这两个城市的例子的主要目的。故D排除。所以本题选B。
    \end{nlz}
\end{blk}
