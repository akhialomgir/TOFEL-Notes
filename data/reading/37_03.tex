\section{TPO 37 Passage 03}

\subsection{Words}

\begin{tabular}{lll}
    utilitarian & n.adj. & 功利主义者;实用主义者;有效用;实用 \\
    undisguised & adj.   & 无争议的               \\
    unadorned   & adj.   & 未装饰的;朴素的           \\
    aesthetic   & adj.   & 美感的;审美的            \\
    masonry     & n.     & 砖,石料               \\
    brick       & n.     & 砖                  \\
    premise     & n.     & 假定,前提              \\
    hallmark    & n.     & 特征,特点              \\
    balcony     & n.     & 阳台                 \\
    residence   & n.     & 住所;住房;宅第           \\
    signature   & n.     & 签名                 \\
    thrust      & v.     & 推挤;刺;戳;插入          \\
    standpoint  & n.     & 观点,立场              \\
    retardant   & n.adj. & 阻滞剂,阻化剂;起阻滞作用的,阻止的 \\
    hollow      & adj.   & 空的,空心的             \\
    tile        & n.     & 瓦片;瓷砖              \\
    pervasive   & adj.   & 充斥各处的;弥漫的,遍布的      \\
    impervious  & adj.   & 不能渗透的              \\
    lateral     & adj.   & 侧面的;横向运动的          \\
\end{tabular}

\subsection{Collocation}

\newpage

\subsection{Analyse}

\begin{blk}
    \begin{qst}
        Q3. Why does the author mention that Le Corbusier included “\textbf{photographs of American factories and grain storage silos, as well as ships, airplanes, and other industrial objects}” in Toward a New Architecture?
    \end{qst}

    \begin{chc}
        B. To support the claim that modern architecture was influenced by practical structures and the ways they were built
    \end{chc}

    \begin{psgq}
        \textbf{The development of modern architecture might in large part be seen as an adaptation of this sort of functional building and its pervasive application for daily use.} \textbf{Indeed}, in his influential book Toward a New Architecture, the Swiss architect Le Corbusier illustrated his text with \textbf{photographs of American factories and grain storage silos, as well as ships, airplanes, and other industrial objects}. \textbf{Nonetheless}, modern architects did not simply employ these new materials in a strictly practical fashion—they consciously exploited their aesthetic possibilities.
    \end{psgq}

    \begin{nlz}
        application for daily use 对应 practical structures
    \end{nlz}
\end{blk}

\begin{blk}
    \begin{qst}
        Q8. According to paragraph 3, which of the following is true of steel-frame buildings?
    \end{qst}

    \begin{chc}
        C. They have greater lateral strength than masonry buildings.
    \end{chc}

    \begin{psgq}
        and of inadequate \textbf{lateral strength} to combat wind shear
    \end{psgq}

    \begin{nlz}
        前文提到masonry buildings没有横向力来应对风的剪切,而后面说steel-frame buildings恰恰相反,说明steel-frame buildings有更大的横向力
    \end{nlz}
\end{blk}

\begin{blk}
    \begin{qst}
        Q14. summary
    \end{qst}

    \begin{chc}
        B. In his influential book Toward a New Architecture, Le Corbusier argued that builders and engineers ought to lead a new revolution in building design.
    \end{chc}

    \begin{nlz}
        F 属于细节而非主干
    \end{nlz}

    \begin{chc}
        C. Modern architects did not accept the traditional distinction between “fine” architecture and buildings that used ordinary materials and a utilitarian design.
    \end{chc}

    \begin{nlz}
        T 第一段内容
    \end{nlz}
\end{blk}

\newpage