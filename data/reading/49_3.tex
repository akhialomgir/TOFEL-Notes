\section{TPO 49 Passage 03}

\subsection{Words}

\begin{tabular}{lll}
    potential    & n.   & 潜力,潜能            \\
    canal        & n.   & 运河               \\
    peasant      & n.   & 农民               \\
    entrepreneur & n.   & (尤指涉及风险的)企业家,创业者 \\
    thrift       & n.   & 节约,节俭            \\
    luxurious    & adj. & 奢华的              \\
    marsh        & n.   & 沼泽;湿地            \\
    fen          & n.   & 沼泽地带,湿地          \\
    livestock    & n.   & 牲畜;家畜;家禽         \\
    burgeoning   & adj. & 迅速发展的            \\
    enclosure    & n.   & 围起来的区域;围场;围地     \\
    pasture      & n.   & 牧场               \\
    fence        & n.   & 栅栏;篱笆;围栏         \\
    flock        & v.   & 聚集;蜂拥            \\
    workforce    & n.   & 劳动力;工人;劳动人口      \\
\end{tabular}

\subsection{Collocation}

\begin{tabular}{ll}
    conducive to & 有利于        \\
    sell out     & 销售一空,售罄;脱销 \\
\end{tabular}

\subsection{Analyse}

\begin{blk}
    \begin{qst}
        Q10. Paragraph 6 suggests which of the following about land enclosure?
    \end{qst}

    \begin{chc}
        A. It entered a period of steady decline after 1820.

        B. It was a farming reform caused by industrialization.

        C. It included a range of agricultural activities by the eighteenth century.

        D. It was primarily used to provide sheep pastures in the sixteenth century.
    \end{chc}

    \begin{psgq}
        Much of the increased production was consumed by Great Britain’s burgeoning population. At the same time, people were moving to the city, partly because of the enclosure movement; that is, the fencing of common fields and pastures in order to provide more compact, efficient privately held agricultural parcels that would produce more goods and greater profits. In the sixteenth century enclosures were usually \textbf{used for creating sheep pastures}, but by the eighteenth century new farming techniques made it advantageous for large landowners to seek enclosures in order to improve agricultural production. Between 1714 and 1820 over 6 million acres of English land were enclosed. As a result, many small, independent farmers were forced to sell out simply because they could not compete. Nonlandholding peasants and cottage workers, who worked for wages and grazed cows or pigs on the village common, were also hurt when the common was no longer available. It was such people who began to flock to the cities seeking employment and who found work in the factories that would transform the nation and, the world.
    \end{psgq}

    \begin{nlz}
        本段第三句 In the …… sheep pastures与D选项完全对应,所以D选项正确。A选项steady decline文中未提及,B选项说这是由工业化引起的,文中未提及,C选项a range of agricultural activities文中未提及。
    \end{nlz}
\end{blk}

\begin{blk}
    \begin{qst}
        Q11. According to paragraph 6, \textbf{the growth of the workforce in British factories} was influenced by
    \end{qst}

    \begin{chc}
        A. the competition for jobs between established and new city inhabitants

        B. a decrease in the farming profits of large landowners

        C. the failure of small independent farms

        D. an attempt by large landowners to take control of the cities
    \end{chc}

    \begin{psgq}
        Much of the increased production was consumed by Great Britain’s burgeoning population. At the same time, people were moving to the city, partly because of the enclosure movement; that is, the fencing of common fields and pastures in order to provide more compact, efficient privately held agricultural parcels that would produce more goods and greater profits. In the sixteenth century enclosures were usually used for creating sheep pastures, but by the eighteenth century new farming techniques made it advantageous for large landowners to seek enclosures in order to improve agricultural production. Between 1714 and 1820 over 6 million acres of English land were enclosed. As a result, many small, independent farmers were forced to sell out simply because they could not compete. Nonlandholding peasants and cottage workers, who worked for wages and grazed cows or pigs on the village common, were also hurt when the common was no longer available. \textbf{It was such people who began to flock to the cities seeking employment and who found work in the factories} that would transform the nation and, the world.
    \end{psgq}

    \begin{nlz}
        XXX
    \end{nlz}
\end{blk}

\begin{blk}
    \begin{qst}
        Q13. Insert
    \end{qst}

    \begin{chc}
        Cities would not only provide \textbf{job} opportunities but also profoundly affect social patterns, standards of living, political movements, and ideologies.
    \end{chc}

    \begin{psgq}
        Much of the increased production was consumed by Great Britain’s burgeoning population. At the same time, people were moving to the city, partly because of the enclosure movement; that is, the fencing of common fields and pastures in order to provide more compact, efficient privately held agricultural parcels that would produce more goods and greater profits. In the sixteenth century enclosures were usually used for creating sheep pastures, but by the eighteenth century new farming techniques made it advantageous for large landowners to seek enclosures in order to improve agricultural production. Between 1714 and 1820 over 6 million acres of English land were enclosed.$\blacksquare$ As a result, many small, independent farmers were forced to sell out simply because they could not compete.$\blacksquare$ Nonlandholding peasants and cottage workers, who worked for wages and grazed cows or pigs on the village common, were also hurt when the common was no longer available.$\blacksquare$ It was such people who began to flock to the cities seeking employment and who found work in the factories that would transform the nation and, the world.$\blacksquare$
    \end{psgq}

    \begin{nlz}
        待插入的句子意思为,城市不仅提供就业机会,还会深刻影响到社会形态,生活标准,政治运动,以及意识形态。前文应该在讨论城市,文中只有D选项前提到城市,所以D选项正确。A选项之后是as a result,所以A选项错误。A选项后描述了大地主圈地导致独立农场主和雇佣劳动者失业,所以B选项错误。而C选项后such people指代前文两种人,所以C选项错误。
    \end{nlz}
\end{blk}