\section{TPO 7 Passage 2}

\subsection{Words}

\begin{tabular}{lll}
    cohesiveness & n.   & 凝聚力              \\
    peculiarly   & adv. & 奇怪地,古怪地;非常;特别,尤其 \\
    assemble     & v.   & 组装;装配            \\
    imperative   & adj. & 极重要的;紧急的;迫切的     \\
    novelty      & n.   & 创新               \\
    speculative  & adj. & 猜测的;推测的,推断的      \\
\end{tabular}

\subsection{Analyse}

\begin{blk}
    \begin{qst}
        Q10.According to paragraph 4, intellectual Romans such as Horace held which of the following opinions about their civilization?
    \end{qst}

    \begin{chc}
        A. Ancient works of Greece held little value in the Roman world.

        B. The Greek civilization had been surpassed by the Romans.

        C. Roman civilization produced little that was original or memorable.

        D. Romans valued certain types of innovations that had been ignored by ancient Greeks.
    \end{chc}

    \begin{psgq}
        Modern attitudes to Roman civilization range from the infinitely impressed to the thoroughly disgusted. As always, there are the power worshippers, especially among historians, who are predisposed to admire whatever is strong, who feel more attracted to the might of Rome than to the subtlety of Greece. At the same time, there is a solid body of opinion that dislikes Rome. For many, Rome is at best the imitator and the continuator of Greece on a larger scale. Greek civilization had quality; Rome, mere quantity. Greece was original; Rome, derivative. Greece had style; Rome had money. Greece was the inventor; Rome, the research and development division. Such indeed was the opinion of some of the more intellectual Romans. “Had the Greeks held novelty in such disdain as we,” asked Horace in his epistle, “what work of ancient date would now exist?”
    \end{psgq}

    \begin{nlz}
        C 以人名做关键词定位至最后一句,这句话一开始就有个 such,所以上文才是 Horace 的观点,前面不停说希腊是原创者,罗马只是跟着学,所以答案是 C,罗马没什么有价值的东西,A/B 都说反了;D 不同类型没说

        从前文中可以看到关于希腊创造罗马照搬的叙述,可以据此推断
    \end{nlz}
\end{blk}

\begin{blk}
    \begin{qst}
        Q12.Which of the following statements about leading Roman soldiers and statesmen is supported by paragraphs 5 and 6?
    \end{qst}

    \begin{chc}
        A. They could read and write the Greek language.

        B. They frequently wrote poetry and plays.

        C. They focused their writing on military matters.

        D. They wrote according to the philosophical laws of the Greeks.
    \end{chc}

    \begin{psgq}
        Rome’s debt to Greece was enormous. The Romans adopted Greek religion and moral philosophy. In literature, Greek writers were consciously used as models by their Latin successors. It was absolutely accepted that an educated Roman should be fluent in Greek. In speculative philosophy and the sciences, the Romans made virtually no advance on early achievements.

        Yet it would be wrong to suggest that Rome was somehow a junior partner in Greco-Roman civilization. The Roman genius was projected into new spheres—especially into those of law, military organization, administration, and engineering. Moreover, the tensions that arose within the Roman state produced literary and artistic sensibilities of the highest order. It was no accident that many leading Roman soldiers and statesmen were writers of high caliber.
    \end{psgq}

    \begin{nlz}
        A,以 Roman soldiers and statesman 做关键词定位至\textbf{第六段最后一句,说这些人是很有才的作家},然后似乎没答案,但问的是根据第五和第六两段啊,上一段还没看,上段倒数第二句说受过良好教育的罗马人都精通希腊语,所以上面说的那些有才的作家肯定也精通了,A 正确;没找到的话用排除法,其他三项都没说
    \end{nlz}
\end{blk}
