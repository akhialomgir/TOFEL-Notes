\section{TPO 52 Passage 01}

\subsection{Words}

\begin{tabular}{lll}
    competence     & n.   & 能力;才干;水平            \\
    wane           & n.   & 衰减;减弱               \\
    silt           & n.   & (尤指河湾或河口处的)淤泥,泥沙    \\
    tributary      & n.   & (河流或湖泊的)支流          \\
    choke          & v.   & (使)窒息;(使)哽噎;(使)呼吸困难 \\
    boulder        & n.   & 巨石                  \\
    cobble         & n.   & (老式路面用的)鹅卵石         \\
    abrasion       & n.   & 磨损,磨耗;擦伤            \\
    alluvial       & adj. & 冲积的,淤积的             \\
    transient      & adj. & 短暂的,转瞬即逝的;暂时的       \\
    mound          & n.   & 土堆,沙石堆;小丘           \\
    inertia        & n.   & 缺乏活力,惰性             \\
    gravelly       & adv. & 像砂砾的;含碎石的           \\
    braid          & n.   & (作装饰用的)穗带,镶边        \\
    simultaneously & adv. & 同时地                 \\
    glacier        & n.   & 冰川,冰河               \\
\end{tabular}

\subsection{Collocation}

\begin{tabular}{ll}
    collocation & 中文 \\
\end{tabular}

\subsection{Analyse}

\begin{blk}
    \begin{qst}
        Q3. According to paragraph 1, all of the following are true of stream sorting EXCEPT:
    \end{qst}

    \begin{chc}
        A. Most of the particles in mountain streams pile up behind boulders and cobbles.

        B. When particles of different sizes settle in a place, the smaller ones sit atop the larger ones.

        C. There are generally more large particles upstream than downstream in a river.

        D. In some situations, downstream particles are created from rocks that eroded as they traveled downstream.
    \end{chc}

    \begin{psgq}
        A large, swift stream or river can carry all sizes of particles, from clay to boulders. When the current slows down, its competence (how much it can carry) decreases and the stream deposits the largest particles in the streambed. If current velocity continues to decrease—as a flood wanes, for example—finer particles settle out on top of the large ones. Thus, a stream sorts its sediment according to size. A waning flood might deposit a layer of gravel, overlain by sand and finally topped by silt and clay. Streams also sort sediment in the downstream direction. Many mountain streams are choked with boulders and cobbles, but far downstream, their deltas are composed mainly of fine silt and clay. This downstream sorting is curious because stream velocity generally increases in the downstream direction. Competence increases with velocity, so a river should be able to transport larger particles than its tributaries carry. One explanation for downstream sorting is that abrasion wears away the boulders and cobbles to sand and silt as the sediment moves downstream over the years. Thus, only the fine sediment reaches the lower parts of most rivers.
    \end{psgq}

    \begin{nlz}
        本题为否定事实信息题。我们可以从选项中找到关键词,回到原文中进行定位用排除法解题。 A选项,根据关键词“boulders and cobbles”定位到第一段中的这一句“Many mountain streams are choked with boulders and cobbles, but far downstream, their deltas are composed mainly of fine silt and clay.”这句话的意思是“许多山间的溪流会被巨砾和鹅卵石阻塞,但在更远处的下游,三角洲主要是由细泥和黏土堆积成的。”但是A选项说大多数山溪中的颗粒都在“boulders and cobbles”后面堆积起来了,这与原文是矛盾的,故A选项为正确答案。 B选项,根据关键词“atop”和“large”定位到第一段第3句话“If current velocity continues to decrease—as a flood wanes, for example—finer particles settle out on top of the large ones.”小颗粒确实会推挤在大颗粒上,故B选项符合原文,排除。 C选项,根据关键词“downstream”定位到这句话“Many mountain streams are choked with boulders and cobbles, but far downstream, their deltas are composed mainly of fine silt and clay.”这句话说,山溪通常被巨砾和鹅卵石阻塞,而下游的三角洲主要是由细泥和黏土堆积成的。很明显,上游的巨砾要比下游的细颗粒大。故C选项符合原文,排除。 D选项定位到倒数第2句话“One explanation for downstream sorting is that abrasion wears away the boulders and cobbles to sand and silt as the sediment moves downstream over the years.” 随着沉淀物年复一年地向下游移动,水流的摩擦力将巨砾和卵石磨成了沙子和淤泥。选项中的“eroded”对应原文中的“abrasion”,故D选项也符合原文,排除。
    \end{nlz}
\end{blk}

\begin{blk}
    \begin{qst}
        Q5. According to paragraph 2, which of the following is true about bars in streams?
    \end{qst}

    \begin{chc}
        A. They start forming in the stream channel and then expand over the banks.

        B. They seldom form in rivers that are used for commercial navigation.

        C. They tend to grow longer each year.

        D. They often last no more than a year.
    \end{chc}

    \begin{psgq}
        A stream deposits its sediment in three environments: Alluvial fans and deltas form where stream gradient (angle of incline) suddenly decreases as a stream enters a flat plain, a lake, or the sea; floodplain deposits accumulate on a floodplain adjacent to the stream channel; and channel deposits form in the stream channel itself. Bars, which are elongated mounds of sediment, are transient features that form in the stream channel and on the banks. They commonly form in one year and erode the next. Rivers used for commercial navigation must be recharged frequently because bars shift from year to year. Imagine a winding stream. The water on the outside of the curve moves faster than the water on the inside. The stream erodes its outside bank because the current’s inertia drives it into the outside bank. At the same time, the slower water on the inside point of the bend deposits sediment, forming a point bar. A mid-channel bar is a sandy and gravelly deposit that forms in the middle of a stream channel.
    \end{psgq}

    \begin{nlz}
        本题为事实信息题。根据题干中的关键词“bar”,可以定位到第二段第2句“Bars, which are elongated mounds of sediment, are transient features that form in the stream channel and on the banks.”这句话之后的内容都是在介绍bars(砂坝)。下面我们来看选项: A选项:砂坝一开始在河道中形成,然后延伸到岸上。错误,因为文中说“Bars, which are elongated mounds of sediment, are transient features that form in the stream channel and on the banks.” 砂坝是河流内部和河流两岸所形成的沉淀物的瞬态特征。所以,在河流内和岸上,都可以形成砂坝,这两者是并列关系,不分先后。故A选项与原文不符,排除。 B选项:砂坝很少在用于通上航行的河道中形成。错误,因为文中说“Rivers used for commercial navigation must be recharged frequently because bars shift from year to year.” 用于通商航行的河道必须频繁地修整,因为砂坝每年都会变。这说明通上航行的河道中每年都会形成砂坝,故B选项与原文矛盾,排除。 C选项:它们每年都会增长。错误,因为文中说“They commonly form in one year and erode the next.” 砂坝通常在一年内形成,下一年就会被消磨掉。它们的位置可能会变换,但是不会增长,因为第二年就被消磨掉了。故C选项与原文不符,排除。 D选项:它们通常存在不超过一年。正确,出处同C选项“They commonly form in one year and erode the next.”这句话恰恰能说明砂坝存在不会超过1年的时间,故D选项符合原文,为正确答案。
    \end{nlz}
\end{blk}

\begin{blk}
    \begin{qst}
        Q8. Why does the author include the information that “Glaciers grind bedrock into fine sediment, which is carried by streams flowing from the melting ice”?
    \end{qst}

    \begin{chc}
        A.
        To give a reason why heavily sedimented braided streams are common in glacial environments

        B.
        To explain why some mountain streams deposit most of their sediment in a fan-shaped mound

        C.
        To identify the most common source of sediment in arid and semiarid mountainous regions

        D.
        To help explain why glacial sediment decreases the gradient and velocity of steep mountain streams
    \end{chc}

    \begin{psgq}
        Most streams flow in a single channel. In contrast, a braided stream flows in many shallow, interconnecting channels. A braided stream forms where more sediment is supplied to a stream than it can carry. The stream dumps the excess sediment, forming mid-channel bars. The bars gradually fill a channel, forcing the stream to overflow its banks and erode new channels. As a result, a braided stream flows simultaneously in several channels and shifts back and forth across its floodplain. Braided streams are common in both deserts and glacial environments because both produce abundant sediment. A desert yields large amounts of sediment because it has little or no vegetation to prevent erosion. \textbf{Glaciers grind bedrock into fine sediment, which is carried by streams flowing from the melting ice.} If a steep mountain stream flows onto a flat plain, its gradient and velocity decrease sharply. As a result, it deposits most of its sediment in a fan-shaped mound called an alluvial fan. Alluvial fans are common in many arid and semiarid mountainous regions.
    \end{psgq}

    \begin{nlz}
        本题为功能目的题。题目问:为什么作者会提到“冰川将基岩研磨成细小的沉积物,这些沉淀物会被冰川融化后形成的水流所携带。”这一信息?我们在文中找到这句话,然后看它的上下文内容。后文是在讲冲积扇的形成过程,与冰川关系不大,所以我们往前看。前一句话“A desert yields large amounts of sediment because it has little or no vegetation to prevent erosion.”这句话是在解释为什么沙漠地带能产生大量沉积物。再往前看一句“Braided streams are common in both deserts and glacial environments because both produce abundant sediment.”这句话是说辫状河在沙漠和冰川环境中都很常见,因为沙漠和冰川都能产生大量的沉淀物。至此,这段话的结构已经非常明显:即先说明沙漠和冰川都能产生大量沉积物,然后再各用一句话解释它们为什么能产生沉积物。故这里A选项:解释为什么含有大量泥沙的辫状河在冰川环境很常见,正确。 B选项:解释为什么一些山溪会将泥沙沉淀在冲积扇地区。错误,因为冲积扇和题干这句话的关系不大,是下文新展开的一个层次,故B选项排除。 C选项:解释干旱和半干旱山区最常见的沉淀物来源。错误,因为题干这句话的目的,不是为了解释干旱和半干旱山区沉淀物来源,而是为了说明冰川和前文所提到的辫状河之间的联系,解释为什么辫状河在冰川环境下很常见。故C选项错误。 D选项:解释为什么冰川沉淀物会降低陡峭的山溪的坡度和水流速度。错误,因为文中并未提到该信息,故排除。
    \end{nlz}
\end{blk}

\begin{blk}
    \begin{qst}
        Q12. According to paragraph 4, what are engineers trying to accomplish in the Mississippi delta?
    \end{qst}

    \begin{chc}
        A.
        To expand the channels into which the river flows

        B.
        To keep the river flowing in the existing channels

        C.
        To control the amount of sediment the river brings to the delta

        D.
        To increase the part of the delta that lies above water level
    \end{chc}

    \begin{psgq}
        A stream also slows abruptly where it enters the still water of a lake or ocean. The sediment settles out to form a nearly flat landform called a delta. Part of the delta lies above water level, and the remainder lies slightly below water level. Deltas are commonly fan-shaped, resembling the Greek letter “delta” (Δ). Both deltas and alluvial fans change rapidly. Sediment fills channels (waterways), which are then abandoned while new channels develop as in a braided stream. As a result, a stream feeding a delta or fan splits into many channels called distributaries. A large delta may spread out in this manner until it covers thousands of square kilometers. Most fans, however, are much smaller, covering a fraction of a square kilometer to a few square kilometers. The Mississippi River has flowed through seven different delta channels during the past 5,000 to 6,000 years. But in recent years, engineers have built great systems of levees (retaining walls) in attempts to stabilize the channels.
    \end{psgq}

    \begin{nlz}
        本题为事实信息题。根据题干关键词“engineers”直接定位到文章最后一句话“But in recent years, engineers have built great systems of levees (retaining walls) in attempts to stabilize the channels.”这句话说近年来,工程师们建造了堤坝(挡土墙)系统以试图加固河道。我们再往前看,前文内容说,过去的5000到6000年内,密西西比河流经之处,已经形成了7个三角洲。而“But”表转折关系,所以工程师加固河道的目的显而易见,就是希望密西西比河不要再产生支流,形成新的三角洲。故B选项:让密西西比河在已有的河道中流淌,正确。 A选项:扩张河道。错误,因为文中已经出现了“but”这个表示转折的逻辑词,再扩张河道,在逻辑上行不通。 C选项:控制密西西比河带到三角洲的泥沙量。但文中并没有说加固河道和控制泥沙量有什么联系,故C选项未提及,排除。 D选项:增加三角洲在水上的部分。同理,该信息在文中未提及,排除。
    \end{nlz}
\end{blk}