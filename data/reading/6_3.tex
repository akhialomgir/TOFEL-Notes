\section{TPO 6 Passage 3}

\subsection{Words}

\begin{tabular}{lll}
    repression & n.   & (尤指通过武力进行的)镇压,压制 \\
    explicit   & adj. & 清楚明白的;明确的;不含糊的   \\
\end{tabular}

\subsection{Analyse}

\begin{blk}
    \begin{qst}
        Q5.What does paragraph 3 suggest about \textbf{long-term memory} in children?
    \end{qst}

    \begin{chc}
        A. Maturation of the frontal lobes of the brain is important for the long-term memory of motor activities but not verbal descriptions.

        B. Young children may form long-term memories of actions they see earlier than of things they hear or are told.

        C. Young children have better long-term recall of short verbal exchanges than of long ones.

        D. Children’s long-term recall of motor activities increases when such activities are accompanied by explicit verbal descriptions.
    \end{chc}

    \begin{psgq}
        Three other explanations seem more promising. One involves physiological changes relevant to memory. Maturation of the frontal lobes of the brain continues throughout early childhood, and this part of the brain may be critical for remembering particular episodes in ways that can be retrieved later. Demonstrations of infants’ and toddlers' \textbf{long-term memory} have involved their repeating motor activities that they had seen or done earlier, such as reaching in the dark for objects, putting a bottle in a doll’s mouth, or pulling apart two pieces of a toy. The brain’s level of physiological maturation may support these types of memories, but not ones requiring explicit verbal descriptions.
    \end{psgq}

    \begin{nlz}
        B 以 long-term memory 定位至倒数第二句,说婴儿会重复他们看到的动作,接着就说大脑成熟导致他们能形成关于这些的记忆,但那些需要清楚解释的不行,也就是这个阶段还不能记住听到的东西,所以 B 说看到的比听到的早,正确。A 错,没说对 verbal description 不重要;C/D 都没说。
    \end{nlz}
\end{blk}

\begin{blk}
    \begin{qst}
        Q10.According to paragraphs 5 and 6, one \textbf{disadvantage very young children face in processing information} is that they cannot
    \end{qst}

    \begin{chc}
        A. process a lot of information at one time

        B. organize experiences according to type

        C. block out interruptions

        D. interpret the tone of adult language
    \end{chc}

    \begin{psgq}
        This view is supported by a variety of factors that can create mismatches between very young children's encoding and older children's and adults' retrieval efforts. The world looks very different to a person whose head is only two or three feet above the ground than to one whose head is five or six feet above it. Older children and adults often try to retrieve the names of things they saw, but \textbf{infants would not have encoded the information verbally. General knowledge of categories of events such as a birthday party or a visit to the doctor's office helps older individuals encode their experiences, but again, infants and toddlers are unlikely to encode many experiences within such knowledge structures.}
    \end{psgq}

    \begin{nlz}
        第五段和第六段都在说成人和大孩子与婴儿解析信息的方式不同,第六段给出了具体例子,最后一句中出答案:General knowledge of categories of events such as a birthday party or a visit to the doctor's office helps older individuals encode their experiences, but again, infants and toddlers are unlikely to encode many experiences within such knowledge structures.说成人和大孩子关于类别的常识可以帮他们解析信息,但小孩子不能,也就是小孩子不会分类,所以 B 是答案,其他都没说。
    \end{nlz}
\end{blk}

\begin{blk}
    \begin{qst}
        Q11.Which of the sentences below best expresses the essential information in the highlighted sentence in the passage? Incorrect choices change the meaning in important ways or leave out essential information.
    \end{qst}

    \begin{chc}
        A. Incomplete physiological development may partly explain why hearing stories does not improve long-term memory in infants and toddlers.

        B. One reason why preschoolers fail to comprehend the stories they hear is that they are physiologically immature.

        C. Given the chance to hear stories, infants and toddlers may form enduring memories despite physiological immaturity.

        D. Physiologically mature children seem to have no difficulty remembering stories they heard as preschoolers.
    \end{chc}

    \begin{psgq}
        Physiological immaturity may be part of \textbf{why} infants and toddlers do not form extremely enduring memories, \textbf{even when} they hear stories that promote such remembering in preschoolers .
    \end{psgq}

    \begin{nlz}
        A : 这个句子比较简单,主要逻辑是让步状语从句。说 immaturity 是 do not form enduring memories 的原因,即使听了 stories,A 正确。B 错在改变了原文结构,原文的结果是 do not form enduring memories,B 改成了 comprehend stories;C 与原文 even 之后的部分相反;D 没说,注意不能推断。
    \end{nlz}
\end{blk}

\begin{blk}
    \begin{qst}
        Q14
    \end{qst}

    \begin{chc}
        C.Children recall physical activities more easily if they are verbalized. XXX

        D.The opportunity to hear chronologically narrated stories may help three-year-old children produce long-lasting memories.
    \end{chc}

    \begin{nlz}
        B.D.F preschoolers 选项原文没说,不选。frontal lobe 选项对应原文第三段第二句,正确。children 选项原文没说,不选。the opportunity 选项对应原文第四段第二句和倒数第二句,正确。the content 选项原文没说,不选。the contrasting ways 选项对应原文第五段第一句,正确。

        C段没有读懂 原因为那一段没有读完 D认为不是主旨 然而段中有
    \end{nlz}
\end{blk}
