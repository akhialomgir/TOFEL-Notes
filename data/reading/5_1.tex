\section{TPO 5 Passage 1}

\subsection{Words}

\begin{tabular}{lll}
    notoriously & adv. & 臭名昭著地,声名狼藉地 \\
    symptom     & n.   & 症状          \\
    scenario    & n.   & 可能发生的事态;设想  \\
    excavation  & n.   & (古物的)发掘     \\
\end{tabular}

\subsection{Analyse}

\begin{blk}
    \begin{qst}
        Q14
    \end{qst}

    \begin{chc}
        B.Though beneficial in lower levels, high levels of salts, other minerals, and heavy metals can be harmful to plants.

        D.\textbf{Because} high concentrations of \textbf{sodium chloride} and other salts limit growth in most plants, much research has been done in an effort to develop salt-tolerant agricultural crops.
    \end{chc}

    \begin{nlz}
        A.B.C 从文章题目可以大致推测可能会有几种方向。物质对植物的重要性几个方面的描述/为什么特别重要解释/缺乏物质或过多的解决方法。

        首段大致说的是物质对植物的重要性。

        二段描述物质缺乏的症状。

        三段补充介绍二段所说物质缺乏的研究方法:控制变量

        四段说物质过多对植物的危害。

        五段特定植物能用来吸收物质,实际说的是问题的可能解决方案。

        末段说的是五段解决方案的具体操作,最新进展。

        引出句概括的是首段内容。正确选项应概括后段内容。

        A(some plants are)选项对应原文第五段和末段,正确。

        B(though)选项对应原文第四段,正确。

        C(when)选项对应原文第二段,正确。

        D(because)选项中的\textbf{因果关系}原文没说,而且\textbf{氯化钠也是个细节},不选。

        E(some plants can)选项没说,不选。

        F(mineral deficiency)选项是原文第三段末的一个细节,不选。
    \end{nlz}
\end{blk}
