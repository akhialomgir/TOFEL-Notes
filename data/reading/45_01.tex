\section{TPO 45 Passage 01}

\subsection{Words}

\begin{tabular}{lll}
    remnant   & n.   & 残余(部分);剩余(部分);零头;零料 \\
    fauna     & n.   & 动物群                 \\
    flora     & n.   & 植物群                 \\
    dune      & v.   & 预期;期望;要求            \\
    tundra    & n.   & 冻原                  \\
    sparse    & adj. & 稀少的                 \\
    glaciated & adj. & 由冰川作用形成的;冰川覆盖的      \\
    moisture  & n.   & 潮气,湿气;水分            \\
    herd      & n.v. & 群;放牧                \\
\end{tabular}

\subsection{Collocation}


\subsection{Analyse}

\begin{blk}
    \begin{qst}
        Q4. The purpose of paragraph 3 is to
    \end{qst}

    \begin{chc}
        B. describe the \textbf{Beringian landscape} during the last ice age

        D. summarize the information about Beringia that \textbf{historians agree on}
    \end{chc}

    \begin{psgq}
        The \textbf{Beringian landscape} was very different from what it is today. Broad, windswept valleys; glaciated mountains; sparse vegetation; and less moisture created a rather forbidding land mass. This land mass supported herds of now-extinct species of mammoth, bison, and horse and somewhat modern versions of caribou, musk ox, elk, and saiga antelope. These grazers supported in turn a number of impressive carnivores, including the giant short-faced bear, the saber-tooth cat, and a large species of lion.
    \end{psgq}

    \begin{nlz}
        Beringian landscape与B对应,D中没有提到historians
    \end{nlz}
\end{blk}

\begin{blk}
    \begin{qst}
        Q12. Which of the following best describes the organization of paragraph 6?
    \end{qst}

    \begin{chc}
        B. An argument is offered, and reasons both for and against the argument are presented.

        D. New information is presented, and the information is used to show that two competing explanations can each be seen as correct in some way.
    \end{chc}

    \begin{psgq}
        The argument seemed to be at a standstill until a number of recent \textbf{studies resulted in a spectacular suite of new finds}. \textbf{The first} was the discovery of a 1,000-square-kilometer preserved patch of Beringian vegetation dating to just over 17,000 years ago—the peak of the last ice age. The plants were preserved under a thick ash fall from a volcanic eruption. Investigations of the plants found grasses, sedges, mosses, and many other varieties in a nearly continuous cover, as was predicted by Guthrie. \textbf{But} this vegetation had a thin root mat with no soil formation, demonstrating that there was little long-term stability in plant cover, a finding supporting some of the arguments of Colinvaux. A mixture of continuous but thin vegetation supporting herds of large mammals is one that seems plausible and realistic with the available data.
    \end{psgq}

    \begin{nlz}
        第6段先明确说recent studies resulted in new findings. 然后又说the first… as was predicted by Guthrie. 即说明此发现是与Guthrie的观点吻合。之后又说But…., a finding supporting some of the arguments of Colinvaux.也就是说此发现也说明Colinvaux的部分观点也是对的。因此D选项正确。
    \end{nlz}
\end{blk}

\begin{blk}
    \begin{qst}
        Q13. insert
    \end{qst}

    \begin{chc}
        \textbf{Nevertheless}, \textbf{large animals} managed to survive in Beringia.
    \end{chc}

    \begin{psgq}
        The Beringian landscape was very different from what it is today. $\blacksquare$ Broad, windswept valleys; glaciated mountains; sparse vegetation; and less moisture created a rather forbidding land mass. $\blacksquare$ This land mass supported herds of now-extinct species of mammoth, bison, and horse and somewhat modern versions of caribou, musk ox, elk, and saiga antelope. $\blacksquare$ \textbf{These grazers} supported in turn a number of impressive carnivores, including the giant short-faced bear, the saber-tooth cat, and a large species of lion. $\blacksquare$
    \end{psgq}

    \begin{nlz}
        插入的句子中Nevertheless表示转折,A处的句子说明在Beringia地区的环境是非常forbidding(恶劣的), 插入句说“不管怎样,大型到你动物还是设法在Beringia地区生存了下来”,正好与A处的表述构成转折且下文的C处和D处具体说明了大型动物是如何生存的。These grazers和上一句中逻辑紧密,所以C位置不能分隔,B位置正确。
    \end{nlz}
\end{blk}
