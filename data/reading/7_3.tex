\section{TPO 7 Passage 3}

\subsection{Words}

\begin{tabular}{lll}
    receptiveness & n.   & (对于新思想和建议)乐于接受,从善如流 \\
    penetrate     & v.   & 穿透;进入;渗入            \\
    ingenious     & adj. & (物品等)制作精巧的;(方法等)巧妙的 \\
    desiccation   & n.   & 干旱                  \\
\end{tabular}

\subsection{Analyse}

\begin{blk}
    \begin{qst}
        Q6.What function does paragraph 3 serve in the organization of the passage as a whole?
    \end{qst}

    \begin{chc}
        A. It contrasts the development of iron technology in West Asia and West Africa.

        B. It discusses a non-agricultural contribution to Africa from Asia.

        C. It introduces evidence that a knowledge of copper working reached Africa and Europe at the same time.

        D. It compares the rates at which iron technology developed in different parts of Africa.
    \end{chc}

    \begin{psgq}
        Iron came from West Asia, although its routes of diffusion were somewhat different than those of agriculture. Most of Africa presents a curious case in which societies moved directly from a technology of stone to iron without passing through the intermediate stage of copper or bronze metallurgy, although some early copper-working sites have been found in West Africa. Knowledge of iron making penetrated into the forest and savannahs of West Africa at roughly the same time that iron making was reaching Europe. Evidence of iron making has been found in Nigeria, Ghana, and Mali.
    \end{psgq}

    \begin{nlz}
        B 问整段在文章中的作用,看首句。说铁也是从西亚来的,但跟农业扩散的线路不同,所以答案是 B,亚洲对非洲的除农业之外的影响;原文没有比较西非和西亚的 iron 技术,重点是描述非洲的,所以 A 错;C 和 D 只是段落中的小细节,不足以呈现整段的作用。

        段落主旨句为首句,与上一段主旨句相关
    \end{nlz}
\end{blk}
