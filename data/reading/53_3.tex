\section{TPO 53 Passage 03}

\subsection{Words}

\begin{tabular}{lll}
    triumph    & n.   & 巨大成功;胜利;狂喜          \\
    prompt     & v.   & 引起;导致;激起            \\
    repertoire & n.   & 全部剧目,全部节目,全部曲目      \\
    daub       & v.   & (随意)涂抹,乱涂,乱抹        \\
    recess     & n.   & (议会)休会期             \\
    nook       & n.   & 角落;隐蔽处;幽深处          \\
    cranny     & n.   & 裂缝,缝隙               \\
    awkward    & adj. & 难用的;棘手的;难处理的,难对付的   \\
    notion     & n.   & 观念;看法               \\
    ample      & adj. & 足够的,充足的,充裕的         \\
    stark      & adj. & 裸露的;简单的;明显的;严重的;极端的 \\
    diligently & adv. & 勤奋地,认真地             \\
\end{tabular}

\subsection{Analyse}

\begin{blk}
    \begin{qst}
        Q1.
        According to paragraph 1, what is significant about the paintings in the Lascaux caves?
    \end{qst}

    \begin{chc}
        A.
        They provide accurate depictions of the bulls and other animals living in Paleolithic France.

        B.
        They are the best available source of information about daily life during the Paleolithic era.

        C.
        They are some of the best surviving examples of what was possibly one of the world’s earliest artistic movements.

        D.
        They are the only evidence of creative expression among Paleolithic human beings.
    \end{chc}

    \begin{psgq}
        In any investigation of the origins of art, attention focuses on the cave paintings created in Europe during the Paleolithic era (C. 40,000-10,000 years ago) such as those depicting bulls and other animals in the Lascaux cave in France. Accepting that they are the best preserved and most visible signs of what was a global creative explosion, how do we start to explain their appearance? Instinctively, we may want to update the earliest human artists by assuming that they painted for the sheer joy of painting. The philosophers of Classical Greece recognized it as a defining trait of humans to “delight in works of imitation”—to enjoy the very act and triumph of representation. If we were close to a real lion or snake, we might feel frightened. But a well-executed picture of a lion or snake will give us pleasure. Why suppose that our Paleolithic ancestors were any different?
    \end{psgq}

    \begin{nlz}
        本题定位到原文:Accepting that they are the best preserved and most visible signs of what was a global creative explosion这半句话。 此处原文的大意是:在我们接受了它们是保存最好的、最可视化的全球创造大爆发的产物后…... 题干问的是Lascaux cave最重要的是什么。 选项A的意思是它们提供了旧石器时代最精确的对于牛和其他动物的描绘信息,选项B的意思是它们是描述旧石器时代日常生活最有效的信息源,选项C的意思是它们是最早的世界上的艺术运动的最好的存在例子,选项D的意思是它们是旧石器时代人类的创造性表达的唯一证据。只有选项C符合原文。其余三个选项均不合适。
    \end{nlz}
\end{blk}

\begin{blk}
    \begin{qst}
        Q9.
        According to paragraph 3, which of the following is true of the paintings located in the Lascaux caves?
    \end{qst}

    \begin{chc}
        A.
        They are all found in recesses that are difficult for viewers to reach.

        B.
        They fill every nook and cranny of a large underground gallery.

        C.
        Their location was probably more convenient for viewers than for the artists.

        D.
        They are easier to view than cave paintings at other locations.
    \end{chc}

    \begin{psgq}
        A further question to the theory of art for art’s sake is posed by the high incidence of Paleolithic images that appear not to be imitative of any reality whatsoever. These are geometrical shapes or patterns consisting of dots or lines. Such marks may be found isolated or repeated over a particular surface, but also scattered across more recognizable forms. A good example of this may be seen in the geologically spectacular grotto of Pêche Merle, in the Lot region of France. Here we encounter some favorite animals from the Paleolithic repertoire—a pair of stout-bellied horses. But over and around the horses’ outlines are multiple dark spots, daubed in disregard for the otherwise naturalistic representation of animals. What does such patterning imitate? There is also the factor of location. The caves of Lascaux might conceivably qualify as underground galleries, but many other paintings have been found in recesses totally unsuitable for any kind of viewing—tight nooks and crannies that must have been awkward even for the artists to penetrate, let alone for anyone else wanting to see the art.
    \end{psgq}

    \begin{nlz}
        KMF解析:根据题干关键词the paintings located in the Lascaux caves定位原文There is also the factor of location. The caves of Lascaux might conceivably qualify as underground galleries, but many other paintings have been found in recesses totally unsuitable for any kind of viewing—tight nooks and crannies that must have been awkward even for the artists to penetrate, let alone for anyone else wanting to see the art.此外,还有地点的因素。Lascaux 的洞穴或许可以想象,适合作为秘密的画廊,但是在深处发现的许多其他壁画完全不适合任何形式的观赏——狭窄的角落和裂缝,即使对艺术家来说要穿过去,一定都很奇怪,更别说对于任何想要去观看艺术的人了。对应D选项。
    \end{nlz}
\end{blk}

\begin{blk}
    \begin{qst}
        Q13. Insert
    \end{qst}

    \begin{chc}
        Obtaining this level of nourishment from such a harsh environment must have consumed most of Paleolithic people’s time and attention.
    \end{chc}

    \begin{psgq}
        Finally, we may doubt the notion that the Upper Paleolithic period was a paradise in which food came readily, leaving humans ample time to amuse themselves with art.$\blacksquare$ For Europe it was still the Ice Age.$\blacksquare$ An estimate of the basic level of sustenance then necessary for human survival has been judged at 2200 calories per day.$\blacksquare$ This consideration, combined with the stark emphasis upon animals in the cave art, has persuaded some archaeologists that the primary motive behind Paleolithic images must lie with the primary activity of Paleolithic people: hunting.$\blacksquare$
    \end{psgq}

    \begin{nlz}
        本题定位到原文:倒数第二段。 此处原文的大意是:本段主要讲述了旧石器岩画和当时的人们的狩猎行为的关联。 题干问的是“从艰苦的环境里获得这种程度的营养物质想必消耗掉了旧石器人们大多数的时间和注意力”该插入哪个位置。 此题做题的关键是需要插入句子里的this level of nourishment。只有选项C的前面的一句话里提到 2200 calories per day,对应了nourishment,选项C合适。其他三个可插入点都没有可以对应的内容。
    \end{nlz}
\end{blk}

\begin{blk}
    \begin{qst}
        Q14. Summary
    \end{qst}

    \begin{chc}
        A. It is generally agreed that art as imitation arose during the age of Classical Greece.

        B. Paleolithic artists often chose to paint pictures that were intended to frighten people.

        C. People in the Paleolithic era may not have had time for art, and the placement of the paintings does not indicate that they were meant to be looked at.

        D. Paleolithic artists chose to represent only a small segment of the natural world, and their paintings were not always strict imitations of nature.

        E. Hunting was central to Paleolithic life, and animals are central to cave art, leading some to believe that the paintings were created to bring luck to hunters.

        F. Humans were rarely the subjects of cave paintings because it was thought that capturing the image of a hunter would cause the hunter to be virtually trapped.
    \end{chc}

    \begin{nlz}
        选项A无中生有,文中说Greece是为了表明这是他们哲学家的观点,而不是为了说明时间,不选; 选项B的“目的是去恐吓别人”这个信息是无中生有,原文中没有对应的信息,不选; 选项C正确概括了第二段的内容,选; 选项D正确概括了第三段的内容,选; 选项E正确概括了倒数第二段的内容,选; 选项F的因果关系是无中生有的,文中没有给出过人类很少成为绘画对象的原因,不选。
    \end{nlz}
\end{blk}
