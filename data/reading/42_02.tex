\section{TPO 42 Passage 02}

\subsection{Words}

\begin{tabular}{lll}
    supposedly   & adv. & 恐怕               \\
    constipation & n.   & 便秘               \\
    allegedly    & adv. & 据说,据称            \\
    halt         & v.   & 停止,休息            \\
    horn         & n.   & 角;角质             \\
    grind        & v.   & 碾碎,磨碎,把…磨成粉      \\
    shroud       & n.v. & 遮蔽物;罩;幕;遮蔽;覆盖;隐藏 \\
\end{tabular}

\subsection{Collocation}

\begin{tabular}{ll}
    except that & (用于说明为什么某事不可能或者不真实)只是…,不过… \\
\end{tabular}

\subsection{Analyse}

\begin{blk}
    \begin{qst}
        Q1. In paragraph 1, why does the author include a discussion of \textbf{when flowering plants evolved}?
    \end{qst}

    \begin{chc}
        A. To help explain why some scientists believe that the development of flowering plants led to dinosaur extinction

        B. To cast doubt on the theory that the development of flowering plants caused dinosaurs to become extinct
    \end{chc}

    \begin{psgq}
        Many explanations have been proposed for why dinosaurs became extinct. \textbf{For example}, some have blamed dinosaur extinction on the development of flowering plants, which were supposedly more difficult to digest and could have caused constipation or indigestion—\textbf{except that flowering plants first evolved in the Early Cretaceous, about 60 million years before the dinosaurs died out}.
    \end{psgq}

    \begin{nlz}
        文章中提到有花植物出现的时间是为了质疑一些说开花植物导致恐龙灭绝的理论
    \end{nlz}
\end{blk}

\begin{blk}
    \begin{qst}
        Q6. What makes the extinction of “\textbf{the ammonites}” especially significant?
    \end{qst}

    \begin{chc}
        B. They existed at the lowest level of the food chain.

        D. They had \textbf{survived many previous mass extinctions}.
    \end{chc}

    \begin{psgq}
        It wiped out many kinds of plankton in the ocean and many marine organisms that lived on the plankton at the base of the food chain. These included a variety of clams and snails, and especially \textbf{the ammonites}, a group of shelled squidlike creatures that dominated the Mesozoic seas and had \textbf{survived many previous mass extinctions}.
    \end{psgq}

    \begin{nlz}
        survived many previous mass extinctions为D中内容的同义替换
    \end{nlz}
\end{blk}

\begin{blk}
    \begin{qst}
        Q10. How does paragraph 3 relate to paragraph 2?
    \end{qst}

    \begin{chc}
        A. Paragraph 3 provides an alternative explanation \textbf{to the one provided} in paragraph 2.

        B. Paragraph 3 provides an explanation that \textbf{satisfies the conditions set forth} in paragraph 2.
    \end{chc}

    \begin{psgq}
        The Cretaceous extinctions were a global phenomenon, and dinosaurs were just a part of a bigger picture.

        According to one theory, the Age of Dinosaurs ended suddenly 65 million years ago when a giant rock from space plummeted to Earth.
    \end{psgq}

    \begin{nlz}
        第二段主要讲白垩纪灭绝事件是一个全球现象,第三段继续解释另外一个理论,在六千五百万年前,当太空里的一个巨大的石头垂直落向地球时,恐龙时代就突然结束了。继而以火流星坠落展开详细解释。所以可知B选项正确。
    \end{nlz}
\end{blk}

\begin{blk}
    \begin{qst}
        Q13. Insert

        \textbf{Some explanations} seem plausible until the facts are considered.
    \end{qst}

    \begin{psgq}
        Dinosaurs rapidly became extinct about 65 million years ago as part of a mass extinction known as the K–T event, because it is associated with a geological signature known as the K–T boundary, usually a thin band of sedimentation found in various parts of the world (K is the traditional abbreviation for the Cretaceous, derived from the German name Kreidezeit). $\blacksquare$ \textbf{Many explanations} have been proposed for why dinosaurs became extinct. $\blacksquare$ For example, some have blamed dinosaur extinction on the development of flowering plants, which were supposedly more difficult to digest and could have caused constipation or indigestion—\textbf{except that} flowering plants first evolved in the Early Cretaceous, about 60 million years before the dinosaurs died out. $\blacksquare$ In fact, several scientists have suggested that the duckbill dinosaurs and horned dinosaurs, with their complex battery of grinding teeth, evolved to exploit this new resource of rapidly growing flowering plants. $\blacksquare$ Others have blamed extinction on competition from the mammals, which allegedly ate all the dinosaur eggs—except that mammals and dinosaurs appeared at the same time in the Late Triassic, about 190 million years ago, and there is no reason to believe that mammals suddenly acquired a taste for dinosaur eggs after 120 million years of coexistence. Some explanations (such as the one stating that dinosaurs all died of diseases) fail because there is no way to scientifically test them, and they cannot move beyond the realm of speculation and guesswork.
    \end{psgq}

    \begin{nlz}
        For example 后例子对应 seem plausible until the facts are considered
    \end{nlz}
\end{blk}
