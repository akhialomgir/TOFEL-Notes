\section{TPO 44 Passage 01}

\subsection{Words}

\begin{tabular}{lll}
    dwell       & v.     & 居住,栖身                  \\
    terrestrial & aj.    & 地球的;与地球有关的             \\
    requisite   & n.adj. & 必需品;必要条件;必要的;需要的;必不可少的 \\
    sturdy      & adj.   & 结实的,牢固的;强壮的            \\
\end{tabular}

\subsection{Collocation}


\subsection{Analyse}

\begin{blk}
    \begin{qst}
        Q1. Paragraph 1 supports which of the following statements about fish evolution?
    \end{qst}

    \begin{chc}
        B. Fish began living in freshwater habitats only after originating elsewhere.

        C. \textbf{Lobe-finned fish} radiated into almost all available aquatic habitats.
    \end{chc}

    \begin{psgq}
        One of the most significant evolutionary events that occurred on Earth was the transition of water-dwelling fish to terrestrial tetrapods (four-limbed organisms with backbones). Fish probably originated in the oceans, and our first records of them are in marine rocks. However, by the Devonian Period (408 million to 362 million years ago), they had radiated into almost all available aquatic habitats, including freshwater settings. One of the groups whose fossils are especially common in rocks deposited in fresh water is the lobe-finned fish.
    \end{psgq}

    \begin{nlz}
        选项B与原文中鱼类出现的顺序阐述一致;选项C中的lobe-finned fish只是其中的一种。所以选B
    \end{nlz}
\end{blk}

\begin{blk}
    \begin{qst}
        Q13. insert

        \textbf{These} would have been deposited by the receding waters of droughts, during which \textbf{many aquatic animals must have died}.
    \end{qst}

    \begin{psgq}
        Another impetus may have been new sources of food. The edges of ponds and streams surely had \textbf{scattered dead fish} and other water-dwelling creatures. $\blacksquare$ In addition, plants had emerged into terrestrial habitats in areas near streams and ponds, and crabs and other arthropods were also members of this earliest terrestrial community. $\blacksquare$ Thus, by the Devonian the land habitat marginal to freshwater was probably a rich source of protein that could be exploited by an animal that could easily climb out of water. $\blacksquare$ Evidence from teeth suggests that these earliest tetrapods did not utilize land plants as food; they were presumably carnivorous and had not developed the ability to feed on plants. $\blacksquare$
    \end{psgq}

    \begin{nlz}
        These would have been deposited by the receding waters of droughts, during which many aquatic animals must have died. 预期:待插入句子中these 指示代词是有效线索,前面话需要明确these指的是什么。排查:A.后面句子中in addition 与前面句子形成平行结构。B.后面句子和之前用thus 构成因果关系。C.后面句子说动物不食用植物。选择A选项是因为these 指代的是scattered dead fish and other water-dwelling creatures,这部分内容也和待插入句子中的during which many aquatic animals must have died.相呼应。
    \end{nlz}
\end{blk}
