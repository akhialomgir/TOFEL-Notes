\section{TPO 39 Passage 03}

\subsection{Words}

\begin{tabular}{lll}
    resembling   & adj. & 相似               \\
    intervention & n.   & 干预;干涉            \\
    intensity    & n.   & 强烈;剧烈;强度         \\
    extent       & n.   & 面积,范围;长度;数量      \\
    trunk        & n.   & 树干               \\
    flammable    & adj. & 易燃的              \\
    suppression  & n.   & 镇压;抑制;制止         \\
    extinguish   & v.   & 扑灭               \\
    stump        & n.   & 树桩;残余部分          \\
    sapling      & n.   & 树苗               \\
    choke        & v.   & 停止呼吸,(使)窒息;(使)哽噎 \\
    arson        & n.   & 纵火               \\
    understory   & n.   & 下层林木             \\
\end{tabular}

\subsection{Collocation}

\begin{tabular}{ll}
    springs up & 突然出现;涌现 \\
    cart away  & 运走,带走   \\
\end{tabular}

\newpage

\subsection{Analyse}

\begin{blk}
    \begin{qst}
        Q6. According to paragraph 3, all of the following have been used to \textbf{determine the frequency of forest fires under natural conditions} in \textbf{Montana’s ponderosa pine forests} EXCEPT
    \end{qst}

    \begin{chc}
        A. recent records of fire-suppression efforts in the region

        B. historical documents

        C. examination of tree rings on burned trees

        D. the dating of scars on remaining stumps of fire-damaged trees
    \end{chc}

    \begin{psgq}
        To take Montana’s low-altitude ponderosa pine forest as an example, \textbf{historical records}\textsuperscript{B}, plus \textbf{counts of annual tree rings}\textsuperscript{C} and \textbf{datable fire scars on tree stumps}\textsuperscript{D}, demonstrated that a ponderosa pine forest experiences a lightning-lit fire about once a decade under natural conditions (i.e., before fire suppression began around 1910 and became effective after 1945).
    \end{psgq}

    \begin{nlz}
        只有A没有提到
    \end{nlz}
\end{blk}

\begin{blk}
    \begin{qst}
        Q8. In paragraph 3, what is the author’s purpose in describing \textbf{the natural cycle of fires} in ponderosa pine forests?
    \end{qst}

    \begin{chc}
        A. To emphasize the importance of replanting seedlings after a forest fire

        C. To describe how fire affects a typical ponderosa pine forest in the absence of human intervention
    \end{chc}

    \begin{psgq}
        The mature ponderosa trees have bark two inches thick and are relatively resistant to fire, which instead burns out the understory—the lower layer—of fire-sensitive Douglas fir seedlings that have grown up since the previous fire. But after only a decade’s growth until the next fire, those young seedling plants are still too low for fire to spread from them into the crowns of the ponderosa pine trees. Hence the fire remains confined to the ground and understory. As a result, many natural ponderosa pine forests have a parklike appearance, with low fuel loads, big trees spaced apart, and a relatively clear understory.
    \end{psgq}

    \begin{nlz}
        这本段主要讲述了ponderosa树的着火的循环。每次新的火起来后,都会因为小树太矮了而烧不到大树的树干,因此火焰被控制在了地面,破坏性也被控制了。作者形容这个过程是为了向我们说明ponderosa树在着火时候的情况,当然也是在无人干预的情况下,选项C符合意思。
    \end{nlz}
\end{blk}

\newpage