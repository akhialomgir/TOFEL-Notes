\section{TPO 44 Passage 02}

\subsection{Words}

\begin{tabular}{lll}
    projected     & adj. & 计划的,预计的,推断的;投影的   \\
    faithful      & adj. & 忠诚的;忠实的;如实的;虔诚的   \\
    preliminary   & adj. & 初步的,起始的;预备的       \\
    complicated   & adj. & 复杂的;麻烦的;费解的       \\
    foreshorten   & v.   & 减少;缩短;用透视法缩小      \\
    recede        & v.   & 逐渐远离;变得模糊,逐渐淡漠    \\
    chandelier    & n.   & 枝形吊灯;(尤指旧时用的)枝形烛台 \\
    bride         & n.   & 新娘;即将(或刚)结婚的女子    \\
    garment       & n.   & (一件)衣服            \\
    speculate     & v.   & 猜测;推测,推断          \\
    precursor     & n.   & 前身                \\
    investigation & n.   & 调查                \\
    intricate     & adj. & 复杂的               \\
    stature       & n.   & 声望;声誉             \\
\end{tabular}

\subsection{Collocation}


\subsection{Analyse}

\begin{blk}
    \begin{qst}
        Q9. According to paragraph 3, Hockney believes that all of the following indicate use of a camera obscura EXCEPT
    \end{qst}

    \begin{chc}
        A. very detailed, realistic work

        B. increased contrast between light and dark

        C. oversimplification of forms when the image is traced

        D. complicated foreshortening of objects
    \end{chc}

    \begin{psgq}
        In recent times the British artist David Hockney has published his investigations into the secret use of the camera obscura, claiming that for up to 400 years, many of Western art’s great masters probably used the device to produce \textbf{almost photographically realistic details}\textsubscript{A} in their paintings. He includes in this group Caravaggio, Hans Holbein, Leonardo da Vinci, Diego Velázquez, Jean-Auguste-Dominique Ingres, Agnolo Bronzino, and Jan van Eyck. From an artist’s point of view, Hockney observed that a camera obscura \textbf{compresses the complicated forms of a three-dimensional scene into two-dimensional shapes that can easily be traced}\textsubscript{C} and also \textbf{increases the contrast between light and dark}\textsubscript{B}, leading to the chiaroscuroartistic term for a contrast between light and dark effect seen in many of these paintings. In Jan van Eyck’s The Marriage of Giovanni Arnolfini and Giovanna Cenami, the \textbf{complicated foreshortening}\textsubscript{D} a technique for representing an image in art that makes it appear to recede in space in the chandelier and the intricate detail in the bride’s garments are among the clues that Hockney thinks point to the use of the camera obscura.
    \end{psgq}

    \begin{nlz}
        选项C中的oversimplification过于简化与原文信息相悖
    \end{nlz}
\end{blk}
