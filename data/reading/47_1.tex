\section{TPO 47 Passage 01}

\subsection{Words}

\begin{tabular}{lll}
    conquest    & n.   & 征服;制服;攻克;驾驭 \\
    troop       & n.   & 部队,军队       \\
    literally   & adv. & 确实地;名副其实地   \\
    amply       & adv. & 充分地         \\
    dismiss     & v.   & 拒绝考虑;否定     \\
    carve       & v.   & 切;雕刻        \\
    utilitarian & adj. & 功利主义的;实用主义的 \\
    utensil     & n.   & 用具;器具       \\
    worship     & v.   & 景仰;十分喜爱     \\
    diplomatic  & adj. & 外交的         \\
    pretension  & n.   & 自命,标榜;意图;抱负 \\
    exceptional & adj. & 杰出的         \\
    influx      & n.   & 突然大量涌入      \\
    supremacy   & n.   & 最高地位,至高无上   \\
    curvilinear & adj. & 曲线的,弯曲的     \\
\end{tabular}

\subsection{Collocation}

\begin{tabular}{ll}
    On occasion & 有时 \\
\end{tabular}


\subsection{Analyse}

\begin{blk}
    \begin{qst}
        Q8. In paragraph 4, why does the author mention that “Pre-Roman Britain was highly localized, with people rarely traveling beyond their own region”?
    \end{qst}

    \begin{chc}
        A. To suggest that the Roman conquest of Britain increased the standard of living for natives

        B. To indicate that pre-Roman Britain was more interested in festivals and community life than conquering other regions

        C. To explain why architecture during this period was not built to be particularly large

        D. To illustrate how the traditional roundhouse evolved under the influence of Roman civil architecture
    \end{chc}

    \begin{psgq}
        This art had a major impact on the native peoples, and one of the most important factors was a change in the scale of buildings. \textbf{Pre-Roman Britain was highly localized, with people rarely traveling beyond their own region.} \textbf{On occasion large groups amassed for war or religious festivals, but society remained centered on small communities.} Architecture of this era reflected this with even the largest of the fortified towns and hill forts containing no more than clusters of medium-sized structures. The spaces inside even the largest roundhouses were modest, and the use of rounded shapes and organic building materials gave buildings a human scale. But the effect of Roman civil architecture was significant. The sheer size of space enclosed within buildings like the basilica of London must have been astonishing. This was an architecture of dominance in which subject peoples were literally made to feel small by buildings that epitomized imperial power. Supremacy was accentuated by the unyielding straight lines of both individual buildings and planned settlements since these too provided a marked contrast with the natural curvilinear shapes dominant in the native realm.
    \end{psgq}

    \begin{nlz}
        原文On occasion large groups amassed for war or religious festivals, but society remained centered on small communities,说明当时建筑比较小的原因是人们活动范围比较小,对应选项C中:To explain why architecture … not built … large。
    \end{nlz}
\end{blk}

\begin{blk}
    \begin{qst}
        Q12. According to paragraph 4, buildings from the pre-Roman period differed sharply from buildings reflecting Roman civil architecture in each of the following respects EXCEPT
    \end{qst}

    \begin{chc}
        A. their outside and inside dimensions

        B. the impact they had on people

        C. the geometric shapes in which they were built

        D. the positioning of buildings in clusters
    \end{chc}

    \begin{psgq}
        This art had a major impact on the native peoples, and one of the most important factors was a change in the scale of buildings. Pre-Roman Britain was highly localized, with people rarely traveling beyond their own region. On occasion large groups amassed for war or religious festivals, but society remained centered on small communities. Architecture of this era reflected this with even the largest of the fortified towns and hill forts containing no more than clusters of medium-sized structures. The spaces inside even the largest roundhouses were modest, and the use of rounded shapes and organic building materials gave buildings a human scale. But the effect of Roman civil architecture was significant. \textbf{The sheer size of space enclosed within buildings like the basilica of London must have been astonishing.}\textsubscript{A} \textbf{This was an architecture of dominance in which subject peoples were literally made to feel small by buildings that epitomized imperial power.}\textsubscript{B} Supremacy was accentuated by the unyielding straight lines of both individual buildings and planned settlements since these too provided a marked contrast with the natural \textbf{curvilinear shapes}\textsubscript{C} dominant in the native realm.
    \end{psgq}

    \begin{nlz}
        原文The sheer size of space enclosed within buildings like the basilica of London must have been astonishing,说明建筑尺寸变大,对应选项A中:dimensions;原文Supremacy was accentuated by the unyielding straight lines … since … a marked contrast with … curvilinear shapes … native realm,说明几何形状不同,对应选项C中:the geometric shapes;原文… subject peoples … made to feel small by buildings … imperial power,说明建筑对人的影响不同,对应选项B中:impact。没有提到位置,所以选择D选项。
    \end{nlz}
\end{blk}