\section{TPO 4 Passage 3}

\subsection{Words}

\begin{tabular}{lll}
    crude     & adj. & 生的  \\
    hostile   & adj. & 恶劣的 \\
    dismantle & v.   & 拆除  \\
    subside   & v.   & 下沉  \\
\end{tabular}

\subsection{Analyse}

\begin{blk}
    \begin{qst}
        Q4.Which of the sentences below best expresses the essential information in the highlighted sentence in the passage? Incorrect choices change the meaning in important ways or leave out essential information.
    \end{qst}

    \begin{chc}
        A. Higher temperatures and pressures promote sedimentation, which is responsible for petroleum formation.

        B. Deposits of sediments on top of organic matter increase the temperature of and pressure on the matter.

        C. Increase pressure and heat from the weight of the sediment turn the organic remains into petroleum.

        D. The remains of microscopic organisms transform into petroleum once they are buried under mud.
    \end{chc}

    \begin{psgq}
        Continued sedimentation—the process of deposits’ settling on the sea bottom—buries the organic matter and subjects it to higher temperatures and pressures, which convert the organic matter to oil and gas .
    \end{psgq}

    \begin{nlz}
        C 去掉原句中的插入语,原句变成了 sedimentation bury and subject to blabla,convert to petro。A 错,原文没说温度压力提升 sedimentation;B 遗漏了重要信息,原句的变成石油没说;C 正确;D 完全没重现原文的重要信息,错。
    \end{nlz}
\end{blk}

\begin{blk}
    \begin{qst}
        Q8.What does the development of the Alaskan oil field mentioned in paragraph 4 demonstrate?
    \end{qst}

    \begin{chc}
        A. More oil is extracted from the sea than from land.

        B. Drilling for oil requires major financial investments.

        C. The global demand for oil has increased over the years.

        D. The North Slope of Alaska has substantial amounts of oil.
    \end{chc}

    \begin{psgq}
        As oil becomes increasingly difficult to find, the search for it is extended into more-hostile environments. The development of the oil field on the North Slope of Alaska and the construction of the Alaska pipeline are examples of the great expense and difficulty involved in new oil discoveries. Offshore drilling platforms extend the search for oil to the ocean’s continental shelves—those gently sloping submarine regions at the edges of the continents. More than one-quarter of the world’s oil and almost one-fifth of the world’s natural gas come from offshore, even though offshore drilling is six to seven times more expensive than drilling on land. A significant part of this oil and gas comes from under the North Sea between Great Britain and Norway.
    \end{psgq}

    \begin{nlz}
        B 第二句和第三句说到了阿拉斯加的石油开采是一个例子,great expense and difficulty involved in new oil discoveries,说明使用开采花钱又需要技术,正确答案 B,需要大量投资;其他选项都没说。
    \end{nlz}
\end{blk}


\begin{blk}
    \begin{qst}
        Q12.In paragraph 6, the author’s primary purpose is to
    \end{qst}

    \begin{chc}
        A. provide examples of how oil exploration can endanger the environment

        B. describe accidents that have occurred when oil activities were in progress

        C. give an analysis of the effects of oil spills on the environment

        D. explain how technology and legislation help reduce oil spills
    \end{chc}

    \begin{psgq}
        Moreover, getting petroleum out of the ground and from under the sea and to the consumer can create environmental problems anywhere along the line. Pipelines carrying oil can be broken by faults or landslides, causing serious oil spills. Spillage from huge oil-carrying cargo ships, called tankers, involved in collisions or accidental groundings (such as the one off Alaska in 1989) can create oil slicks at sea. Offshore platforms may also lose oil, creating oil slicks that drift ashore and foul the beaches, harming the environment. Sometimes, the ground at an oil field may subside as oil is removed. The Wilmington field near Long Beach, California, has subsided nine meters in 50 years; protective barriers have had to be built to prevent seawater from flooding the area. Finally, the refining and burning of petroleum and its products can cause air pollution. Advancing technology and strict laws, however, are helping control some of these adverse environmental effects.
    \end{psgq}

    \begin{nlz}
        A,问整个第六段,看第一句,说整个从石油开采一直到市场上的任何一环都有可能污染环境,所以 A 是正确答案;只是说有可能污染环境,没说事故,B 不对;原文只是给出事实,没有任何分析,C 错;D 错因为没说减少污染。
    \end{nlz}
\end{blk}

\begin{blk}
    \begin{qst}
        Q14
    \end{qst}

    \begin{chc}
        A.Petroleum formation is the result of biological as well as chemical activity.

        D.Petroleum tends to rise to the surface, since it is lower in density than water.
    \end{chc}

    \begin{nlz}
        A 选项为 Petroleum formation is the result of biological as well as chemical activity.意为石油形成是生物和化学活动的结果。对应文章第一句:Petroleum, consisting of crude oil and natural gas, seems to originate from organic matter in marine sediment.意思是石油是由原油和天然气组成,似乎都源自于海洋的有机物沉淀。原油和天然气是化学活动,海洋有机物沉淀是生物活动。B 选项为对应原文 examples of the great expense and difficulty involved in new oil discoveries. Offshore drilling platforms extend the search for oil to the ocean’s continental shelves.C 选项对应原文:getting petroleum out of the ground and from under the sea and to the consumer can create environmental problems anywhere along the line.D 选项正确,但为细节 minor idea,不选。E 选项错误,不是 half 而是30%~40%.F 选项原文未提及。
    \end{nlz}
\end{blk}

