\section{TPO 50 Passage 01}

\subsection{Words}

\begin{tabular}{lll}
    bustling    & adj. & 熙熙攘攘的;热闹的             \\
    flow        & n.   & (尤指液体、气体或电)流动         \\
    spearhead   & v.   & 领导(攻击、行动等);充当…的先锋;带头做 \\
    lumber      & v.   & 缓慢笨拙地移动               \\
    estate      & n.   & (位于乡村的)大片私有土地,庄园,种植园  \\
    speculation & n.   & 猜测;推测;推断投机,投机买卖       \\
    onset       & n.   & (指不愉快的事情)的开始,发作       \\
    scrap       & v.   & 放弃;取消                 \\
    ambitious   & adj. & 有抱负的;志向远大的;雄心勃勃的      \\
    strain      & n.   & 压力;拉力,张力;作用力          \\
\end{tabular}

\subsection{Collocation}

\begin{tabular}{ll}
    collocation & 中文 \\
\end{tabular}

\subsection{Analyse}

\begin{blk}
    \begin{qst}
        Q9. Which of the sentences below best expresses the essential information in the highlighted sentence in the passage? Incorrect choices change the meaning in important ways or leave out essential information.
    \end{qst}

    \begin{chc}
        A. Private investment in railroads \textbf{began in the 1850s} following the dramatic expansion of the railroad network, which had been financed by local governments.

        B. \textbf{Railroads’ relations with local governments} became strained in the 1850s, when railroads turned to private investors for financing to expand their capacity.

        C. Local governments’ limited capacity to finance railroad expansion was a \textbf{long-standing} problem that railroads solved in the 1850s by turning toward private investment.

        D. When local governments could not adequately finance the railroads’ dramatic expansion in the 1850s, private investment became increasingly important.
    \end{chc}

    \begin{psgq}
        The dramatic expansion of the railroad network in the 1850s, \textbf{however}, strained the financing capacity of local governments and required a turn toward private investment(, which had never been absent from the picture).
    \end{psgq}

    \begin{nlz}
        本题为句子简化题。做句子简化题时要把握逻辑和语义两个方面。首先我们来看句子,句子的意思是“然而,1850年铁路网络的急剧扩张,给当地政府的融资能力造成了不小的压力,铁路资金的需求转向了私人投资方向,而私人投资一直以来都是铁路资金来源的一部分。” A选项说“Private investment in railroads began in the 1850s following……”,认为私人投资是从19世纪50年代开始的,但是原句中说“……which had never been absent from the picture.”说明私人投资一直存在,所以A选项在语义上与原文矛盾,故排除。 B选项说“Railroads' relations with local governments became strained in the 1850s”,但是原文中并没有说铁路公司和政府的关系开始变得紧张,文中说的是“The dramatic expansion of the railroad network in the 1850s, however, strained the financing capacity of local governments”,意思是1850年铁路网络的急剧扩张,给当地政府的融资能力造成了不小的压力,所以B选项的这部分内容在文中并未提及,B选项可以排除。 C选项说政府融资的局限性是一个长久以来的问题,在50年代通过转向私人投资的方式得到解决。但文中说是因为经济危机,才使政府融资困难,C选项中的“long-standing problem”与原文矛盾;此外,原文并没有说私人投资就解决了融资问题,故C选项排除。 D选项说对于50年代的铁路扩张,地方政府的融资并不能完全满足铁路建设需求,因此私人投资就显得更加重要。符合原文,逻辑正确,主干完整,故为正确答案。
    \end{nlz}
\end{blk}

\begin{blk}
    \begin{qst}
        Q10. Paragraph 5 supports which of the following ideas about \textbf{people who held railroad stock}?
    \end{qst}

    \begin{chc}
        A. Many of them were not particularly wealthy.

        B. Many of them overestimated the economic benefits of railroads.

        C. Most of them bought their stock for less than it was worth.

        D. Most of them had been employed by a railroad.
    \end{chc}

    \begin{psgq}
        本题为细节题。根据题干中的关键词“stock”,定位到原文第5段的第2句话“Well aware of the economic benefits of railroads, individuals living near them had long purchased railroad stock issued by governments and had directly bought stock in railroads, often paying by contributing their labor to building the railroads.”持有铁路股份的人需要通过付出劳动力才能换取股份,可知选项A:他们不是十分富有,而是需要出卖劳动的劳工阶层。是正确的。

        选项B:他们高估了铁路公司的经济效益,文中未提及;选项C:他们以比实际价格更低的价钱买股票,文中并没有明确提及。

        选项D说:大部分购买铁路股票的人都在铁路公司工作,并不能从原文“often paying by contributing their labor to building the railroads.”直接得出,他们可能是仅仅临时为铁路工作,不是正式员工。所以这道题正确答案为A。
    \end{psgq}

    \begin{nlz}
        XXX
    \end{nlz}
\end{blk}

\begin{blk}
    \begin{qst}
        Q14. Summary
    \end{qst}

    \begin{chc}
        A. Increased rail line between the East and the Midwest resulted in the rapid rise of major Midwestern cities such as Chicago, as well as in the growth of small towns along railroad routes.

        B. Once Chicago became a major commercial hub with direct rail connections to New Orleans and the East, Midwestern farmers were no longer limited to \textbf{selling most of their products locally}.

        \textbf{C.} Real estate speculation by railroads in the 1850s drove up the value of farmland and \textbf{encouraged many Midwestern farmers to sell their land and make a new life in the cities}.

        D. State government financing of railroads largely ended in the 1830s and was replaced by a combination of local and federal government support and money from private investors.

        E. Both canals and railroads fell out of public favor in the early 1840s, but by the \textbf{mid-1850s} \textbf{the economic benefits} of railroads had once again become generally recognized.

        \textbf{F.} In the 1850s railroads turned to investment banks in New York City for capital to expand and by doing so, helped establish the city as the main financial center in the United States.
    \end{chc}

    \begin{nlz}
        本题为概要小结题。我们逐一来看选项,排除错误选项。

        A选项对应文章第一段和第三段。第一段中描述了铁路对于芝加哥发展的积极影响,使芝加哥发展为一个有着10万人口的大城市;第三段主要描述铁路推动了沿线小城镇的发展。所以A选项是对一、三段的概率总结,故正确。

        B选项说因为铁路的建设,芝加哥成为了主要的商业腹地,中西部地区的农民不再局限于在当地出售农作物。B选项对应文章第一段,但文中并没有提到农民在当地出售农作物,而是说“芝加哥北部和西部的农民不再需要用船装载他们的谷物、牲畜和乳制品,沿着密西西比河一路向下,运送到新奥尔良;他们现在可以直接将他们的产品运送到东部。”所以B选项排除。

        C选项中说铁路公司对房地产的投资,使农场的价值上升,很多农民卖掉他们的土地,以求在城市开始新生活。但是文中并没有提到农民卖掉土地,搬到城市生活,故C选项在文中未提及,排除。

        D选项对应文章第四、第五段。第四段说因为经济危机,州政府不再为铁路建设提供资金,财政负担转移到了当地政府和联邦政府身上;第五段强调了私人投资对铁路建设的重要性。故D选项是对文章第四、第五段的概括总结,故D选项正确。

        E选项说在40年代,运河建设和铁路建设都失去了人们的支持,但是在50年代中期,由于铁路的经济效益,铁路建设再次被人们重视。根据关键词“1850s”,对应文中最后一段,文中着重强调的是50年代的铁路扩张对当地政府融资造成压力,只能寻求私人投资,并没有说是因为经济效益,铁路才被人们再次重视。故E选项排除。

        F选项对应文章第五段,说50年代铁路公司转向纽约的投资银行寻求资金,使纽约成为了美国经济中心。
        F选项与第五段内容完全符合,是第五段的总结概括,故F选项正确。
    \end{nlz}
\end{blk}
