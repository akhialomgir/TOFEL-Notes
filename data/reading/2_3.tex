\section{TPO 2 Passage 3}

\subsection{Words}

\begin{tabular}{lll}
    spectacle    & n.   & 景象;大场面              \\
    manipulation & n.   & 操纵(不公平或不诚信的方式)      \\
    advent       & n.   & (事件、发明或人物的)出现,来临,到来 \\
    minuscule    & adj. & 极小的                 \\
\end{tabular}

\subsection{Analyse}

\begin{blk}
    \begin{qst}
        Q3.Which of the sentences below best expresses the essential information in the highlighted sentence from the passage? Incorrect answer choices change the meaning in important ways or leave out essential information.
    \end{qst}

    \begin{chc}
        A. Edison was more interested in developing a variety of machines than in developing a technology based on only one.

        B. Edison refused to work on projection technology because he did not think exhibitors would replace their projectors with newer machines.

        C. Edison did not want to develop projection technology because it limited the number of machines he could sell.

        D. Edison would not develop projection technology unless exhibitors agreed to purchase more than one projector from him.
    \end{chc}

    \begin{psgq}
        He refused to develop projection technology, reasoning that if he made and sold projectors, then exhibitors would purchase only one machine-a projector-from him instead of several.
    \end{psgq}

    \begin{nlz}
        该句指出:爱迪生之所以不愿改进放映技术是因为那样放映者便不会购买大量机器,从而使他收益下降。这句话中的主要逻辑就是因果。C 选项完美的重现原句的因故逻辑,且因果内容完全正确,所以选 C。A 选项的 variety 和原句完全无关,且没有重现因果逻辑;B 选项的结果正确,原因和原句说反;D 的 unless 逻辑不对,而且 exibitors 开始的内容原句没有提及。
    \end{nlz}
\end{blk}

\begin{blk}
    \begin{qst}
        Q14
    \end{qst}

    \begin{chc}
        A.Kinetoscope parlors for viewing films were modeled on phonograph parlors.

        B.Thomas Edison's design of the Kinetoscope inspired the development of large screen projection.

        C.Early cinema allowed individuals to use special machines to view films privately.

        D.Slide-and-lantern shows had been presented to audiences of hundreds of spectators.

        E.The development of projection technology made it possible to project images on a large screen.

        F.Once film images could be projected, the cinema became form of mass consumption.
    \end{chc}

    \begin{nlz}
        正确答案是 CEF。C.E.F 在文中都有提及,A 错的原因在于是细节并不是文章的主旨,D 错的原因也是因为它是 minor idea,B 未提及。
    \end{nlz}
\end{blk}
