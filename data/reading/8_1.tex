\section{TPO 8 Passage 1}

\subsection{Words}

\begin{tabular}{lll}
    complex   & n. & 综合大楼;建筑群    \\
    implicate & v. & 牵连,涉及       \\
    rival     & n. & 竞争对手;敌手     \\
    coerce    & v. & 强制,强迫;威逼,胁迫 \\
\end{tabular}

\subsection{Analyse}

\begin{blk}
    \begin{qst}
        Q9.Which of the following allowed Teotihuacán to have “a competitive edge over its neighbors”?
    \end{qst}

    \begin{chc}
        A. A well-exploited and readily available commodity

        B. The presence of a highly stable elite class

        C. Knowledge derived directly from the Olmecs about the art of toolmaking

        D. Scarce natural resources in nearby areas such as those located in what are now the Guatemalan and Mexican highlands
    \end{chc}

    \begin{psgq}
        It seems likely that Teotihuacán’s natural resources, along with the city elite’s ability to recognize their potential, gave the city a competitive edge over its neighbors . The valley, like many other places in Mexican and Guatemalan highlands, was rich in obsidian. The hard volcanic stone was a resource that had been in great demand for many years, at least since the rise of the Olmecs (a people who flourished between 1200 and 400 B.C.), and it apparently had a secure market. Moreover, recent research on obsidian tools found at Olmec sites has shown that some of the obsidian obtained by the Olmecs originated near Teotihuacán. Teotihuacán obsidian must have been recognized as a valuable commodity for many centuries before the great city arose.
    \end{psgq}

    \begin{nlz}
        A 关键词已经划出,所在句说 natural resourc 给了这个地方 edge,然后就用大量笔墨说 obsidian 黑曜石是这里一种很主要的资源,所以答案是存在 commodity,A 正确;注意 B 项颇具干扰性,使城市有优势的不是 elite,是他们对于这种潜力的认识;C 完全没提到;D 说到了资源,但又说资源是在邻近的地方,也错。

        不能直接按照词来判断,需要判断选项是否准确
    \end{nlz}
\end{blk}

\begin{blk}
    \begin{qst}
        Q12.In paragraph 6, the author discusses “The thriving obsidian operation” in order to
    \end{qst}

    \begin{chc}
        A. explain why manufacturing was the main industry of Teotihuacán

        B. give an example of an industry that took very little time to develop in Teotihuacán

        C. Illustrate how several factors influenced each other to make Teotihuacán a powerful and wealthy city

        D. explain how a successful industry can be a source of wealth and a source of conflict at the same time
    \end{chc}

    \begin{psgq}
        The picture of Teotihuacán that emerges is a classic picture of positive feedback among obsidian mining and working, trade, population growth, irrigation, and religious tourism. The thriving obsidian operation , for example, would necessitate more miners, additional manufacturers of obsidian tools, and additional traders to carry the goods to new markets. All this led to increased wealth, which in turn would attract more immigrants to Teotihuacán. The growing power of the elite, who controlled the economy, would give them the means to physically coerce people to move to Teotihuacán and serve as additions to the labor force. More irrigation works would have to be built to feed the growing population, and this resulted in more power and wealth for the elite.
    \end{psgq}

    \begin{nlz}
        功能目的题,往前看,前一句就是本段中心,说 T 的兴起是黑曜石开采和其他一系列因素互动的结果,所以 C 正确。A 和 B 彻底没说,D 中的 conflict 冲突没说,也不对。

        D选项错在conlict没提及,需要看清选项,不要急于选相近的
    \end{nlz}
\end{blk}

\begin{blk}
    \begin{qst}
        Q14
    \end{qst}

    \begin{chc}
        A.The number and sophistication of the architectural, administrative, commercial, and religious features of Teotihuacan indicate the existence of centralized planning and control.

        B.Teotihuacán may have developed its own specific local religion as a result of the cultural advances made possible by the city’s great prosperity.

        C.As a result of its large number of religious shrines, by the first century A.D., Teotihuacan become the most influential religious center in all of Mesoamerica.

        D.Several factors may account for Teotihuacán’s extraordinary development, including its location, rich natural resources, irrigation potential, intelligent elite, and the misfortune of rival communities.

        E.In many important areas, from the obsidian industry to religious tourism, Teotihuacán’s success and prosperity typified the classic positive feedback cycle.

        F.Although many immigrants settled in Teotihuacán between A.D.150 and 700, the increasing threat of coerced labor discouraged further settlement and limited Teotihuacán’s population growth.
    \end{chc}

    \begin{nlz}
        A.D.E the number 选项对应第一段倒数第二句,正确。Teothihuacan 选项原文没说,不选。as a result 选项错,第五段提到了 shirine,但只说它能吸引外来人口,没说 T 因为这个变得很重要,不选。several 选项对应第二段第一句,正确。in many 选项对应第六段第一句,正确。 although 选项原文没说,不选。

        不要默认A为细节,需要确认两个选项是否正确,需要节约更多时间
    \end{nlz}
\end{blk}