\section{TPO 9 Passage 1}

\subsection{Words}

\begin{tabular}{lll}
    crumble & v. & (使)粉碎;(使)成碎屑 \\
\end{tabular}

\subsection{Analyse}

\begin{blk}
    \begin{qst}
        Q1.According to paragraph 1, the theory that people first migrated to the Americans by way of an ice-free corridor was seriously called into question by
    \end{qst}

    \begin{chc}
        A. paleoecologist Glen MacDonald's argument that the original migration occurred much later than had previously been believed

        B. the demonstration that certain previously accepted radiocarbon dates were incorrect

        C. evidence that the continental ice began its final retreat much later than had previously been believed

        D. research showing that the ice-free corridor was not as long lasting as had been widely assumed
    \end{chc}

    \begin{psgq}
        It has long been accepted that the Americas were colonized by a migration of peoples from Asia, slowly traveling across a land bridge called Beringia (now the Bering Strait between northeastern Asia and Alaska) during the last Ice Age. The first water craft theory about the migration was that around 11,000-12,000 years ago there was an ice-free corridor stretching from eastern Beringia to the areas of North America south of the great northern glaciers. It was the midcontinental corridor between two massive ice sheets-the Laurentide to the west-that enabled the southward migration. But belief in this ice-free corridor began to crumble when paleoecologist Glen MacDonald demonstrated that some of the most important radiocarbon dates used to support the existence of an ice-free corridor were incorrect. He persuasively argued that such an ice-free corridor did not exist until much later, when the continental ice began its final retreat.
    \end{psgq}

    \begin{nlz}
        B 以 ice-free corridor 做关键词定位至第二句和第四句,第二句只是单纯说这个理论是什么,问题问 called into question 遭到质疑为什么,所以答案应该在第四句。说这个理论 crumble 是因为 GM 先生发现 radiocarbon dates 是错的,所以答案 B。
    \end{nlz}
\end{blk}

\begin{blk}
    \begin{qst}
        Q3.Paragraph 2 begins by presenting a theory and \textbf{then} goes on to
    \end{qst}

    \begin{chc}
        A. discuss why the theory was rapidly accepted but then rejected

        B. present the evidence on which the theory was based

        C. cite evidence that now shows that the theory is incorrect

        D. explain why the theory was not initially considered plausible
    \end{chc}

    \begin{psgq}
        Support is growing for the alternative theory that people using watercraft, possibly skin boats, moved southward from Beringia along the Gulf of Alaska and then southward along the Northwest coast of North America possibly as early as 16,000 years ago. This route would have enabled humans to enter southern areas of the Americas prior to the melting of the continental glaciers. Until the early 1970s,most archaeologists did not consider the coast a possible migration route into the Americas because geologists originally believed that during the last Ice Age the entire Northwest Coast was covered by glacial ice. It had been assumed that the ice extended westward from the Alaskan/Canadian mountains to the very edge of the continental shelf, the flat, submerged part of the continent that extends into the ocean. This would have created a barrier of ice extending from the Alaska Peninsula, through the Gulf of Alaska and southward along the Northwest Coast of north America to what is today the state of Washington.
    \end{psgq}

    \begin{nlz}
        D, 问题问先提出理论接着怎么了,所以看第二句。说这个路线能让人们在冰川融化之前进入美洲南部,所以这个是支持本段第一句内容的,而且后一句说直到70年代,人们才认同 coast 路线,也就是之前人们不认同这个理论,而后认为这个理论可能是对的。A 和 C 都不对;原文没说前面的理论是基于这个事实的,B 错,D 正确。
    \end{nlz}
\end{blk}

\begin{blk}
    \begin{qst}
        Q14
    \end{qst}
\end{blk}
