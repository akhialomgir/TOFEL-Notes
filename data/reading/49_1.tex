\section{TPO 49 Passage 01}

\subsection{Words}

\begin{tabular}{lll}
    irreversibly & adv. & 不可改变地,不可逆转地         \\
    relevance    & n.   & 相关性;实用性;意义          \\
    archaeology  & n.   & 考古学                 \\
    sheer        & adj. & 完全的,彻底的             \\
    tumble       & v.   & 跌倒;滚下,坠落;倒塌         \\
    gnaw         & v.   & 咬,啮,啃(通常指啃出孔洞或逐渐啃坏) \\
    cliff        & n.   & 悬崖                  \\
    silt         & n.   & 淤泥                  \\
    insular      & adj. & 思想狭隘的;保守的           \\
    isostatic    & adj. & 均衡                  \\
    pebble       & n.   & 鹅卵石                 \\
    dune         & n.   & 沙丘                  \\
    topography   & n.   & 地貌                  \\
\end{tabular}

\subsection{Analyse}

\begin{blk}
    \begin{qst}
        Q5. By indicating that flora and fauna of isolated or insular areas were often \textbf{“irreversibly affected”} by the changes due to the Ice Age, the author means that the flora and fauna were
    \end{qst}

    \begin{chc}
        A. unable to return to their previous conditions

        B. in a constant state of change

        C. completely destroyed by human colonization

        D. unevenly distributed across the area
    \end{chc}

    \begin{psgq}
        Nevertheless, for archaeologists concerned with the long periods of time of the Paleolithic period there are variations in coastlines of much greater magnitude to consider. The expansion and contraction of the continental glaciers caused huge and uneven rises and falls in sea levels worldwide. When the ice sheets grew, the sea level would drop as water became locked up in the glaciers; when the ice melted, the sea level would rise again. Falls in sea level often exposed a number of important land bridges, such as those linking Alaska to northeast Asia and Britain to northwest Europe, a phenomenon with far-reaching effects not only on human colonization of the globe but also on the environment as a whole—the flora and fauna of isolated or insular areas were radically and often \textbf{irreversibly affected}. Between Alaska and Asia today lies the Bering Strait, which is so shallow that a fall in sea level of only four meters would turn it into a land bridge. When the ice sheets were at their greatest extent some 18,000 years ago (the glacial maximum), it is thought that the fall was about 120 meters, which therefore created not merely a bridge but a vast plain, 1,000 kilometers from the north to the south, which has been called Beringia. The existence of Beringia (and the extent to which it could have supported human life) is one of the crucial pieces of evidence in the continuing debate about the likely route and date of human colonization of the New World.
    \end{psgq}

    \begin{nlz}
        本题可以从短语本身着手。irreversibly affected意为“不可逆转地受到了影响“,植物和动物受到了影响即发生了改变,不可逆转即无法回到改变之前的状态,故答案为a。
    \end{nlz}
\end{blk}
