\section{TPO 41 Passage 01}

\subsection{Words}

\begin{tabular}{lll}
    enlist    & v.   & 招募;使入伍                \\
    delicate  & adj. & 微妙的;精美的;易碎的           \\
    scattered & adj. & 分散的;散布的;疏疏落落的         \\
    pulverize & v.   & 把(某物)压(或磨)成粉          \\
    pollen    & n.   & 花粉                    \\
    sift      & v.   & 筛(面粉、糖等)              \\
    chant     & n.v. & 反复说(或吟唱)的词语;反复吟唱;反复念诵 \\
    examine   & v.   & (仔细地)检查,审查,调查         \\
    dynamism  & n.   & 活力,精力                 \\
    rendition & n.   & (对歌曲,音乐或诗歌的)诠释        \\
    recount   & v.   & 讲述;叙述;描述              \\
\end{tabular}

\subsection{Collocation}


\subsection{Analyse}

\begin{blk}
    \begin{qst}
        Q5. It can be inferred from the discussion of illness and curing in paragraph 2 that
    \end{qst}

    \begin{chc}
        B. rituals involving songs and sand paintings may be used to treat an illness

        D. after a serious illness, a Navajo will take part in a ceremony
    \end{chc}

    \begin{psgq}
        The purpose and meaning of the sand paintings can be explained by examining one of the most basic ideals of Navajo society, embodied in their word hozho (beauty or harmony, goodness, and happiness). It coexists with hochxo (“ugliness,” or “evil,” and “disorder”) in a world where opposing forces of dynamism and stability create constant change. When the world, which was created in beauty, becomes ugly and disorderly, \textbf{the Navajo gather to perform rituals with songs and make sand paintings to restore beauty and harmony to the world}. Some illness is itself regarded as a type of disharmony. Thus, the restoration of harmony through a ceremony can be part of a curing process.
    \end{psgq}

    \begin{nlz}
        纳瓦霍人聚集在一起表演歌曲并制作沙画是为了使世界恢复美丽以及和平,一些疾病本身就被看作是一种不和谐。B选项为同义替换,其中Some illness is itself regarded as a type of disharmony对应B选项中的treat an illness。所以B选项正确。D选项说纳瓦霍人在大病后将参加仪式,是否本人参加没有提到,所以错误。
    \end{nlz}
\end{blk}
