\section{TPO 4 Passage 2}

\subsection{Words}

\begin{tabular}{lll}
    consistent & adj. & 符合 \\
\end{tabular}

\subsection{Analyse}

\begin{blk}
    \begin{qst}
        Q10.According to paragraph 4, which of the following may best represent the attitude of hunters toward deer and reindeer in the Upper Paleolithic period?
    \end{qst}

    \begin{chc}
        A. Hunters did not fear deer and reindeers as much as they did large game animals such as horses and mammoths.

        B. Hunters were not interested in hunting deer and reindeer because of their size and speed.

        C. Hunters preferred the meat and hides of deer and reindeer to those of other animals.

        D. Hunters avoided deer and reindeer because of their natural weapons, such as horns.
    \end{chc}

    \begin{psgq}
        The particular symbolic significance of the cave paintings in southwestern France is more explicitly revealed, perhaps, by the results of a study conducted by researchers Patricia Rice and Ann Paterson. The data they present suggest that the animals portrayed in the cave paintings were mostly the ones that the painters preferred for meat and for materials such as hides. For example, wild cattle (bovines) and horses are portrayed more often than we would expect by chance, probably because they were larger and heavier (meatier) than other animals in the environment. In addition, the paintings mostly portray animals that the painters may have feared the most because of their size, speed, natural weapons such as tusks and horns, and the unpredictability of their behavior. That is, mammoths, bovines, and horses are portrayed more often than deer and reindeer. Thus, the paintings are consistent with the idea that the art is related to the importance of hunting in the economy of Upper Paleolithic people. Consistent with this idea, according to the investigators, is the fact that the art of the cultural period that followed the Upper Paleolithic also seems to reflect how people got their food. But in that period, when getting food no longer depended on hunting large game animals (because they were becoming extinct), the art ceased to focus on portrayals of animals.
    \end{psgq}

    \begin{nlz}
        A 以 deer 和 reindeer 做关键词定位至原文第五句,接上题,说大的动物比诸如 deer 和 reindeer 这类小动物更多出现在岩画上,前文说因为怕那些大动物的很多方面才画,所以 A 对,B 将两个概念杂合到一起,而且原文没说猎人对 deer 不感兴趣,错;原文没说 deer and reindeer 的肉和兽皮更受青睐,不存在比较级,原文说的是马和牛,所以 C 项错;原文说有 horn 的是大型动物,不是 deer,D 错
    \end{nlz}
\end{blk}

\begin{blk}
    \begin{qst}
        Q11.According to paragraph 4, what change is evident in the art of the period following the Upper Paleolithic?
    \end{qst}

    \begin{chc}
        A. This new art starts to depict small animals rather than large ones.

        B. This new art ceases to reflect the ways in which people obtained their food.

        C. This new art no longer consists mostly of representations of animals.

        D. This new art begins to show the importance of hunting to the economy.
    \end{chc}

    \begin{psgq}
        The particular symbolic significance of the cave paintings in southwestern France is more explicitly revealed, perhaps, by the results of a study conducted by researchers Patricia Rice and Ann Paterson. The data they present suggest that the animals portrayed in the cave paintings were mostly the ones that the painters preferred for meat and for materials such as hides. For example, wild cattle (bovines) and horses are portrayed more often than we would expect by chance, probably because they were larger and heavier (meatier) than other animals in the environment. In addition, the paintings mostly portray animals that the painters may have feared the most because of their size, speed, natural weapons such as tusks and horns, and the unpredictability of their behavior. That is, mammoths, bovines, and horses are portrayed more often than deer and reindeer. Thus, the paintings are consistent with the idea that the art is related to the importance of hunting in the economy of Upper Paleolithic people. Consistent with this idea, according to the investigators, is the fact that the art of the cultural period that followed the Upper Paleolithic also seems to reflect how people got their food. But in that period, when getting food no longer depended on hunting large game animals (because they were becoming extinct), the art ceased to focus on portrayals of animals.
    \end{psgq}

    \begin{nlz}
        以 following the upper Paleolithic 做关键词定位至最后两句:But in that period, when getting food no longer depended on hunting large game animals (because they were becoming extinct), the art ceased to focus on portrayals of animals. 这句话说:当获取食物不再依赖于捕猎大型动物,这个绘画艺术就不在专注于展现动物,C 正确;A 错,原文并没说不画大动物之后,就转向小型动物;还是反映生活的,B 错;D 从来没说。
    \end{nlz}
\end{blk}

\begin{blk}
    \begin{qst}
        Q14
    \end{qst}

    \begin{chc}
        A.Researchers have proposed several different explanations for the fact that animals were the most common subjects in the cave paintings.

        B.The art of the cultural period that followed the Upper Paleolithic ceased to portray large game animals and focused instead on the kinds of animals that people of that period preferred to hunt.

        C.Some researchers believe that the paintings found in France provide more explicit evidence of their symbolic significance than those found in Spain, southern Africa, and Australia.

        E.Some researchers have argued that the cave paintings mostly portrayed large animals that provided Upper Paleolithic people with meat and materials.
    \end{chc}

    \begin{nlz}
        文章标题暗示有几种展开方向:艺术的特点(现象描述)/艺术的发展(历史叙述)/为什么画这些艺术(现象解释)

        首段说旧石器时代艺术水平高,年代久,算是背景介绍

        二段进入主题,说艺术发现的3个位置。

        三段说一个特点,主要画动物,给出解释,各有褒贬。

        四段说两个人的观点,给出正面支持。

        五段从主题引申,扯到其他艺术。

        引出句概括首段内容,正确答案应概括以后各段大意。

        A(researchers have proposed)选项对应整个第三段,提出了三个主要画动物的解释,正确。

        B(the art)选项文章没有说,不选。

        C(some researchers believe)选项文章没有说,不选。

        D(the cave)选项原文没说,不选。

        E(some researchers have)选项对应原文第四段,正确。

        F(besides)选项对应第五段,正确.
    \end{nlz}
\end{blk}
