\section{TPO 52 Passage 03}

\subsection{Words}

\begin{tabular}{lll}
    cereal        & n.   & 谷类植物,谷物                 \\
    sophisticated & adj. & 富有经验的                   \\
    fringe        & n.   & (地区或群体的)边缘,外围;(活动的)次要部分 \\
    resprout      & v.   & 重生                      \\
    primitive     & adj. & 原始的,早期的,远古的             \\
    graft         & v.   & 移植(皮肤、骨等);嫁接            \\
    inhabit       & v.   & 居住于                     \\
    livestock     & n.   & 牲畜,家畜                   \\
    pasture       & n.   & 牧场                      \\
    constraint    & n.   & 限制;束缚;约束                \\
\end{tabular}

\subsection{Analyse}

\begin{blk}
    \begin{qst}
        Q14. Summary
    \end{qst}

    \begin{chc}
        A. Food production started with the cultivation of root plants and developed to include the cultivation of cereal crops.

        \textbf{B}. Pastoralists who moved south across the Sahara to find suitable land for cattle grazing may have also cultivated some crops for food.

        C. In order to avoid human and animal sleeping sickness, which posed a danger to herders and cattle, more and more herders took up cultivation.

        \textbf{D}. Hunter-gatherer groups in eastern and southern Africa raided their herding neighbors to acquire cattle and other domesticated animals.

        E. By 1500 B.C. cereal agriculture was widespread throughout the savanna belt south of the Sahara, and shifting agriculture was used effectively and widely by farmers.

        F. Slash-and-burn agriculture was initially rejected by farmers because it was too labor-intensive, but once the technique was improved, it expanded gradually to eastern and southern Africa.
    \end{chc}

    \begin{nlz}
        本题为概要小结题。我们逐一来看选项,找出正确答案。

        A选项:食物的生产是从根茎植物的种植开始的,然后渐渐地谷类作物也开始被种植。正确。A选项是全文的概括总结,第一段提到一开始人们种植块根植物和树本作物,而后人们用轮耕法来种植谷类,故A选项正确。

        B选项:穿过撒哈拉地区向南寻找合适的牧场的牧民可能也会种植农作物。正确,对应第四段第1句“Contrary to popular belief: there is no such phenomenon as 'pure' pastoralists, a society that subsists on its herds alone.”说明牧民不仅放牧,同时也种植农作物。第四段第2句还说,牧民会种植高粱、小米和其他热带降雨作物。故B选项正确。

        C选项:为了防止人类和牲畜患上昏睡症,危及到牧民和牛群,越来越多的牧民开始种植农作物。错误,第四段第一句话说“Contrary to popular belief: there is no such phenomenon as 'pure' pastoralists, a society that subsists on its herds alone.”说明牧民不仅放牧,同时也种植农作物。文中并没有说因为昏睡症,牧民就不放牧转而去种植农作物了。畜牧和耕种两者是同步进行的。故C选项与原文矛盾,排除。

        \textbf{D选项:东非和南非以狩猎和采集为生的人,靠掠夺邻居的牲畜,来获得牛以及其他的家养动物。对应第三段最后一句,但是这只是细节信息,并不是文章的主干内容,所以不选。}

        E选项:到公元前1500年,谷类农业已经在撒哈拉以南地区的稀树草原带传播开了,轮耕法能高效种植,被农民广泛采用。正确,对应文章第四、第五段的内容。四、五段都在论述轮耕法的好处和传播情况,故E选项正确。

        F选项:“刀耕火种”法最早被农民们排斥,因为它需要大量的劳动力,但是一旦技术提高了之后,它逐渐扩展到了东非和南非。错误,因为第四第五段说,轮耕法,即“刀耕火种”一开始就被农民们所采用。所以F选项与原文矛盾,且F选项中的“技术提高”并没有在原文中提到,故F选项排除。
    \end{nlz}
\end{blk}
