\section{TPO 54 Passage 2}

\subsection{Words}

\begin{tabular}{lll}
    distinctive   & adj. & 与众不同的,独特的,特别的         \\
    mammoth       & n.   & 猛犸(象),毛象(现已绝种)        \\
    coincidence   & n.   & 同时发生;(尤指令人吃惊的)巧合,碰巧的事 \\
    champion      & v.   & 支持,声援;捍卫,为…斗争         \\
    Significantly & adv. & 意味深长地;显著地;相当数量地       \\
    minuscule     & adj. & 微小的,极小的               \\
    scenario      & n.   & 可能发生的事态;设想            \\
    slaughter     & n.   & (尤指战争中的)屠杀,杀戮         \\
    succumb       & v.   & 屈从,屈服;放弃抵抗;承认失败       \\
    paleontology  & n.   & 古生物学                  \\
    sparse        & adj. & 稀少的                   \\
\end{tabular}

\subsection{Analyse}

\begin{blk}
    \begin{qst}
        Q5.
        According to paragraph 2, what suggests that human activity played a role in the extinction of mammals about 11,000 years ago?
    \end{qst}

    \begin{chc}
        A.
        Climate changes that would have favored human population expansion occurred at the time of the extinctions.

        B.
        The presence of human hunters had caused animal extinctions in other time periods.

        C.
        There was a pattern of climate change earlier than 11,000 years ago that had not caused animal extinctions.

        D.
        Harmful climate changes 11,000 years ago would have required humans to hunt larger numbers of animals for food.
    \end{chc}

    \begin{psgq}
        Although the climate changed at the end of the Pleistocene, warming trends had happened before. A period of massive extinction of large mammals like that seen about 11,000 years ago had not occurred during the previous 400,000 years, despite these changes. The only apparently significant difference in the Americas 11,000 years ago was the presence of human hunters of these large mammals. Was this coincidence or cause-and-effect?
    \end{psgq}

    \begin{nlz}
        根据题目所问,是什么暗示了人类活动才是导致动物灭绝的原因。定位到句子“Although the climate changed at the end of the Pleistocene, warming trends had happened before. A period of massive extinction of large mammals like that seen about 11,000 years ago had not occurred during the previous 400,000 years, despite these changes.”意思是“尽管更新世末期气候发生了变化,但变暖趋势发生在此之前。 尽管发生了这些变化,但在过去的40万年中,大约11,000年前,大型哺乳动物大规模灭绝的时期并没有出现。” A选项说气候变化为适合人类人口增长导致了当时的灭绝,文章并没有提及气候变化得适合人口增长,无法推断。 B选项说人类猎手的存在,在其他时期导致了动物灭绝,错误,文章就没有提及其他时期。 C选项说在早于11000年前,气候变化的模式并没有导致动物灭绝,因而表示动物灭绝另有原因,而之后又说了灭绝是出现在人类猎手出现之时的,能推断出人类活动才是导致动物灭绝。 D选项说有危害的气候变化在11000年前将使得人们去狩猎更大数量的动物作为食物,错误,无法从原文推断出气候促使人们去捕食大数量动物。
    \end{nlz}
\end{blk}

\begin{blk}
    \begin{qst}
        Q7.
        Which of the following best describes the results of the research discussed in paragraph 4?
    \end{qst}

    \begin{chc}
        A.
        Scientists used mathematical models to show that most of the extinctions occurred in areas where humans had recently arrived.

        B.
        Scientists established that the main population of North Americans who hunted lived in Canada during the time of the megafauna extinctions.

        C.
        Scientists used numerical models to confirm that a small population of humans could have caused big-game extinctions in a relatively short period of time.

        D.
        Scientists used statistics to prove beyond doubt the currently accepted view that human hunters were the main cause of the megafauna extinctions.
    \end{chc}

    \begin{psgq}
        The researchers ran the model several ways, always beginning with a population of 100 humans in Edmonton, in Alberta, Canada, at 11,500 years ago.Assuming different initial North American big-game-animal populations (75-150 million animals) and different population growth rates for the human settlers (0.65\%-3.5\%), and varying kill rates, Mosimann and Martin derived figures of between 279 and 1,157 years from initial contact to big-game extinction.
    \end{psgq}

    \begin{nlz}
        题目问,以下哪一项是第四段中讨论的研究结果的最佳描述?来看选项:

        A:科学家们使用数学模型来表明大多数的灭绝发生在人类最近到达的地区,错误,原文没有提到是人类“最近到达”,所以排除。

        B:科学家们确定,在巨型动物灭绝期间,狩猎的大部分北美人居住在加拿大,错误,原文只是说研究者拿加拿大的100人作为数据分析,并非推断历史事实。

        C:科学家们使用数值模型来证实一小群人可能会在相对较短的时间内造成大猎物灭绝,选项中的“big-game”还有大猎物的意思,而这句话是最贴近选段的意义的,故为正确答案。

        D:科学家利用统计数据来证明人类捕猎者是巨型动物灭绝的主要原因,这是毫无疑问的,理解错误,文中提到了关键的时间因素此处并没有说到,而这个模型也并不是用来证明人类是导致巨型动物灭绝的主要原因,故排除。
    \end{nlz}
\end{blk}

\begin{blk}
    \begin{qst}
        Q8.
        Which of the following statements about Larry Agenbroad's work is implied in the discussion in paragraph 5?
    \end{qst}

    \begin{chc}
        A.
        Agenbroad showed that Mosimann and Martin's estimates of the amount of time needed to drive big-game to extinction were correct.

        B.
        Agenbroad's maps were the first to indicate the ages of the Clovis sites.

        C.
        Agenbroad reinforced the idea that humans could have caused the extinctions.

        D.
        Agenbroad's studies of wooly mammoths led to his discovery of Clovis sites.
    \end{chc}

    \begin{psgq}
        Many scholars continue to support this scenario.For example, geologist Larry Agenbroad has mapped the locations of dated Clovis sites alongside the distribution of dated sites where the remains of wooly mammoths have been found in both archaeological and purely paleontological contexts.These distributions show remarkable synchronicity (occurrence at the same time).
    \end{psgq}

    \begin{nlz}
        由Larry Agenbroad’s work可以找到geologist Larry Agenbroad has mapped the locations of dated Clovis sites alongside the distribution of dated sites where the remains of wooly mammoths have been found in both archaeological and purely paleontological contexts. These distributions show remarkable synchronicity (occurrence at the same time). 翻译为:地质学家Larry Agenbroad已经绘制了那个时期的克洛维斯遗址的位置,以及考古遗址的分布,在这些地方已经发现了羊毛猛犸遗体,这些都可以在考古学和纯粹生物学的文章中体现出来了。这些分布显示出明显的同步性(同时发生)。段落第一句所提到的“许多学者继续支持这种说法”此处指代的的应是前几段提到的生态学家保罗•马丁(Paul S. Martin)提出将更新世末期大型哺乳动物灭绝与人类捕食相关联的模式。 A选项为:Agenbroad表明,莫西曼和马丁对驱使大型动物消亡所需时间的估计是正确的。错误,这里并非应和莫西曼和马丁的估算结果。 B选项为:Agenbroad的地图首次显示了Clovis遗址的年代。原文并未提到是首次显示 C选项:Agenbroad强化了人类可能导致其灭绝的想法,正确,通过发现人文遗迹中的大型动物化石来说明人类可能是导致其灭绝的原因。 D选项:Agenbroad对猛犸兽的研究导致了克洛维斯遗址的发现,错误,前后颠倒了,是绘制了遗址再发现的猛犸遗体。
    \end{nlz}
\end{blk}

\begin{blk}
    \begin{qst}
        Q11.
        In paragraph7, why does the author mention that there is abundant archaeological evidence for the extinction of the New Zealand moa?
    \end{qst}

    \begin{chc}
        A.
        To show that extinctions occurred in areas other than North America.

        B.
        To challenge Martin's claim that the lack of megafauna remains supports his model of the megafauna extinctions.

        C.
        To identify a country where humans were highly skilled as hunters.

        D.
        To help explain why it is unclear whether all large herbivores of late Pleistocene America became extinct after the appearance of Clovis.
    \end{chc}

    \begin{psgq}
        Though Martin claims the lack of evidence actually supports his model—the evidence is sparse because the spread of humans and the extinction of animals occurred so quickly—this argument seems weak. And how could we ever disprove it?As archaeologist Donald Grayson points out, in other cases where extinction resulted from the quick spread of human hunters—for example, the extinction of the moa, the large flightless bird of New Zealand—archaeological evidence in the form of remains is abundant. Grayson has also shown that the evidence is not so clear that all or even most of the large herbivores in late Pleistocene America became extinct after the appearance of Clovis. Of the 35 extinct genera, only 8 can be confidently assigned an extinction date of between 12,000 and 10,000 years ago.Many of the older genera, Grayson argues, may have succumbed before 12,000 B.C., at least half a century before the Clovis showed up in the American West.
    \end{psgq}

    \begin{nlz}
        定位第七段“Martin claims the lack of evidence actually supports his model”而后面的这些证据都是来disprove这个观点的,题目问为何作者提了这些考古学证据。 A选项:来表现在美国北部以外的地区发生了灭绝。 B选项:反驳了马丁所声称的巨型动物遗骸的缺失支持了他巨型动物灭绝模式的说法 C选项:来识别一个人类非常擅长打猎的国家 D选项:为了解释为什么不清楚晚更新世美国的大型草食动物在Clovis出现后灭绝了。 上述选项从除了B点出了是反驳马丁观点,别的都未提及,所以都是错的。
    \end{nlz}
\end{blk}

\begin{blk}
    \begin{qst}
        Q12.
        Paragraph 7 suggests that Donald Grayson believes which of the following about the remains at Clovis sites and megafaunal extinctions?
    \end{qst}

    \begin{chc}
        A.
        The rapid rate of the spread of humans explains why the extinctions also occurred at a rapid rate.

        B.
        The lack of evidence of human-caused extinctions is not surprising in view of the speed with which the extinctions occurred.

        C.
        It is likely that more evidence will be found as dating methods improve.

        D.
        If humans did contribute to the extinctions, much more evidence of that would have been found by now.
    \end{chc}

    \begin{psgq}
        Though Martin claims the lack of evidence actually supports his model—the evidence is sparse because the spread of humans and the extinction of animals occurred so quickly—this argument seems weak. And how could we ever disprove it?As archaeologist Donald Grayson points out, in other cases where extinction resulted from the quick spread of human hunters—for example, the extinction of the moa, the large flightless bird of New Zealand—archaeological evidence in the form of remains is abundant. Grayson has also shown that the evidence is not so clear that all or even most of the large herbivores in late Pleistocene America became extinct after the appearance of Clovis. Of the 35 extinct genera, only 8 can be confidently assigned an extinction date of between 12,000 and 10,000 years ago.Many of the older genera, Grayson argues, may have succumbed before 12,000 B.C., at least half a century before the Clovis showed up in the American West.
    \end{psgq}

    \begin{nlz}
        题目问第7段表明Donald Grayson相信以下哪一个关于Clovis遗址的遗骸和巨型动物灭绝?定位到“As archaeologist Donald Grayson points out, in other cases where extinction resulted from the quick spread of human hunters—for example, the extinction of the moa, the large flightless bird of New Zealand—archaeological evidence in the form of remains is abundant. Grayson has also shown that the evidence is not so clear that all or even most of the large herbivores in late Pleistocene America became extinct after the appearance of Clovis.”前面说其他的证据都由考古形式遗骸得到证明,而对于遗址和巨型动物灭绝的证据却很模糊。 A选项:人类的快速传播解释了为什么灭绝也以很快的速度发生。原文并未提及两者关系。 B选项:鉴于灭绝发生的速度,缺乏人类导致灭绝的迹象并不令人意外。原文没有关于Donald对此做出“并不令人意外”的表示。 C选项:随着测定方法的改进,可能会发现更多证据。原文并没有关于对测定方法的论述,错误。 D选项:人类如果确实对它们造成了灭绝,现在已经发现了更多的证据。这句话就是定位的句子的隐藏意义,故选D。
    \end{nlz}
\end{blk}

\begin{blk}
    \begin{qst}
        Q14. Summary
    \end{qst}

    \begin{chc}
        A. That the first humans migrated to North America near the same time as the extinctions of the megafauna has led many to believe that hunting by humans was a significant cause of those extinctions.

        B. Support for the hypothesis that hunting by humans caused the extinctions has been provided by computer models, as well as by the discovery of some mammoths' remains near human settlements.

        C. There is more evidence that human settlers hunted large flightless birds like the moa into extinction than there is that hunters caused the extinction of large mammals like the mammoth.

        D. Early North Americans known as the Clovis society developed spears in order to hunt enough large animals to feed their population as it expanded across vast areas of the continent.

        E. Scientists have proven that the human hunters of large animals who migrated across North America grew in number so quickly that they killed off most of the megafauna within a few hundred years.

        F. Some scholars argue that the evidence linking mammoth remains to human settlements is insufficient to establish that hunting by humans was a significant factor in the megafauna extinctions.
    \end{chc}

    \begin{nlz}
        本文为总结题。我们依次分析选项找出正确答案: 所给提示意思为:大约11000年前,所有北美的大型哺乳动物都灭绝了。

        A选项对应文章第二段,A period of massive extinction of large mammals like that seen about 11,000 years ago had not occurred during the previous 400,000 years, despite these changes. The only apparently significant difference in the Americas 11,000 years ago was the presence of human hunters of these large mammals. 说明了正是人类开始活动之后出现了大型哺乳动物的灭绝,故A正确。

        B选项对应四、五段,分别对应的是Mosimann和Martin的研究和Larry Agenbroad这两个例子,故B正确。

        C选项说比起人类让大型哺乳动物比如猛犸的证据,人类猎杀鸟类导致灭绝的例子更多。逻辑正确,但并不是文章主要内容,因而不选。

        D说被称为克洛维斯社会的早期北美人开发了长矛,以便捕捉足够大的动物来喂养他们的人口,由于克洛维斯在非洲大陆的广大地区扩张。文中没有提到开发长矛是为了捕捉足够大的动物。

        E选项说科学家已经证明,在北美洲上迁移的大型动物的人类猎人数量增长如此之快,以至于在几百年内灭绝了大部分的巨型动物群。错误,科学家并没有足够证据说明人类是导致这些巨型动物群灭绝的原因。

        F选项说一些学者认为,将猛犸象留在人类住区的证据不足以证明人类狩猎是巨型动物灭绝的重要因素。这是对的,对应了文章的最后两段。
    \end{nlz}
\end{blk}
