\section{TPO 41 Passage 03}

\subsection{Words}

\begin{tabular}{lll}
    outgas        & v.   & (通过加热的方法)除去气体(或水)     \\
    emanate       & v.   & 表现;显示                 \\
    predominantly & adv. & 占主导地位地;占绝大多数地;显着地     \\
    bound         & adj. & 肯定的;极有可能的;必然的;注定了的    \\
    oppressive    & adj. & 令人焦躁的;令人压抑的;闷热的,令人窒息的 \\
    humid         & adj. & 潮湿的                   \\
    envelop       & v.   & 覆盖;包住;围绕;笼罩           \\
    affair        & n.   & 事务;事情;秘密;活动           \\
    evaporate     & v.   & (尤指通过加热)(使)挥发,蒸发      \\
    ultraviolet   & adj. & 紫外线的                  \\
\end{tabular}

\subsection{Collocation}


\subsection{Analyse}

\begin{blk}
    \begin{qst}
        10. According to paragraph 6, extremely high temperatures increased the amount of carbon dioxide in Venus’ atmosphere by
    \end{qst}

    \begin{chc}
        B. baking out carbon dioxide from carbonate rocks

        D. replacing the previous mechanisms for removing carbon dioxide with less effective ones
    \end{chc}

    \begin{psgq}
        Once Venus’ oceans disappeared, so did the mechanism for removing carbon dioxide from the atmosphere. With no oceans to dissolve it, outgassed carbon dioxide began to accumulate in the atmosphere, intensifying the greenhouse effect even more. \textbf{Temperatures eventually became high enough to “bake out” any carbon dioxide that was trapped in carbonate rocks}\textsubscript{B}. This liberated carbon dioxide formed the thick atmosphere of present-day Venus. Over time, the rising temperatures would have leveled off, solar ultraviolet radiation having broken down atmospheric water vapor molecules into hydrogen and oxygen. With all the water vapor gone, the greenhouse effect would no longer have accelerated.
    \end{psgq}

    \begin{nlz}
        从该句可知,选项B为同义替换
    \end{nlz}
\end{blk}
