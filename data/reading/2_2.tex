\section{TPO 2 Passage 2}

\subsection{Analyse}

\begin{blk}
    \begin{qst}
        Q4. Pakicetus and modern cetaceans have similar
    \end{qst}

    \begin{chc}
        A. hearing structures

        B. adaptations for diving

        C. skull shapes

        D. breeding locations
    \end{chc}

    \begin{psgq}
        The fossil consists of a complete skull of an archaeocyte, an extinct group of ancestors of modern cetaceans. Although limited to a skull, the Pakicetus fossil provides precious details on the origins of cetaceans. The skull is cetacean-like but its jawbones lack the enlarged space that is filled with fat or oil and used for receiving underwater sound in modern whales. Pakicetus probably detected sound through the ear opening as in land mammals. The skull also lacks a blowhole, another cetacean adaptation for diving. Other features, however, show experts that Pakicetus is a transitional form between a group of extinct flesh-eating mammals, the mesonychids, and cetaceans. It has been suggested that Pakicetus fed on fish in shallow water and was not yet adapted for life in the open ocean. It probably bred and gave birth on land.
    \end{psgq}

    \begin{nlz}
        整段大部分内容都在描述两者的差异,只有第三句话描述到了相似之处: The skull is cetacean-like but its jawbones lack the enlarged space that is filled with fat or oil and used for receiving underwater sound in modern whales.这一句说 P 的头盖骨是 cetacen-like,这就对应了 C 选项。所以正确答案是 C。
    \end{nlz}
\end{blk}

\begin{blk}
    \begin{qst}
        Q8.It can be inferred that Basilosaurus bred and gave birth in which of the following locations
    \end{qst}

    \begin{chc}
        A. On land

        B. Both on land and at sea

        C. In shallow water

        D. In a marine environment
    \end{chc}

    \begin{psgq}
        Another major discovery was made in Egypt in 1989. Several skeletons of another early whale, Basilosaurus, were found in sediments left by the Tethys Sea and now exposed in the Sahara desert. This whale lived around 40 million years ago, 12 million years after Pakicetus. Many incomplete skeletons were found but they included, for the first time in an archaeocyte, a complete hind leg that features a foot with three tiny toes. Such legs would have been far too small to have supported the 50-foot-long Basilosaurus on land. Basilosaurus was undoubtedly a fully marine whale with possibly nonfunctional, or vestigial, hind legs.
    \end{psgq}

    \begin{nlz}
        根据 Basilosaurus 定位在4段,or the first time in an archaeocyte, a complete hind leg that features a foot with three tiny toes. Such legs would have been far too small to have supported the 50-foot-long Basilosaurus on land. D 选项是同义改写,正确。ABC 明显错误。
    \end{nlz}
\end{blk}
