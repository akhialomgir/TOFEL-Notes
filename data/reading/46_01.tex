\section{TPO 46 Passage 01}

\subsection{Words}

\begin{tabular}{lll}
    boast    & v.   & 自吹自擂,吹嘘,夸耀 \\
    reed     & n.   & 芦苇(杆)      \\
    clumsy   & adj. & 笨拙的,不灵活的   \\
    bulky    & adj. & 庞大而占地方的    \\
    leather  & n.   & 皮革         \\
    conquer  & v.   & 征服         \\
    wedge    & n.   & 楔子,三角木;楔形物 \\
    hallmark & n.   & 特点         \\
\end{tabular}

\subsection{Collocation}

\subsection{Analyse}

\begin{blk}
    \begin{qst}
        Q13. insert

        \textbf{However}, the Sumerian language did not entirely disappear.
    \end{qst}

    \begin{psgq}
        The Akkadians conquered the Sumerians around the middle of the third millennium B.C.E., and they took over the various cuneiform signs used for writing Sumerian and gave them sound and word values that fit their own language. $\blacksquare$ The Babylonians and Assyrians did the same, and so did peoples in Syria and Asia Minor. $\blacksquare$ The literature of the Sumerians was treasured throughout the Near East, and long after Sumerian ceased to be spoken, the Babylonians and Assyrians and others kept it alive as a literary language, the way Europeans kept Latin alive after the fall of Rome. $\blacksquare$ For the scribes of these non-Sumerian languages, training was doubly demanding since they had to know the values of the various cuneiform signs for Sumerian as well as for their own language. $\blacksquare$
    \end{psgq}

    \begin{nlz}
        插入的句子中however说明其后面的句子内容与上文需要构成转折关系。not entirely disappear说明上文提到了Sumerian language 部分消失或者后文提到了Sumerian language没有disappear的事实。插入口B后面的句子说明了Sumerian literature仍然被treasure,也就是Sumerian language没有disappear,故应插入B处。
    \end{nlz}
\end{blk}


\begin{blk}
    \begin{qst}
        Q14. summary
    \end{qst}

    \begin{chc}
        A. Writing was invented in the same areas in which civilization began by the ancient civilizations of Mesopotamia, Asia Minor and the Mediterranean.

        B. Writing was developed first by the Sumerians using wedge-shaped marks (cuneiform) on clay tablets and then by the Egyptians using papyrus paper.

        C. The development of cuneiform is known because it was written on a long-lasting material and because it was long and widely used throughout the ancient Near East.

        D. Scribes using cuneiform in Assyria, Babylon, Syria and Asia Minor had to learn all the languages that used the cuneiform script.

        E. Cuneiform tablets generally dealt with business and factual matters, but other topics, including literature, were also recorded and valued.

        F. Batches of clay tablets, sometimes with as many as a thousand tablets each, are often found by archaeologists.
    \end{chc}

    \begin{nlz}
        文章结构分析:

        文章标题 The Origins of Writing 预示文章有可能为时间顺序结构,或者因果结构(文字发明的原因条件及后果影响)

        首段,引入主题,文字是古文明的重要标志,苏美尔人有clay tablets的文字纪录;

        二段,前半段通过对比的方式,说明泥土作为书写材料的优势:durable. 该优势使得苏美尔文字有考古证据而埃及没有。后半段说泥土作为书写材料广泛使用的原因:便宜,容易书写;

        三段,泥土对书写方式的影响:cunei楔形文字的出现,楔形文字对社会的影响:难学,少数人掌握;

        四段,后续文明继续使用楔形文字,楔形文字对后续文明的影响;

        五段,早期文字的内容:从简单账本纪录到日常生活纪录;

        六段,大批的泥土字母被发现。文字的内容涉及生活各个方面,甚至高级智力方面。

        选项分析:

        A, Writing was invented 选项虽指向首段主旨,但提及的“Asia Minor and the Mediterranean.”,错误;

        B, Writing was developed 选项对应与原文第二段内容;

        C, The development of cuneiform 选项是对第三四段的内容,正确;

        D, Scribes using cuneiform 选项:had to learn all the languages说的太多绝对,文中第四段只是说“需要了解楔形文字的含义和标志”,错误;

        E, Cuneiform tablets 选项是对五、六段的主旨概括,正确;

        F, Batches of clay tablets 选项属于细节信息,不选。
    \end{nlz}
\end{blk}