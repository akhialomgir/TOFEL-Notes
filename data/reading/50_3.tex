\section{TPO 50 Passage 03}

\subsection{Words}

\begin{tabular}{lll}
    proton & n. & 质子       \\
    bulge  & n. & 凸起;鼓起    \\
    reside & v. & 居住;定居    \\
    ripple & n. & 涟漪;微波;细浪 \\
\end{tabular}

\subsection{Analyse}

\begin{blk}
    \begin{qst}
        Q2. According to paragraph 1, the energy that comes from stars and that is seen as light is the result of
    \end{qst}

    \begin{chc}
        A. protons combining with helium atoms

        B. atoms of heavier elements smashing together

        C. various particles fusing with one another

        D. hydrogen atoms breaking apart
    \end{chc}

    \begin{psgq}
        Until the early- to mid-twentieth century, scientists believed that stars generate energy by shrinking. As stars contracted, it was thought, they would get hotter and hotter, giving off light in the process. This could not be the primary way that stars shine, however. If it were, they would scarcely last a million years, rather than the billions of years in age that we know they are. We now know that stars are fueled by nuclear fusion. Each time fusion takes place, energy is released as a by-product. \textbf{This energy, expelled into space, is what we see as starlight.} The fusion process begins when two hydrogen nuclei smash together to form a particle called the deuteron (a combination of a positive proton and a neutral neutron). Deuterons readily combine with additional protons to form helium. Helium, in turn, can fuse together to form heavier elements, such as carbon. In a typical star, merger after merger takes place until significant quantities of heavy elements are built up.
    \end{psgq}

    \begin{nlz}
        本题为细节题,题目问恒星释放的能量,即我们所看到的星光,是由什么造成的?根据题干中的关键词“energy”和“light”我们可以定位到第一段第7句话“This energy, expelled into space, is what we see as starlight.”那么这句话中的“this energy”具体指的是什么呢?我们再往前看第6句“Each time fusion takes place, energy is released as a by-product.”从这句话中我们知道,这种能量是核聚变的副产物。因此,本题的答案就应该是核聚变。虽然答案中并没有直接出现“nuclear fusion”核聚变,但是C选项“各种微粒互相融合”其实就是对核聚变的一种解释。文中“The fusion process begins when two hydrogen nuclei smash together ……can fuse together to form heavier elements, such as carbon.”这段话都是在描述核聚变的过程,即2个氢原子碰撞→氘核,氘核+其他质子→氦,氦融合→重元素,e.g.碳。总之,核聚变的过程就是微粒不断互相碰撞融合的过程。故C选项正确。 A选项:质子与氦原子结合。错误,因为文中只说质子和中子结合成氘核,而氘核又与其他质子结合形成氦,并没有说质子与氦原子结合,故排除。 B选项:重元素的原子互相碰撞。错误,因为文中只说2个氢原子碰撞能形成氘核,并没有提到重元素原子的碰撞,故排除。 D选项:氢原子分裂。错误,因为整段话没有提到氢原子分裂,故排除。
    \end{nlz}
\end{blk}

\begin{blk}
    \begin{qst}
        Q14. Summary
    \end{qst}

    \begin{chc}
        D. Population I stars, including the Sun, are relatively young stars that are mostly hydrogen and helium gas but also contain heavier elements.

        E. The Sun and stars like it will separate into inner cores and outer envelopes before all nuclear reactions in the cores stop and the stars finally die.
    \end{chc}

    \begin{nlz}
        本题为概要小结题,我们逐一来看选项,找出正确答案。 A选项说太阳是星族I恒星的一个例子,因为它通过核聚变产生能量,而不是通过收缩产生能量。首先,后半句话是错误的,对应文章第三段,区分星族I和星族II的关键,一是恒星年龄,而是恒星位置,和是否通过核聚变产生能量无关。其次,这句话只是一个举例,属于细枝末节信息,不是文章概要,故排除。 B选项描述星族II的恒星,对应文章第三段1~3句,符合原文内容,故为正确答案。C选项说银河系中,星族I的恒星处于中心凸起的位置或周围,而星族II的恒星处于扁平的星盘上。这句话与第二段的最后一句话直接矛盾,故C排除。D选项描述星族I的恒星,对应文章第二、三段的内容,符合原文,故为正确选项。 E选项说太阳和与它相似的恒星,会先分裂成内部和外部,然后内核的核聚变会停止,最终恒星会死亡。对应文章第四、第五段。但是在第四段末提到“Then, at some point in the far future, all nuclear reactions in the Sun’s center will cease.”然后第五段才开始描述太阳分为内部和外部的情况,所以说是核聚变停止在先,然后太阳才开始分为2个部分。E选项时间顺序颠倒,故排除。   F选项对应文章第五段,描述太阳变成红巨星之后的过程,外部会先释放完所有物质,然后内核变成白矮星,在释放完内核最后的能量之后,太阳会死亡。F选项符合原文,故为正确答案。
    \end{nlz}
\end{blk}
