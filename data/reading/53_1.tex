\section{TPO 53 Passage 01}

\subsection{Words}

\begin{tabular}{lll}
    debt      & n.   & 借款,欠款;债务                  \\
    token     & n.   & 标志;表示;象征                  \\
    conical   & adj. & 圆锥形的;圆锥的                  \\
    redundant & adj. & (尤指词、短语等)多余的,不需要的,累赘的,啰唆的 \\
    vessel    & n.   & 船,舰                       \\
    flock     & n.   & 羊群;鸟群;人群                  \\
    speculate & v.   & 猜测;推测,推断                  \\
\end{tabular}

\subsection{Analyse}

\begin{blk}
    \begin{qst}
        Q10.
        According to paragraph 5, all of the following statements about the development of numerals are true EXCEPT:
    \end{qst}

    \begin{chc}
        A.
        Numerals first developed around 3100 B.C.E.

        B.
        Numerals were created to keep records of commodities.

        C.
        The numeral “18” developed from the sign for grain.

        D.
        Accountants introduced unique numeral signs for use with signs for commodities.
    \end{chc}

    \begin{psgq}
        And not only of literacy, but numeracy (the representation of quantitative concepts) as well. The evidence of the tokens provides further confirmation that mathematics originated in people’s desire to keep records of flocks and other goods. Another immensely significant step occurred around 3100 B.C. E., when Sumerian accountants extended the token-based signs to include the first real numerals. Previously, units of grain had been represented by direct one-to-one correspondence—by repeating the token or symbol for a unit of grain the required number of times. The accountants, however, devised numeral signs distinct from commodity signs, so that eighteen units of grain could be indicated by preceding a single grain symbol with a symbol denoting “18.” Their invention of abstract numerals and abstract counting was one of the most revolutionary advances in the history of mathematics.
    \end{psgq}

    \begin{nlz}
        本题定位到原文:整个第五段。 此处原文的大意是:第五段主要讲述了数字的发展。 题干问的是哪一个选项错误地描述了数字发展的情况。

        选项A的意思是数字在公元前3100被发明,对应原文中的Another immensely significant step occurred around 3100 B.C.E.,正确;

        选项B的意思是数字是为了记录商品而产生的,对应The evidence of the tokens provides further confirmation that mathematics originated in people’s desire to keep records of flocks and other goods.,正确;

        \textbf{选项C的意思是18这个数字来自于谷物的符号,但是原文中提到18的时候只是在举例说如果有18个单位的谷物,则标号为18+谷物符号,并没有说18这个数字起源于谷物,属于无关缠绕,错误;}

        选项D的意思是会计把数字符号和商品符号结合起来用,对应原文The accountants, however, devised numeral signs distinct from commodity signs这一句,正确。
    \end{nlz}
\end{blk}

\begin{blk}
    \begin{qst}
        Q13. Insert
    \end{qst}

    \begin{chc}
        Such a system was clearly awkward for large inventories.
    \end{chc}

    \begin{psgq}
        And not only of literacy, but numeracy (the representation of quantitative concepts) as well. The evidence of the tokens provides further confirmation that mathematics originated in people’s desire to keep records of flocks and other goods. Another immensely significant step occurred around 3100 B.C. E., when Sumerian accountants extended the token-based signs to include the first real numerals.$\blacksquare$ Previously, units of grain had been represented by direct one-to-one correspondence—by repeating the token or symbol for a unit of grain the required number of times.$\blacksquare$ The accountants, however, devised numeral signs distinct from commodity signs, so that eighteen units of grain could be indicated by preceding a single grain symbol with a symbol denoting “18.”$\blacksquare$ Their invention of abstract numerals and abstract counting was one of the most revolutionary advances in the history of mathematics.$\blacksquare$
    \end{psgq}

    \begin{nlz}
        本题定位到原文:整个第五段。 此处原文的大意是:此段主要讲述了数字的发明发展历程。 题干问的是“这样的一套系统对于大的存货清单来说是很笨拙的。”该插入哪个位置。 此题做题的关键是需要插入句子里的such,such意味着前面那一句应该是已经提到了一遍这种计数系统。只有选项B的前面的一句话里提到了一种一一对应、靠重复来计数的系统,选项B合适。其他三个可插入点都没有可以对应such的内容。
    \end{nlz}
\end{blk}

\begin{blk}
    \begin{qst}
        Q14. Summary
    \end{qst}

    \begin{chc}
        \textbf{A. Three-dimensional tokens used to keep track of debts of grain and livestock eventually gave way to two-dimensional symbols on clay tablets.}

        B. Writing was probably developed by farmers and artisans, since the symbols were first used to keep track of agricultural products and items produced by artisans.

        \textbf{C. Two separate sets of symbols were used to keep track of each accounting of goods to avoid mistakes in the accounting when only one set of tokens or marks was used.}

        D. Early tokens representing three-dimensional geometric shapes show that ancient Mesopotamians invented geometry as well as writing.

        E. The inventors of numerals were probably lower-ranking accountants because they were assigned the job of developing new methods of accounting for large inventories.

        F. Symbols were first used in a direct one-to-one correspondence with the commodity being counted, but eventually true numerals were developed.
    \end{chc}

    \begin{nlz}
        选项A正确概括了第二段的主要内容,正确;

        选项B正确概括了第四段的主要内容,正确; 选项C的意思是为了避免使用一套系统导致的错误,计数的时候要使用两套不同的系统,不符合文意,错误;

        选项D的意思是早期的几何符号意味着苏美尔人不仅发明了文字,还发明了几何学,无中生有,不选;

        选项E说数字的发明者可能是低阶级的会计,因为他们被分配了发明统计大量库存的新方法的任务,不符合文意,同时也是一处细节信息。对应的信息在文章最后一句,大意是:数字的发明者是低阶级的会计(抄写员)更合乎情理,因为这可以减少他们的苦工。

        选项F正确概括了第五段的主要内容,正确。
    \end{nlz}
\end{blk}