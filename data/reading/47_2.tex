\section{TPO 47 Passage 02}

\subsection{Words}

\begin{tabular}{lll}
    ingenuity  & n.   & 心灵手巧;足智多谋;独创力             \\
    desiccated & adj. & 干的;干燥的;脱水的;干巴巴的;无趣的;毫无新意的 \\
    tropical   & adj. & 热带的                       \\
    insulate   & v.   & 使隔热;使隔音;使绝缘               \\
    ventilated & adj. & 使用人工呼吸机辅助换气的              \\
    cellar     & n.   & 地窖                        \\
    attic      & n.   & 阁楼                        \\
\end{tabular}

\subsection{Analyse}

\begin{blk}
    \begin{qst}
        Q12. Paragraph 5 supports which of the following about the air that flows through the interior of a Macrotermes natalensis mound?
    \end{qst}

    \begin{chc}
        A. It has a higher concentration of oxygen in the cellar than in the attic.

        C. It contains over 250 quarts of oxygen which circulate continuously.
    \end{chc}

    \begin{psgq}
        But how is this well-insulated nest ventilated? Its many occupants require over 250 quarts of oxygen (more than 1,200 quarts of air) per day. How can so much oxygen diffuse through the thick walls of the mound? Even the pores in the wall are filled with water, which almost stops the diffusion of gases. The answer lies in the construction of the nest. The interior consists of a large central core in which the fungus is grown, below it is a “cellar” of empty space, above it is an “attic” of empty space, and within the ridges on the outer wall of the nest, there are many small tunnels that connect the cellar and the attic. The warm air in the fungus gardens rises through the nest up to the attic. From the attic, the air passes into the tunnels in the ridges and flows back down to the cellar. Gases, mainly oxygen coming in and carbon dioxide going out, easily diffuse into or out of the ridges, since their walls are thin and their surface area is large because they protrude far out from the wall of the mound. Thus air that flows down into the cellar through the ridges is relatively rich in oxygen, and has lost much of its carbon dioxide. It supplies the nest’s inhabitants with fresh oxygen as it rises through the fungus-growing area back up to the attic.
    \end{psgq}

    \begin{nlz}
        原文From the attic, the air … flows back down to the cellar. .... Thus air that flows down into the cellar … is relatively rich in oxygen, and has lost much of its carbon dioxide,说明顶楼的空气会流回地下室,在这个过程中会失去二氧化碳得到氧气,所以地下室氧气要比顶楼的氧气多,对应选项A中:… higher concentration of oxygen … cellar than … attic.
    \end{nlz}

    \begin{qst}
        Q14. Summary
    \end{qst}

    \begin{chc}
        \textbf{A}. Although termites resemble ants in terms of size, metamorphosis, and social organization, they actually belong to a different order of insects.

        B. Termites are sensitive to dryness and to changes in temperature, so their nests are designed to minimize these factors.

        \textbf{C}. Some termites build their nests under ground, while others construct above-ground structures with thick, insulating walls.

        D. Whether they lie above ground or below ground, termite nests must include special pores that allow air to enter the nests.

        E. Some termite species grow a fungus in their nests so that it will purify the air by taking in carbon dioxide and giving off oxygen.

        F. The nests of Macrotermes natalensis consist of a series of chambers and tunnels that allow for the circulation of air and the exchange of oxygen and carbon dioxide.
    \end{chc}

    \begin{nlz}
        文章首先介绍一些巢穴在地下的白蚁以及另一些巢穴在地上的白蚁,对应选项C中:Some … nests under ground, while other … above-ground …;然后再介绍白蚁会建造湿度和温度都恒定的巢穴,对应选项B中:Termites are sensitive to dryness and to changes in temperature, so their nests are designed to minimize these factors,文章最后介绍Macrotermes natalensis建造的巢穴有许多房间和通道可以让空气循环交换氧气和二氧化碳,对应选项F中:… Macrotermes natalensis … chambers and tunnels … the circulation of air and the exchange of oxygen and carbon dioxide。白蚁的蜕变期以及社会组织和蚂蚁不同,选项A错在metarmorphosis, and social organization。菌类新陈代谢是释放二氧化碳,选项E错在taking in carbon dioxide。气孔并不能使空气进入,选项D错在special pores that allow air to enter。
    \end{nlz}
\end{blk}
