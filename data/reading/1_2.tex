\section{TPO 01 Passage 2}

\subsection{Words}

\begin{tabular}{lll}
    speculation   & n.   & 猜测          \\
    envision      & v.   & 设想;期望       \\
    attribute     & v.   & 把…分配给       \\
    ritual        & n.   & 宗教仪式        \\
    perceive      & v.   & 认为;看待;视为    \\
    sophisticated & adj. & 富有经验的       \\
    asetheic      & adj. & 审美的         \\
    elaborate     & v.   & 精心策划        \\
    antecedent    & n.   & 前事,前情,先例;祖先 \\
    penchant      & n.   & 喜爱          \\
\end{tabular}

\subsection{Collocation}

\begin{tabular}{ll}
    by no means & 决不,一点都不 \\
\end{tabular}

\subsection{Analyse}

\begin{blk}
    \begin{qst}
        Q5.The word “this” in the passage refers to
    \end{qst}

    \begin{chc}
        A. the acting out of rites

        B. the divorce of ritual performers from the rest of society

        C. the separation of myths from rites

        D. the celebration of supernatural forces
    \end{chc}

    \begin{psgq}
        But the myths that have grown up around the rites may continue as part of the group’s oral tradition and may even come to be acted out under conditions divorced from these rites. When \textbf{this} occurs, the first step has been taken toward theater as an autonomous activity, and thereafter entertainment and aesthetic values may gradually replace the former mystical and socially efficacious concerns.
    \end{psgq}

    \begin{nlz}
        就近原则往前,而且 this 不指代单个的名词,而指代整个词组或者句子,所以指代的应该是前面的 the myths may even come to be acted out under conditions divorced from these rites,指的是仪式与神话的分离,C 正确,注意 divorce 不仅仅可以表示离婚,等于 separation。
    \end{nlz}
\end{blk}

\begin{blk}
    \begin{qst}
        Q7.According to paragraph 2, what may cause societies to abandon certain rites?
    \end{qst}

    \begin{chc}
        A. Emphasizing theater as entertainment

        B. Developing a new understanding of why events occur

        C. Finding a more sophisticated way of representing mythical characters

        D. Moving from a primarily oral tradition to a more written tradition
    \end{chc}

    \begin{psgq}
        Stories (myths) may then grow up around a ritual. Frequently the myths include representatives of those supernatural forces that the rites celebrate or hope to influence. Performers may wear costumes and masks to represent the mythical characters or supernatural forces in the rituals or in accompanying celebrations. As a people becomes more sophisticated, its conceptions of supernatural forces and causal relationships may change. As a result, it may abandon or modify some rites. But the myths that have grown up around the rites may continue as part of the group’s oral tradition and may even come to be acted out under conditions divorced from these rites. When this occurs, the first step has been taken toward theater as an autonomous activity, and thereafter entertainment and aesthetic values may gradually replace the former mystical and socially efficacious concerns.
    \end{psgq}

    \begin{nlz}
        以 abandon rites 做关键词定位至全段倒数第三句,有个 as a result,说明之前的句子是导致人们放弃这种仪式的原因,也正是问题的答案。随着人们越来越智慧,他们对超自然的能力的认识,还有超自然能力和他们所期待的结果之间的因果关系会变化,也就是很多人不再认为是超自然的能力在左右他们,所以 B 有了新的认识是正确答案。

        A entertainment 概念在本段最后才提到。

        C sophisticated 概念在前句提到,原文 As a person becomes more sophisticated,是说人变得复杂了,而不是复杂的代表神秘角色的方式。

        D 原文完全没有提到。
    \end{nlz}
\end{blk}

\begin{blk}
    \begin{qst}
        Q13. Insert
    \end{qst}

    \begin{chc}
        To enhance their listeners’ enjoyment, storytellers continually make their stories more engaging and memorable..
    \end{chc}

    \begin{psgq}
        [■]Although origin in ritual has long been the most popular, it is by no means the only theory about how the theater came into being. [■]Storytelling has been proposed as one alternative. [■]Under this theory, relating and listening to stories are seen as fundamental human pleasures. [■]Thus, the recalling of an event (a hunt, battle, or other feat) is elaborated through the narrator’s pantomime and impersonation and eventually through each role being assumed by a different person.
    \end{psgq}

    \begin{nlz}
        待插入句中有 listener 和 storyteller,所以必须插在原文 storytelling 句之后,A 和 B 排除,C 的 under this theory 说明前面必须得有一个理论,Storytelling has been proposed as one alternative 刚好是个理论,所以过渡紧密,不插入任何句子,答案应该是 D,而且 pleasures 和待插入句中的 enjoyment 重复。
    \end{nlz}
\end{blk}

\begin{blk}
    \begin{qst}
        Q14. Summary
    \end{qst}

    \begin{chc}
        A.The presence of theater in almost all societies is thought to have occurred because early storytellers traveled to different groups to tell their stories.

        B.Many theorists believe that theater arises when societies act out myths to preserve social well-being.

        C.The more sophisticated societies became, the better they could influence desirable occurrences through ritualized theater.

        D.Some theories of theater development focus on how theater was used by group leaders to group leaders govern other members of society.

        E.Theater may have come from pleasure humans receive from storytelling and moving rhythmically.

        F.The human capacities for imitation and fantasy are considered possible reasons why societies develop theater.
    \end{chc}

    \begin{nlz}
        整篇文章是典型的现象解释结构,针对戏剧的为什么起源及如何起源给出几种解释。给出句 Anthropologists have developed many theories to help understand why and how theater originated.是对全文大意的概括。典型正确答案应该分别概括几种关于起源的解释。

        A 原文没说 storyteller 会到处走,所以 A 选项不对

        B 正确,对应第一段段

        C 错,原文第二段第四句说人越来越聪明之后,对超自然力量的识会变化,没说社会越复杂,好事越来越多。

        D 错,原文没说。

        E 正确,是对原文第三段第四段的概括。

        F 正确,是对原文第五段的概括。
    \end{nlz}
\end{blk}
