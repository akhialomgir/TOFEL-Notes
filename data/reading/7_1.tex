\section{TPO 7 Passage 1}

\subsection{Words}

\begin{tabular}{lll}
    peculiarity & n. & 特性             \\
    gypsum      & n. & 石膏             \\
    pebble      & n. & 鹅卵石            \\
    possess     & v. & 拥有;具有;影响,控制,支配 \\
    silt        & n. & 淤泥;残渣          \\
    formulate   & v. & 构想出;规划         \\
    faulting    & n. & 断层             \\
\end{tabular}

\subsection{Analyse}

\begin{blk}
    \begin{qst}
        Q3.What does the author imply by saying “Not a single pebble was found that might have indicated that the pebbles came from the nearby continent”?
    \end{qst}

    \begin{chc}
        A. The most obvious explanation for the origin of the pebbles was not supported by the evidence.

        B. The geologists did not find as many pebbles as they expected..

        C. The geologists were looking for a particular kind of pebble.

        D. The different pebbles could not have come from only one source.
    \end{chc}

    \begin{psgq}
        With question such as these clearly before them, the scientists aboard the Glomar Challenger processed to the Mediterranean to search for the answers. On August 23, 1970, they recovered a sample. The sample consisted of pebbles of hardened sediment that had once been soft, deep-sea mud, as well as granules of gypsum and fragments of volcanic rock. \textbf{Not a single pebble was found that might have indicated that the pebbles came from the nearby continent .} In the days following, samples of solid gypsum were repeatedly brought on deck as drilling operations penetrated the seafloor. Furthermore, the gypsum was found to possess peculiarities of composition and structure that suggested it had formed on desert flats. Sediment above and below the gypsum layer contained tiny marine fossils, indicating open-ocean conditions. As they drilled into the central and deepest part of the Mediterranean basin, the scientists took solid, shiny, crystalline salt from the core barrel. Interbedded with the salt were thin layers of what appeared to be windblown silt.
    \end{psgq}

    \begin{nlz}
        A, 先把修辞点所在的句子读清楚,说没有 pebble 能证明是形成在 nearby continent 的,注意这道题如果往前看的话是找不到答案的,即使是看完中心句也没用,所以索性往后看。后面 furthermore 那句说这些 pebble 是形成在沙漠的,也就是说之前认为的很明显的证据是不对的,A 对;剩下的选项都没说。

        D与原意相反
    \end{nlz}
\end{blk}

\begin{blk}
    \begin{qst}
        Q4.Which of the following can be inferred from paragraph 3 about the solid gypsum layer?
    \end{qst}

    \begin{chc}
        A. It did not contain any marine fossil.

        B. It had formed in open-ocean conditions.

        C. It had once been soft, deep-sea mud.

        D. It contained sediment from nearby deserts.
    \end{chc}

    \begin{psgq}
        With question such as these clearly before them, the scientists aboard the Glomar Challenger processed to the Mediterranean to search for the answers. On August 23, 1970, they recovered a sample. The sample consisted of pebbles of hardened sediment that had once been soft, deep-sea mud, as well as granules of gypsum and fragments of volcanic rock. Not a single pebble was found that might have indicated that the pebbles came from the nearby continent. In the days following, samples of solid gypsum were repeatedly brought on deck as drilling operations penetrated the seafloor. Furthermore, the gypsum was found to possess peculiarities of composition and structure that suggested it had formed on desert flats. Sediment above and below the gypsum layer contained tiny marine fossils, indicating open-ocean conditions. As they drilled into the central and deepest part of the Mediterranean basin, the scientists took solid, shiny, crystalline salt from the core barrel. Interbedded with the salt were thin layers of what appeared to be windblown silt.
    \end{psgq}

    \begin{nlz}
        A 以 gypsum layer 做关键词定位至倒数第二句,说 gypsum layer 之上和之下的 sediment 都包含 marine fossils,之间的 gypsum layer 就应该是不包含的,否则就没法分层了。请大家注意,这道题之前我讲的答案是 B,但后来经过与其他老师的讨论,还是觉得 A 正确的概率大些,而且原文倒数第三句明确说 gypsum 形成在 desert flats,所以 B 的 open-ocean 就不对;虽然说了 desert,但 nearby 这个信息是没法推得的,所以 D 不对;C 彻底没说

        不要看到就选 需要看完整句子含义
    \end{nlz}
\end{blk}
