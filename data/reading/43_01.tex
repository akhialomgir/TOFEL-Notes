\section{TPO 43 Passage 01}

\subsection{Words}

\begin{tabular}{lll}
    exceptionally & adv. & 异常地;特殊地;例外地          \\
    repercussion  & n.   & (尤指坏的)影响,反响;恶果       \\
    disruption    & n.   & 颠覆                   \\
    seesaw        & v.   & (情绪、局势等)摇摆不定,不断反复,交替 \\
    copious       & adj. & 大量的,丰富的;过量的          \\
\end{tabular}

\subsection{Collocation}


\subsection{Analyse}

\begin{blk}
    \begin{qst}
        Q5. Paragraph 3 supports which of the following statements about El Niños, \textbf{as that term is now used}?
    \end{qst}

    \begin{chc}
        B. El Niños can arise when warm currents last for two months or less.

        C. El Niños affect water temperatures \textbf{long distances} from the South American coast.
    \end{chc}

    \begin{psgq}
        While the warm-water countercurrent usually lasts for two months or less, there are occasions when the disruption to the normal flow lasts for many months. In these situations, water temperatures are raised not just along the coast, but for thousands of kilometers offshore. \textbf{Over the last few decades, the term El Niño has come to be used to describe these exceptionally strong episodes and not the annual event.} During the past 60 years, at least ten El Niños have been observed. Not only do El Niños affect the temperature of the equatorial Pacific, but the strongest of them impact \textbf{global weather}.
    \end{psgq}

    \begin{nlz}
        从题干中现在的用法定位到段落后半段,与C选项中长距离影响相对应
    \end{nlz}
\end{blk}

\begin{blk}
    \begin{qst}
        Q9. According to paragraph 5, what is the \textbf{end result} of the east-to-west pressure gradient in the eastern Pacific \textbf{during a typical year}?
    \end{qst}

    \begin{chc}
        A. The formation of a thick, warm layer of water in the western Pacific

        D. The eastward flow of warm water from the western Pacific
    \end{chc}

    \begin{psgq}
        \textbf{During a typical year}, the eastern Pacific has a higher pressure than the western Pacific does. This east-to-west pressure gradient enhances the trade winds over the equatorial waters. This results in a warm surface current that moves east to west at the equator. \textbf{The western Pacific develops a thick, warm layer of water while the eastern Pacific has the cold Humboldt Current enhanced by upwelling.}

        \textbf{However}, \textbf{in other years} the Southern Oscillation, for unknown reasons, swings in the opposite direction, dramatically changing the usual conditions described above, with pressure increasing in the western Pacific and decreasing in the eastern Pacific. This change in the pressure gradient causes the trade winds to weaken or, in some cases, to reverse. This then causes the warm water in the western Pacific to flow eastward, increasing sea-surface temperatures in the central and eastern Pacific. The eastward shift signals the beginning of an El Niño.
    \end{psgq}

    \begin{nlz}
        从 during a typical year 定位到文章前半段
    \end{nlz}
\end{blk}
