\section{TPO 01 Passage 1}

\subsection{Words}

\begin{tabular}{lll}

    saturate      & v.   & 使充满;使饱和       \\
    meteoric      & adj. & 流星的;发展迅速的;石质的 \\
    precipitation & n.   & (尤指雨或雪的)降落;降水 \\
\end{tabular}

\subsection{Analyse}

\begin{blk}
    \begin{qst}
        Q4.According to paragraph 2, where is groundwater usually found?
    \end{qst}

    \begin{chc}
        A. Inside pieces of sand and gravel

        B. On top of beds of rock

        C. In fast rivers that are flowing beneath the soil

        D. In spaces between pieces of sediment
    \end{chc}

    \begin{psgq}
        The necessary space is there, however, in many forms. The commonest spaces are those among the particles—sand grains and tiny pebbles—of loose, unconsolidated sand and gravel. Beds of this material, out of sight beneath the soil, are common. They are found wherever fast rivers carrying loads of coarse sediment once flowed. For example, as the great ice sheets that covered North America during the last ice age steadily melted away, huge volumes of water flowed from them. The water was always laden with pebbles, gravel, and sand, known as glacial outwash, that was deposited as the flow slowed down.
    \end{psgq}

    \begin{nlz}
        问的是地下水在哪儿最经常在哪儿发现,找到第二句中的 the commonest spaces: The commonest spaces are those among the particles—sand grains and tiny pebbles—of loose, unconsolidated sand and gravel.最常见的地方是 blablabla,这个 blablabla 就是我们要的答案,罗列的是不同颗粒之间的空隙,所以答案 D 是正确的,答案把原文高度概括。
    \end{nlz}
\end{blk}

\begin{blk}
    \begin{qst}
        Q13. Insert
    \end{qst}

    \begin{chc}
        What, then, determines what proportion of the water stays and what proportion drains away?.
    \end{chc}

    \begin{psgq}
        Much of the water in a sample of water-saturated sediment or rock will drain from it if the sample is put in a suitable dry place. [■]But some will remain, clinging to all solid surfaces. [What, then, determines what proportion of the water stays and what proportion drains away?]It is held there by the force of surface tension without which water would drain instantly from any wet surface, leaving it totally dry. [■]The total volume of water in the saturated sample must therefore be thought of as consisting of water that can, and water that cannot, drain away. [■]
    \end{psgq}

    \begin{nlz}
        待插入句说多少流走多少剩下是什么决定的,所以之前必须得说一部分流走了一部分剩下了,原文最后一句才说到这个,所以 D 是答案。貌似 B 选项之前也说了流走和剩下,但 B 之后有个 it is held there,这个 it 指的是前文的留下来的水,所以与上文过渡紧密,不能插入句子。
    \end{nlz}
\end{blk}


\begin{blk}
    \begin{qst}
        Q14. Summary
    \end{qst}

    \begin{chc}
        A.Sediments that hold water were spread by glaciers and are still spread by rivers and streams.

        B.Water is stored underground in beds of loose sand and gravel or in cemented sediment.

        C.The size of a saturated rock’s pores determines how much water it will retain when the rock is put in a dry place.

        D.Groundwater often remains underground for a long time before it emerges again.

        E.Like sandstone, basalt is a crystalline rock that is very porous.

        F.Beds of unconsolidated sediments are typically located at inland sites that were once underwater.
    \end{chc}

    \begin{nlz}
        Much of the ground is actually saturated with water.

        很大部分地是充满水的。

        A 储存水的沉淀物曾今被冰川扩散并且现在还被河流溪水扩散。

        B 储存在地下的水都处在沙子和沙砾基底或者在坚固的沉淀物中。

        C 当被放置在干燥的地方时,饱和岩石空隙的大小决定了有多少水会被保留。

        D 重新出现之前,地下水会在地下保持很久。

        E 与砂岩一样,玄武岩是一种特别多孔的晶体岩石。

        F 松散沉积物的基地通常位于那些曾今在水下的内陆地点。选项 A 对,第二段“For example, asthe great ice sheetsthat covered …..”提到被冰川扩散;第三段“The same thing happens to this day, …..wherever asediment-laden river or streamemerges from a mountain valley onto relatively flat land,….”

        选项 B 对,第二段讲到必要的空间在沙子和沙砾的微粒间的空隙中;第五段第二句“Consolidated(cemented) sediments, too , contain millions of minute water-holding pores.”

        选项 C 对,最后一段讲到“What happensdepends on pore size.”最后一段也是为了解决倒数第二段提出的问题:什么决定哪部分水留下来,哪部分水流干?

        选项 D 错,这是第一段的一个细节。

        选项 E 错,原文没讲到 basalt 和 sandstone 一样。

        选项 F 错,原文没有提到松散沉积物的基地 typically 都在曾今是水下的陆地。

        因此,答案是:A、B、C;

        这是一个说明型文章,主要讲的是地下水的存在。

        第一段介绍了地下水的定义、来源以及与水循环的关系;

        第二段介绍地下水出现最平常的地方是松散沙石微粒之间空间,而这些松散沙石物质通常存在于携带沉积物的湍流河水曾今流过的地方,为此举了一个冰河时期的例子,冰川融化形成的水流携带沉积物的例子。

        第三段讲了现在的河流湖泊也会携带这样的沉淀物,从山谷流向平地,速度变慢,于是把沉积物沉淀到那儿,形成光滑扇形坡;沉积物在河水入湖口和入海口也会被沉淀,只要水流速度慢下来就行,经常是沉淀在湖底或海床上。

        第四段讲低地几乎每个地方都覆盖了曾今被泥土覆盖的河床,那里的沙石沙砾也会充满地下水。

        第五段开始讲非松散沉积物中储存水,原因是之前微粒的空间并没有完全被粘合化学物质占去,所以部分微粒可能在固化过程中或者之后被穿透的地下水溶化。导致的结果就是砂岩可能和形成它的松散沙子一样多孔。

        第六段说任何沉积物,不管松散还是坚固,都有一部分包含空隙。大多数晶体岩很坚固,但有特例,玄武岩就是一个充满气泡的多孔晶体岩石。

        第七段介绍了孔隙度,还讲了孔隙度与渗透性的差别。

        第八段讲在干燥环境下,沉积物中有些水会留着,有些水会流干,总水量应该是包括两种水。

        第九段讲是空隙的大小尺寸决定了水是否会流干。
    \end{nlz}
\end{blk}
