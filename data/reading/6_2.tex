\section{TPO 6 Passage 2}

\subsection{Words}

\begin{tabular}{lll}
    rudimentary & adj. & 基本的;初步的;粗浅的   \\
    stratum     & n.   & 层;岩层;地层       \\
    quartz      & n.   & 石英            \\
    ubiquitous  & adj. & 普遍存在的,似乎无处不在的 \\
\end{tabular}

\subsection{Analyse}

\begin{blk}
    \begin{qst}
        Q3.Which of the following can be inferred from paragraph 2 about canal building?
    \end{qst}

    \begin{chc}
        A. Canals were built primarily in the south of England rather than in other regions.

        B. Canal building decreased after the steam locomotive was invented.

        C. Canal building made it difficult to study rock strata which often became damaged in the process.

        D. Canal builders hired surveyors like Smith to examine exposed rock strata.
    \end{chc}

    \begin{psgq}
        This was before the steam locomotive, and canal building was at its height. The companies building the canals to transport coal needed surveyors to help them find the coal deposits worth mining as well as to determine the best courses for the canals. This job gave Smith an opportunity to study the fresh rock outcrops created by the newly dug canal. He later worked on similar jobs across the length and breadth of England, all the while studying the newly revealed strata and collecting all the fossils he could find. Smith used mail coaches to travel as much as 10,000 miles per year. In 1815 he published the first modern geological map, “A Map of the Strata of England and Wales with a Part of Scotland,” a map so meticulously researched that it can still be used today.
    \end{psgq}

    \begin{nlz}
        B 以 canal building 做关键词定位至第一句,说在 steam locomotive 出现之前,canal building 达到高潮,也就是说 steam locomotive 出现之后,canal building 的热度开始下降,所以 B 是答案。那些人雇 smith 是帮他们找煤,不是检查暴露的岩层,D 错;A 和 C 都没有相关信息,不选。

        默认B选项没提到而没有去找到出处分析,D选项也没有检查而是直接排除,不够细致
    \end{nlz}
\end{blk}

\begin{blk}
    \begin{qst}
        Q4.According to paragraph2, which of the following is true of the map published by William Smith?
    \end{qst}

    \begin{chc}
        A. It indicates the locations of England's major canals.

        B. It became most valuable when the steam locomotive made rail travel possible.

        C. The data for the map were collected during Smith’s work on canals.

        D. It is no longer regarded as a geological masterpiece.
    \end{chc}

    \begin{psgq}
        This was before the steam locomotive, and canal building was at its height. The companies building the canals to transport coal needed surveyors to help them find the coal deposits worth mining as well as to determine the best courses for the canals. This job gave Smith an opportunity to study the fresh rock outcrops created by the newly dug canal. He later worked on similar jobs across the length and breadth of England, all the while studying the newly revealed strata and collecting all the fossils he could find. Smith used mail coaches to travel as much as 10,000 miles per year. In 1815 he published the first modern geological map, “A Map of the Strata of England and Wales with a Part of Scotland,” a map so meticulously researched that it can still be used today.
    \end{psgq}

    \begin{nlz}
        以 map 做关键词定位至最后一句,但 这句话只说了 map 很好很重要,没有答案,但可以知道 D 说反了;往上看,前句说他走了很远,跟地图无关;再往前看:He later worked on similar jobs across the length and breadth of England, all the while studying the newly revealed strata and collecting all the fossils he could find. 说他边工作边收集了他能找到的所有化石,所以 C 正确;ABD 都没说。

        同上没有挨个校验选项
    \end{nlz}
\end{blk}


\begin{blk}
    \begin{qst}
        Q13
    \end{qst}

    \begin{chc}
        The findings of these geologists inspired others to examine the rock and fossil records in different parts of the world.
    \end{chc}

    \begin{psgq}
        Not only could Smith identify rock strata by the fossils they contained, he could also see a pattern emerging: certain fossils always appear in more ancient sediments,  while others begin to be seen as the strata become more recent. $\blacksquare$By following the fossils, Smith was able to put all the strata of England's earth into relative temporal sequence.   [The findings of these geologists inspired others to examine the rock and fossil records in different parts of the world.]About the same time, Georges Cuvier made the same discovery while studying the rocks around Paris. $\blacksquare$Soon it was realized that this principle of faunal (animal) succession was  valid not only in England or France but virtually everywhere. $\blacksquare$It was actually a principle of floral succession as well, because plants showed the same transformation through time as did fauna. Limestone may be found in the Cambrian or—300 million years later—in the Jurassic strata, but a trilobite—the ubiquitous marine arthropod that had its birth in the Cambrian—will never be found in Jurassic strata, nor a dinosaur in the Cambrian.
    \end{psgq}

    \begin{nlz}
        C 三个过渡点,these geologists,rock and fossil records 还有 different parts of the world,根 these geologists 前面一定得有地质学家,所以 A 和 B 被排除,而且应该先说 different parts of the world,然后再用 England 和 France 做具体例子,所以 C 对 D 错。

        没有遍历所有空位
    \end{nlz}
\end{blk}
