\section{TPO 50 Passage 02}

\subsection{Words}

\begin{tabular}{lll}
    anarchy      & n.   & (政府消亡或垮台而引起的)无政府状态,混乱状态 \\
    dislocation  & n.   & 负面影响,混乱,紊乱              \\
    conspiracy   & n.   & 阴谋;密谋;谋划                \\
    discontent   & n.   & 不满;不满足                  \\
    revolt       & n.   & 反抗;造反;反叛                \\
    republican   & n.   & 拥护共和政体者;共和主义者           \\
    elite        & n.   & (社会)上层集团;掌权人物;出类拔萃的人,精英 \\
    stratum      & n.   & 部分;层;阶层                 \\
    privilege    & n.   & (特定个体或群体的)特权,特别待遇       \\
    resentment   & n.   & 怨恨                      \\
    conservative & adj. & 保守的;守旧的                 \\
    reactionary  & n.   & 反动分子;保守分子;反对进步者         \\
    timid        & adj. & 羞怯的;胆小的,胆怯的             \\
    vacillating  & adj. & 犹豫不决                    \\
    estate       & n.   & 房地产;庄园                  \\
    abrogate     & v.   & 正式废除,废止;撤销              \\
    liberty      & n.   & 自由                      \\
    concession   & n.   & (为结束争端而作出的)认可,让与,让步,妥协  \\
\end{tabular}

\subsection{Collocation}

\begin{tabular}{ll}
    insist on doing sth & (对自己无益,依然)执意做(某事) \\
\end{tabular}

\subsection{Analyse}

\begin{blk}
    \begin{qst}
        Q9. In paragraph 4, why does the author mention that King João’s courtiers “hungered to return to Portugal and their lost estates”?
    \end{qst}

    \begin{chc}
        A. To illustrate how conservative the courtiers were

        B. To help explain the position taken by the courtiers

        C. To give an example of the effects produced by the revolution

        D. To show why King João advised his son the way he did
    \end{chc}

    \begin{psgq}
        Timid and vacillating, King João did not know which way to turn. Under the pressure of his courtiers, \textbf{who hungered to return to Portugal and their lost estates}, he finally approved the new constitution and sailed for Portugal.
    \end{psgq}

    \begin{nlz}
        本题为功能目的题,首先我们来看题干中的这句话:廷臣们“都渴望回到葡萄牙,回到他们所失去的家园”。回到原文当中看,根据上文我们知道若奥六世是个非常优柔寡断的人,不知道如何做选择。而根据下文,他最终批准了新宪法并乘船回到了葡萄牙。那么对于他做出决定的关键性的人物,就是courtiers。句子中说“Under the pressure of his courtiers”,可见,题干中的这个定语是为了强调说明courtiers的地位,他们能够对若奥六世施压,迫使他做出决定,所以B选项正确。 A选项:说明廷臣们有多么的保守。“conservative”一词在上文中有提到,“The Portuguese revolutionaries framed a liberal constitution for the kingdom, but they were conservative or reactionary in relation to Brazil.”但句子的主语是revolutionaries,所以与courtiers没有关系,A排除。 C选项:举例说明革命带来的影响。错误,因为阴影部分内容只是一个定语,不是举例,旨在修饰“courtiers”,所以C排除。 D选项:说明为什么若奥六世在信中建议儿子那样做。根据下文内容,若奥六世在信中劝告佩德罗如果巴西要求独立的话,他应该领导独立运动,但是这和courtiers并没有直接联系。D选项排除。
    \end{nlz}
\end{blk}

\begin{blk}
    \begin{qst}
        Q10. Paragraphs 4 and 5 support the idea that Brazil’s move to declare independence in 1822 was primarily \textbf{the result of}
    \end{qst}

    \begin{chc}
        A. the revolutionaries’ demand that King João return to Portugal

        B. Portugal’s apparent intention to return Brazil to the status of a colony

        C. King João’s decision to leave his son Pedro in Brazil

        D. the growing threat of intervention by the Brazilian masses
    \end{chc}

    \begin{psgq}
        Soon it became clear that the Portuguese parliament intended to set the clock back by abrogating all the liberties and concessions won by Brazil since 1808. One of its decrees insisted on the immediate return of Pedro from Brazil. \textbf{The pace of events moved more rapidly in 1822.} On January 9, urged on by Brazilian advisers who perceived a golden opportunity to make an orderly transition to independence without the intervention of the masses, Pedro refused an order from the parliament to return to Portugal, saying famously, “I remain.” On September 7, regarded by all Brazilians as Independence Day, he issued the even more celebrated proclamation, “Independence or death!” In December 1822, having overcome slight resistance by Portuguese troops, Dom Pedro was formally proclaimed constitutional Emperor of Brazil.
    \end{psgq}

    \begin{nlz}
        本题为细节题,根据关键词“1822”,我们定位到最后一段的第三句话“The pace of events moved more rapidly in 1822.”这句话后面的内容都是在描述1822年之后独立运动的进程,所以其原因肯定在前文中。首先我们来概括第四段的内容,第四段主要描述了1820年的葡萄牙资产阶级革命使巴西与葡萄牙决裂,葡萄牙革命者要求国王若奥六世立刻回到里斯本,结束他所制定的双君主制,并恢复葡萄牙商业垄断。从这段内容中我们可以得知,葡萄牙革命者非常希望能够恢复葡萄牙的统治地位。而第五段的第一句话更是直接地表明了葡萄牙的态度“Soon it became clear that the Portuguese parliament intended to set the clock back by abrogating all the liberties and concessions won by Brazil since 1808.” 明确地表示要“废除巴西自1808年以来所获得的一切自由和特权”。所以这就是促使巴西宣告独立的主要原因——巴西不想再次沦为葡萄牙的殖民地。因此B选项:葡萄牙想要收复巴西使之再度成为葡萄牙殖民地的企图,是正确选项。 A选项:革命者要求若奥六世立刻回到葡萄牙。错误,因为这一点在第四段中已经提到,并且若奥六世已经在1820年回到了葡萄牙,所以这不可能是促使巴西于1822年宣告独立的原因。 C选项:若奥六世将儿子留在巴西的决定。该选项为无关缠绕信息,并不是直接导致巴西宣告独立的原因。 D选项:巴西的群众造成的越来越多的干扰。错误,关键词“intervention”和“masses”出现在最后一段的这句话“On January 9, urged on by Brazilian advisers who perceived a golden opportunity to make an orderly transition to independence without the intervention of the masses, Pedro ……”,这句话是在说巴西幕僚们认为这是一个千载难逢的机会能让巴西在排除干扰的情况下有秩序地过度为独立的王国。但是群众的干扰与巴西宣告独立是无关的,两者不存在因果关系。故排除。
    \end{nlz}
\end{blk}
