\section{TPO 38 Passage 02}

\subsection{Words}

\begin{tabular}{lll}
    transcontinental & adj.        & 横贯大陆的            \\
    coastal          & adj.        & 海岸的              \\
    hemisphere       & n.          & 半球               \\
    surge            & n.v.        & 起伏;上涨;涌出         \\
    indicative       & adj.        & 指示的;象征的          \\
    prosper          & v.          & (使...)成功;繁荣,昌盛   \\
    sustain          & v.          & 承受;支撑;维持         \\
    prairie          & n.          & 大草原              \\
    marsh            & n.          & 湿地               \\
    thrive           & v.          & 兴盛;兴隆            \\
    deliberately     & adv.        & 慎重地;故意地          \\
    restock          & v.          & 重新进货;再储存;补充货源;补足 \\
    presumably       & adv.        & 据推测, 大概, 可能      \\
    fauna            & n.          & (某一地区或某一时期的)动物群  \\
    cereal           & n.          & 谷类植物, 谷物         \\
    abundance        & n.          & 大量, 充足           \\
    exception        & n.          & 例外               \\
    deciduous        & adj.        & (指树木)每年落叶的       \\
    arboreal         & adj.        & 树木的,生活于树上的       \\
    hollow           & adj.n.v.adv & 空的;洞;形成空洞;无用地    \\
    den              & n.v.        & 兽穴;藏到兽穴          \\
\end{tabular}

\subsection{Collocation}

\begin{tabular}{ll}
    all the way & 一路上       \\
    spread out  & 伸展; 延长;分散 \\
    prior to    & 在...之前    \\
\end{tabular}

\newpage

\subsection{Analyse}

\begin{blk}
    \begin{qst}
        Q5. According to paragraph 3, the introduction of raccoons into Utah’s Great Salt Lake Valley appears to have been an example of an introduction that was
    \end{qst}

    \begin{chc}
        A. motivated by a desire to have raccoons among the local wildlife
    \end{chc}

    \begin{psgq}
        Within the United States, they are commonly taken from one area to another, both legally and illegally, to restock hunting areas and, presumably, \textbf{because people simply want them to be part of their local fauna}. Their appearance and subsequent flourishing in Utah’s Great Salt Lake valley within the last 40 years appears to be from such an introduction.
    \end{psgq}

    \begin{nlz}
        because定位到原因
    \end{nlz}
\end{blk}

\begin{blk}
    \begin{qst}
        Q9. According to paragraph 5, what was true about \textbf{raccoons before the arrival of European settlers}?
    \end{qst}

    \begin{chc}
        C. They were not found in most of Canada.
    \end{chc}

    \begin{psgq}
        \textbf{Prior to Europeans settling} and farming the Great Plains regionA vast grassland region in North America extending from central Canada south through the west central United States into Texas, \textbf{raccoons probably were just found along its rivers and streams and in the wooded areas of its southeastern section}.
    \end{psgq}

    \begin{nlz}
        在欧洲殖民者到来并在大平原上耕作之前,raccoons可能只能在大平原的东南部分的河流或者森林里生存。 just表示地方少
    \end{nlz}
\end{blk}

\newpage