\section{TPO 30 Passage 03}

\subsection{Words}

\begin{tabular}{lll}
    rudimentary & adj. & 基础的 \\
\end{tabular}

\subsection{Analyse}

\begin{blk}
    \begin{qst}
        Q3. According to paragraph 2, why did the medieval church need an alarm arrangement?
    \end{qst}

    \begin{chc}
        A. The alarm warned the monks of \textbf{discord or strife} in the town.

        (A. 警报提醒僧侣们镇子里有不和或纷争)

        D. One of the church’s daily rituals occurred during the night.
    \end{chc}

    \begin{psgq}
        Medieval Europe gave new importance to reliable time. The Catholic Church had its seven daily prayers, one of which was at night, requiring \textbf{an alarm arrangement} to waken monks before dawn.
    \end{psgq}

    \begin{nlz}
        正确答案D,对应原文的第二句话,可根据alarm arrangement定位,前文说one of which was at night,后面说to waken monk before dawn,这两个都可以算是原因,选项中符合的只有D。
    \end{nlz}

    \begin{qst}
        Q12. According to paragraph 6, how did the \textbf{mechanical clock affect labor}?
    \end{qst}

    \begin{chc}
        A. It encouraged workers to do more time-filling busywork.

        B. It enabled workers to be more task oriented.

        C. It pushed workers to work more hours every day.

        D. It led to a focus on productivity.
    \end{chc}

    \begin{psgq}
        The clock brought order and control, both collective and personal. Its public display and private possession laid the basis for temporal autonomy: people could now coordinate comings and goings without dictation from above. The clock provided the punctuation marks for group activity, while enabling individuals to order their own work (and that of others) so as to enhance productivity. \textbf{Indeed}, the very notion of productivity is a by-product of the clock: once one can relate performance to uniform time units, work is never the same. One moves \textbf{from the task-oriented}\textsubscript{BX} time consciousness of the peasant (working one job after another, as time and light permit) and the time-filling busyness of the domestic servant (who always had something to do) \textbf{to an effort to maximize product per unit of time}\textsubscript{D}.
    \end{psgq}

    \begin{nlz}
        正确答案D,定位到原文最后一句,说使人们从task-oriented和 time-filling busyness的模式变为maximize product per unit of time,从而提高了productivity,所以选择D。A,B原文都提到了,但是是转化前的状态,C没提到。
    \end{nlz}

    \begin{qst}
        Q13. insert

        \textbf{The division of time} no longer reflected the organization of religious ritual.
    \end{qst}

    \begin{psgq}
        Ironically, the new machine tended to undermine Catholic Church authority. Although church ritual had sustained an interest in timekeeping throughout the centuries of urban collapse that followed the fall of Rome, church time was nature’s time. $\blacksquare$ Day and night were divided into the same number of parts, so that except at the equinoxes, day and night hours were unequal; and then of course the length of these hours varied with the seasons. $\blacksquare$ But the mechanical clock kept equal hours, and this implied a new time reckoning. $\blacksquare$ The Catholic Church resisted, not coming over to the new hours for about a century. $\blacksquare$ From the start, however, the towns and cities took equal hours as their standard, and the public clocks installed in town halls and market squares became the very symbol of a new, secular municipal authority. Every town wanted one; conquerors seized them as especially precious spoils of war; tourists came to see and hear these machines the way they made pilgrimages to sacred relics.
    \end{psgq}

    \begin{nlz}
        C 原句中有个关键短语division of time,这个短语出现在插入句句首,且用定冠词the修饰,说明前文必定提到过。而句中否定了reflected the organization of religious ritual。说明前文也应该提到过这个。符合这个特点的选项就只有B和C,而在这两个选项中,明显C选项更为合适,因为B选项之前还没有开始否定,还没开始转折。放在B非常突兀。而C选项之前这句话用but进行转折,放在C正好顺着这个方向发展。
    \end{nlz}
\end{blk}
