\section{TPO 54 Passage 1}

\subsection{Words}

\begin{tabular}{lll}
    durable    & adj. & 持久的,耐久的            \\
    converge   & v.   & (几条线或道路)会合,交会,汇合   \\
    facilitate & v.   & 促进;促使;使便利          \\
    liberate   & v.   & 解放,使自由             \\
    tortuous   & adj. & 弯弯曲曲的;曲折的;转弯抹角的    \\
    stretch    & v.   & 伸出;伸长;拉伸           \\
    enterprise & n.   & 组织;(尤指)公司,企业;计划,事业 \\
\end{tabular}

\subsection{Collocation}

\begin{tabular}{ll}
    along with & 赞同;支持;和(某人)观点一致 \\
\end{tabular}

\subsection{Analyse}

\begin{blk}
    \begin{qst}
        Q4.
        What can be inferred from paragraph 2 about timber in America before the year 1860?
    \end{qst}

    \begin{chc}
        A.
        Farmers of the American West earned most of their income by selling timber to newly arrived settlers.

        B.
        Timber came primarily from farmers who wished to supplement their income.

        C.
        Timber was much more expensive before the year 1860 because it was less readily available.

        D.
        Timber came primarily from large manufacturing companies in the East.
    \end{chc}

    \begin{psgq}
        By 1860, the settlement of the American West along with timber shortages in the East converged with ever-widening impact on the pine forests of the Great Lakes states. Over the next 30 years, lumbering became a full-fledged enterprise in Michigan, Wisconsin, and Minnesota. Newly formed lumbering corporations bought up huge tracts of pineland and set about systematically cutting the trees. Both the colonists and the later industrialists saw timber as a commodity, but the latter group adopted a far more thorough and calculating approach to removing trees. In this sense, what happened between 1860 and 1890 represented a significant break with the past. No longer were farmers in search of extra income the main source for shingles, firewood, and other wood products. By the 1870s, farmers and city dwellers alike purchased forest products from large manufacturing companies located in the Great Lakes states rather than chopping wood themselves or buying it locally.
    \end{psgq}

    \begin{nlz}
        本题为推理题。根据题干很难准确定位,故由选项到原文中找是否有对应句子。 A选项关键词“farmers’ income”定位到“No longer were farmers in search of extra income the main source for shingles, firewood, and other wood products. ”,可以知道farmers在1860之前需要靠木产品来增加收入,并未提到newly arrived settlers,多了原文没有的信息,故排除。 B选项:Timber came primarily from farmers who wished to supplement their income. 依旧可以对应A句的原文,翻译为“到1870年代,农民不再寻找额外收入,这是木瓦、木柴和其他木制品的主要来源。”可以推断出木材在1870前主要来源于农民想要增加自己的收入。故B为正确选项。 C选项:Timber was much more expensive before the year 1860 because it was less readily available. 原文中没有提到木材在1860年代前的价格,也没有说其不好获得。故排除。 D选项:Timber came primarily from large manufacturing companies in the East. 原文中没有提及,且与B选项冲突,因为已经可推断出木材在1870前主要来源于农民想要增加自己的收入,因而D是错的。
    \end{nlz}
\end{blk}

\begin{blk}
    \begin{qst}
        Q14. Summary
    \end{qst}

    \begin{chc}
        \textbf{A. During the nineteenth century, lumbering became a large-scale industry controlled by manufacturing companies rather than a local enterprise controlled by farmers.}

        B. Technological advances, including the use of steam power, led to increased productivity, efficiency, and commercialization of the lumbering industry.

        C. Seasonal changes and severe winters made the development and laying of track for logging railroads slow and difficult.

        \textbf{D. After 1860 farmers continued to be the main suppliers of new timber, but lumbering companies took over its transport and manufacture into wood products.}

        E. The invention of new technology, such as band saws, allowed American lumbering companies to make a profit by exporting surplus lumber to Britain and other countries.

        F. New methods for transporting logs to mills helped transform lumbering from a seasonal activity to a year-round activity.
    \end{chc}

    \begin{nlz}
        本文为总结题。我们依次分析选项找出正确答案: 所给提示意思为:在19世纪木材运输的需求增加。

        A选项对应文章第二段,文中说“farmers and city dwellers alike purchased forest products from large manufacturing companies located in the Great Lakes states rather than chopping wood themselves or buying it locally.” 说明了manufacturing companies 取代了当地个人业,A选项正确。

        B选项对应文章第3段。文章提到 “streamlined production by allowing for the more efficient, centralized, and continuous cutting of lumber.”并提及了蒸汽工厂,新型锯子等,符合B选项。

        C选项说季节变化和严峻冬天使发展和铺铁路变得又慢又困难,实际上是因为冬天不够严寒而导致需要铺铁路,所以是逻辑错误,因而不选。

        D选项说1860年后农民任然是新木材的主要供应商,但其实是不对的,原文说“No longer were farmers in search of extra income the main source for shingles, firewood, and other wood products. By the 1870s, farmers and city dwellers alike purchased forest products from large manufacturing companies located in the Great Lakes states rather than chopping wood themselves or buying it locally.”说明农民不再是供应商,而全权交给了企业,所以D错误。

        E选项原文中跟没提到与国外的木材贸易,只是根据其他国家的新科技工具随意推测,所以是错的。

        F选项对应第五段,对应原文“By 1887, 89 logging railroads crisscrossed Michigan, transforming logging from a winter activity into a year-round one.”说明了这种新的运输方式不再使得伐木收到季节限制,符合原文,故F对。
    \end{nlz}
\end{blk}
