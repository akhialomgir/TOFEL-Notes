\section{TPO 42 Passage 03}

\subsection{Words}

\begin{tabular}{lll}
    deform      & v.   & (使)变形,(使)扭曲          \\
    outermost   & adj. & 最外层的;(距中心)最远的        \\
    crater      & n.   & 火山口;(火山口似的)坑         \\
    sparse      & adj. & 稀少的;稀疏的,零落的          \\
    terrestrial & adj. & 地球的;与地球有关的;(星球)类似地球的 \\
    molten      & adj. & 熔化的,熔融的              \\
    tectonic    & adj. & 地壳构造的;非常重要的;具有重大影响的  \\
    divergence  & n.   & 差异;分歧                \\
    tidal       & adj. & 有潮的;受潮汐影响的           \\
    convulsion  & n.   & 惊厥;抽搐,痉挛             \\
\end{tabular}

\subsection{Collocation}

\begin{tabular}{ll}
    be to blame for & 责备;责怪;归咎于 \\
\end{tabular}

\newpage

\subsection{Analyse}

\begin{blk}
    \begin{qst}
        Q11. According to paragraph 6, the differences in how Callisto and Ganymede evolved are most probably due to differences in their
    \end{qst}

    \begin{chc}
        A. size and internal heating

        B. distance from Jupiter
    \end{chc}

    \begin{psgq}
        Why is Ganymede different from Callisto? Possibly the small difference in size and internal heating between the two led to this divergence in their evolution. \textbf{But} more likely \textbf{the gravity of Jupiter is to blame} for Ganymede’s continuing geological activity. \textbf{Ganymede is close enough to Jupiter} that tidal forces from the giant planet may have episodically heated its interior and triggered major convulsions on its crust.
    \end{psgq}

    \begin{nlz}
        有转折,所以选B
    \end{nlz}
\end{blk}

\newpage