\section{范文 1}

\subsection{Example}

I agree that providing financial incentives and improving infrastructure can attract more people to rural areas. However, enhancing quality of life factors is equally important. If governments want citizens to relocate willingly to rural villages, they must invest in education, healthcare, andcommunity development.

Rural living is appealing to many due to lower costs of living and closer communitybonds. Yet, limited access to essential services and amenities remain deterrents. Governments should provide funding for new schools, medical clinics, recreation centers and other resources that strengthen community life. Strong, thriving communities where citizens have opportunities to learn, grow, and lead healthy, purposeful lives will draw people naturally.

Simply subsidizing agricultural careers or reducing living costs is not enough. People seek fulfilling lives surrounded by supportive relationships and access to life's necessities regardless of location. Investing in the growth and well-being of citizens and communities should be the priority. With vibrant communities that nurture citizens, development in meaningful ways, relocatingtorural areas may be an attractive choice for more.

\subsection{Discussion Writing}

I agree with the idea that providing fininical support and entertaining infrastruture might be helpful to attract more people to the rural areas. However, in my point of view, addressing the fundamental service is also a topic to consider. If the governments want the citizens to relocate to rural towns, they should invest in both healthcare and education service.

As we all know, living in rural areas means a better environment, in which we are away from the sress and pollution in the morden cities. Nevertheless, limited access to eduaction remains disappointing. For instance, it is really a waste of time for a student to spend 3 hours on reaching the school. Thus government should provide funding for new school nearby.

\section{范文 2}

\subsection{Example}

I believe that grading students based on both performance and effort is a more balanced and comprehensive approach. While performance is a important indicator of understanding and master yof the subject matter, effort should not be disregarded.Recognizing and rewarding hardworkencourages students to develop perseverance,discipline, and a strong work ethic, whicharevaluable qualities for success in both academic and professional settings.

Moreover, grading based solely on performance can overlook the individual circumstancesand challenges that students may face. Some students may put insignificant effort despite personal or learning difficulties, and it is important to acknowledge their dedication and determination. Byconsidering effort alongside performance, teachers can provide a more accurate reflectionof astudent's overall growth and progress.

Additionally, evaluating effort can motivate students to continue striving for improvement.tinstills a sense of intrinsic motivation and self-confidence, as studentsrealize that their hard workisvalued and recognized, even if they haven't achieved the highest grades. In conclusion, combiningperformance and effort in grading allows for a more holistic assessment of students' abilities and potential. I promotes fairness, acknowledges individual circumstances, and cultivates essential
qualities for future success.

\subsection{Discussion Writing}

In my point of view, i believed that grading based on both performance and effort is a more reasonable approach.

To begin with, there is no doubts that grading is a system to qualify the actual level of the students. If the effort is the only factor to be taken in account. It is really for teachers and schools to evaluate whether the students are qualified for the leason. So they might overlook the ability of them and the annual goal is hardly to reach.

On the other hand, if the grading is solely based on the performance, it is hard to stimulate both the low-grade students and the high-grade students. To the low-grade ones, always getting lower grade can give harm to their self-confidence. Thus they can not have the motivation to improve themselves.
