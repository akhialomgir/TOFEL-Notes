\section{TPO 30}

\subsection{Integrated Writing 1}

In the reading section, the author holds the view that the burning mirror should not be an actual weapon with which Greek port city defend themselves from the Roman's attack.However, the speaker disagree with this idea and disprove it mainly with three points.

To begin with, the author hold the point that the Greek haven't the technology to build mirror.<<< However, the speaker says that they don't have to build such a mirror that is about several meters wide. On the contrary, they only need to build the single sheet mirror. Instead, they can add small individual mirrors together into a sheet so that they can make a multi-sheet mirror, which can work as the same fuction.

Moreover, the author point out that it takes about 10 minutes to set a fire on a wooden boat 30 meters away.<<< But the fact is that they don't need to fire the wooden materials. In fact, they only need to set a fire on the pitch, which is much easier to fire, only taking seconds.<<So

Last but not least, the author hold the point that arrows are much effective tools than the burning mirror. However, the speaker point out that arrows are easiler to be found by the enermy. On the contrary, when using the burning mirror. The solders can't see the fire rise. So they are unable to put it off too.

\subsection{Example}

In the reading material, the author raises three strong arguments to cast doubt on the existence of “burning mirror” in ancient Greek port city. However the lecturer states that the burning mirror might have once appeared in history by reputing the writer’s arguments one by one.

First, the author argues that it was technologically impossible to build such a tremendous mirror by a single sheet of copper in ancient Greece. Yet the speaker shows by \textbf{experiment that dozens of small polished copper may do the same job efficiently} and Greek \textbf{mathematicians} were excellent enough to accomplish it perfectly.

Besides, in view of the author, it might take a long time for the burning mirror to set a wooden ship on fire \textbf{and the enemy wouldn’t be stupid to stay still for ten minutes. The lecturer, however, points out those ancient warships were not built by wood only.} A special material called pitch was used to fill the spaces between and pitch catches fire quickly. \textbf{Thus it was possible for a burning mirror to burn the pitch first then the ship. In this way can a burning mirror be an effective weapon.}

Lastly, the author argues that the burning mirror is less effective and flexible than flaming arrows. \textbf{Thus ancient Greek won’t bother to devote much effort to build such a useless weapon. }Nonetheless, the lecturer argues that since flaming arrows are usual weapons used so much, Roman soldiers were \textbf{familiar with them and knew how to protect the ship from burning by flaming arrows.} But on the other hand, it was extremely hard to predict where the burning mirror would fire the ship. Thus it is more effective to use a burning mirror than flaming arrows.

\subsection{Integrated Writing 2}

